%% GABARIT POUR MÉMOIRE STANDARD
%%
%% Consulter la documentation de la classe ulthese pour une
%% description détaillée de la classe, de ce gabarit et des options
%% disponibles.
%%
%% [Ne pas hésiter à supprimer les commentaires après les avoir lus.]
%%
%% Déclaration de la classe avec le type de grade
%%   [l'un de MATDR, MArch, MA, LLM, MErg, MMus, MPht, MSc, MScGeogr,
%%    MServSoc, MPsEd]
%% et les langues les plus courantes. Le français sera la langue par
%% défaut du document.
\documentclass[MSc,english,french]{ulthese}
  %% Encodage utilisé pour les caractères accentués dans les fichiers
  %% source du document. Les gabarits sont encodés en UTF-8. Inutile
  %% avec XeLaTeX, qui gère Unicode nativement.
  \ifxetex\else \usepackage[utf8]{inputenc} \fi

  %% Charger ici les autres paquetages nécessaires pour le document.
  %% Quelques exemples; décommenter au besoin.
  \usepackage{amsmath}       % recommandé pour les mathématiques
  \usepackage{icomma}        % gestion de la virgule dans les nombres
  \usepackage{amssymb}
  %% Utilisation d'une autre police de caractères pour le document.
  %% - Sous LaTeX
  \usepackage{tgpagella}      % texte et mathématiques en Palatino
  % \usepackage{mathptmx}      % texte et mathématiques en Times
  %% - Sous XeLaTeX
  %\setmainfont{TeX Gyre Pagella}      % texte en Pagella (Palatino)
  %\setmathfont{TeX Gyre Pagella Math} % mathématiques en Pagella (Palatino)
  %\setmainfont{TeX Gyre Termes}       % texte en Termes (Times)
  %\setmathfont{TeX Gyre Termes Math}  % mathématiques en Termes (Times)

  %% Options de mise en forme du mode français de babel. Consulter la
  %% documentation du paquetage babel pour les options disponibles.
  %% Désactiver (effacer ou mettre en commentaire) si l'option
  %% 'nobabel' est spécifiée au chargement de la classe.
  \frenchbsetup{%
    StandardItemizeEnv=true,       % format standard des listes
    ThinSpaceInFrenchNumbers=true, % espace fine dans les nombres
    og=«, fg=»                     % caractères « et » sont les guillemets
  }

  %% Style de la bibliographie.

  \titre{Méthodes neuronales de Monte-Carlo pour la génération automatique de résumés de textes}
  \auteur{Mathieu Godbout}
  \programme{Maîtrise en informatique}
  \codirection{Luc Lamontagne, codirecteur de recherche \\
              Audrey Durand, codirectrice de recherche}


\usepackage{tikz}
\usepackage{pgfplots}
\usepackage{pgfplotstable}
\usepgfplotslibrary{colorbrewer}
\pgfplotsset{compat=1.14, colormap/Blues, every axis/.append style={label style={font=\large}, tick label style={font=\large}}}

\usepackage{mathtools}
\DeclarePairedDelimiter{\ceil}{\lceil}{\rceil}

\usepackage{xcolor}
\newcommand\todo[1]{\textcolor{red}{TODO: #1}}
\newcommand\question[1]{\textcolor{blue}{Question: #1}}
\newcommand\commentaire[1]{\textcolor{violet}{Commentaire: #1}}
\newcommand\ngrams{\textit{n}-grammes }

\usepackage{mathtools}

\begin{document}

\frontmatter                    % pages liminaires

\frontispice                    % production de la page frontispice

\chapter*{Résumé}               % ne pas numéroter
\label{chap:resume}             % étiquette pour renvois
\phantomsection\addcontentsline{toc}{chapter}{\nameref{chap:resume}} % inclure dans TdM

\begin{otherlanguage*}{french}

Ce mémoire aborde l'utilisation des méthodes par bandit pour résoudre 
la problématique de l'entraînement de modèles de générations de résumés extractifs.
Les modèles extractifs, qui bâtissent des résumés en sélectionnant des phrases d'un 
document original, sont difficiles à entraîner car le résumé cible 
correspondant à un document n'est habituellement pas constitué de 
manière extractive.
C'est à cet effet que l'on propose de voir la production de résumés
extractifs comme différents problèmes de bandit, lesquels sont 
accompagnés d'algorithmes pouvant être utilisés pour l'entraînement.

On commence ce document en présentant BanditSum, une approche tirée 
de la litérature et qui voit la génération des résumés 
d'un ensemble de documents comme un problème de bandit 
contextuel.
Ensuite, on introduit CombiSum, un nouvel algorithme 
qui formule la génération du résumé d'un 
seul document comme un bandit combinatoire.
En exploitant la formule combinatoire,
CombiSum réussit à incorporer la notion du potentiel 
extractif de chaque phrase à son entraînement.
Enfin, on propose LinCombiSum, la variante linéaire de CombiSum 
qui exploite les similarités entre les phrases d'un document 
et emploie plutôt la formulation en bandit linéaire combinatoire.

\end{otherlanguage*}
                % résumé français
\chapter*{Abstract}             % ne pas numéroter
\label{chap:abstract}           % étiquette pour renvois
\phantomsection\addcontentsline{toc}{chapter}{\nameref{chap:abstract}} % inclure dans TdM

\begin{otherlanguage*}{english}
  This thesis discusses the use of bandit methods to solve 
the problem of training extractive abstract generation models.
The extractive models, which build summaries by selecting sentences from an 
original document, are difficult to train because the target summary of a document is usually not built in an extractive way.
It is for this purpose that we propose to see the production of an extractive summary
as different bandit problems, for which there exist algorithms that can be used for training.

In this document, we first see how the generation of the summary 
of any document can be formulated as a contextual bandit problem
to achieve excellent computational efficiency.
Next, we propose to see the generation of the abstract 
of a single document like a combinatorial bandit.
We present a method for obtaining targets 
for training a model from the resolution 
of the combinatorial bandit.
We present two algorithms, UCB and LinUCB, which can 
be used for this purpose.
UCB, the first algorithm presented, is characterized by 
its universality while LinUCB is computationally
more efficient, at the added cost of an assumption of the existence 
a linear relationship between a sentence and the quality of abstracts 
it produces. 
\end{otherlanguage*}
              % résumé anglais
\cleardoublepage

\maxtocdepth{subsection}
\tableofcontents                % production de la TdM
\cleardoublepage

\listoftables                   % production de la liste des tableaux
\cleardoublepage

\listoffigures                  % production de la liste des figures
\cleardoublepage

\dedicace{<Dédicace si désiré>}
\cleardoublepage

\epigraphe{"<Texte de l'épigraphe>"}{<Source ou auteur>}
\cleardoublepage

\chapter*{Remerciements}        % ne pas numéroter
\label{chap:remerciements}      % étiquette pour renvois
\phantomsection\addcontentsline{toc}{chapter}{\nameref{chap:remerciements}} % inclure dans TdM

Je tiens à remercier mon directeur de recherche, Luc Lamontagne,
pour ses précieux conseils tout au long de mon parcours à la maîtrise.
Sa bienveillance et sa franchise m'auront permis de progresser non 
seulement en tant que chercheur, mais aussi en tant que personne.
Un merci tout particulier à Audrey Durand, ma co-directrice, pour m'avoir initié 
au passionnant domaine des bandits et de l'apprentissage par renforcement.
À travers nos interactions, j'ai découvert tout le plaisir 
qui se cache derrière le processus scientifique bien fait et je suis
impatient de continuer notre collaboration.
Par leurs efforts conjoints, Luc et Audrey m'ont permis de me 
dépasser et de bâtir mon propre bagage scientifique.

Je ne peux passer sous le silence l'incroyable communauté de recherche 
du Groupe de Recherche en Apprentissage Automatique 
de l'université Laval (GRAAL) dans laquelle j'ai eu la chance de compléter 
ma maîtrise.
Tantôt collègues de travail, tantôt partenaires de célébrations, les 
personnes que j'ai rencontrées au GRAAL ont grandement contribué 
à faire de ma maîtrise une période de ma vie que je chérirai toujours.
Merci à David, Jean-Thomas, Nicolas, Jean-Samuel, Frédérik, Gaël
et tous les autres.

Un merci spécial va aussi à mes parents, Martine et Alain, qui, avec leur éternelle confiance en 
mes capacités et leur dévouement à mon épanouissement ont rendu possible
mon cheminement scolaire et personnel.
Je conserve le dernier de mes remerciements pour Samantha, sans qui je ne serais pas 
l'ombre de la personne que je suis aujourd'hui. 
Partenaire de mes joies comme de mes peines, elle a toujours su faire ressortir 
la meilleure version de moi-même.
Par son écoute, ses questions intéressées et son soutien incomparable,
elle a certainement participé à l'écriture de ce mémoire dans une bien plus grande
proportion qu'elle ne peut l'imaginer.

Je remercie aussi le Conseil de Recherches en Sciences Naturelles et en Génie du Canada
(CRSNG), le Fonds de recherche du Québec — Nature et Technologies (FRQ-NT)
et Intact~Corporation~financière pour leur soutien financier ayant permis l'élaboration de ce document.         % remerciements
\include{avantpropos/avantpropos}           % avant-propos

\mainmatter                     % corps du document

% !TeX root = ./main.tex
\chapter*{Introduction}
\label{chap:introduction}       % étiquette pour renvois

\question{
    À quel point est-ce que l'introduction doit déboucher exactement sur mon mémoire ?
    J'ai l'impression qu'à la fin de mon intro, le lecteur devrait être en contexte
    par rapport au titre, i.e. il devrait savoir qu'on va parler de (1) résumés, (2)
    méthodes de MC et (3) que tout ça va être fait avec des NN.
}

Ère du big data: on a une quantité immense de données, notamment textuelles.

Succès des dernières décennies nous montrent le potentiel qui peut se cacher
derrière une utilisation judicieuse de ces données.

Or, il n'est pas toujours possible d'incorporer toutes les données dans 
un processus de traitement.

Notamment, dans certains domaines, les humains sont encore requis dans la loop 
dans des applications de traitement de données textuelles en raison de 
législations ou d'assurance qualité (i.e. processus de réclamations en assurance).

Comment faire profiter du grand afflux de données dans des systèmes où l'humain (et 
sa capacité de traitement limitée) sont essentiels ?

Une option qui peut s'avérer facilement attrayante est l'idée de condenser un ou plusieurs
documents en un texte concis contenant seulement l'information requise pour procéder à la 
tâche à exécuter.

On dit alors que l'on fait appel à un système de génération automatique de résumés de textes.

Comment fonctionnent ces systèmes de génération de résumés ?

L'espace des résumés possibles est très vaste; comment faire pour en trouver un qui est
suffisamment bon dans un délai de temps assez raisonnable pour l'utilisation dans un contexte
du monde réel ?

Des outils probabilistes, connus sous le nom de méthodes de Monte-Carlo, ont été
spécifiquement conçus pour de tels contextes.

Ils se basent sur l'aléatoire pour parcourir efficacement des espaces potentiellement
très grands.

En les jumelant à des approches neuronales ayant démontré largement leur efficacité sur des données
textuelles, il est naturel de s'attendre à ce que ces outils permettent d'obtenir des 
systèmes de génération automatique qui produisent non seulement d'excellents résumés, mais qui
ne le font pas au détriment d'un temps d'entraînement déraisonnable.

\section*{Objectifs}

Dans ce mémoire, on explorera d'abord comment différentes méthodes de Monte-Carlo
peuvent être utilisées pour estimer des mesures liées à la qualité de résumés de textes
dont le calcul est difficile à première vue.

On verra ensuite comment ces approximations obtenues par des approches Monte-Carlo
peuvent être utilisées dans un système complet de génération de résumés.

Les différents systèmes explorés seront enfin comparés aux approches représentant
l'état de l'art en génération de résumés en fonction de leur efficacité d'entraînement
et de prédiction ainsi que selon la qualité des résumés produits.

\section*{Structure du mémoire}

Ce mémoire sera divisé en trois chapitres pour les trois méthodes de Monte-Carlo explorées:
échantillonnage, recherche arborescente et recherche arborescente linéaire.

Pour chacun des chapitres, on posera d'abord une fonction en lien avec la génération de résumé
qui est difficile à évaluer.

On montera ensuite comment une méthode Monte-Carlo
peut être utilisée pour l'approximer.

On évaluera empiriquement la rapidité
de la convergence de la méthode proposée.

On proposera ensuite un modèle neuronal
de génération automatique de textes utilisant la méthode MC à l'essai.

Des expériences seront ensuite effectuées pour évaluer la performance du modèle.

Les hyperparamètres de la portion MC seront choisis en fonction des tests
empiriques préliminaires.
% !TeX root = ./main.tex
\chapter{Prérequis}     % numéroté
\label{chap:montecarlo}                   % étiquette pour renvois (à compléter!)

Tout ceci est seulement en place à titre indicatif.
Toute information pertinente sera insérée ou retirée en fonction
du développement des chapitres de corps du mémoire.

\section{Méthodes de Monte-Carlo}

Approximation statistique de procédés déterministes

\subsection{Estimation de Monte-Carlo}

\subsection{Recherche arborescente de Monte-Carlo}

\section{Réseaux de neurones}

\subsection{Réseaux pleinement connectés}

Présentation haut niveau de leur utilité prouvée empiriquement
sur à peu près n'importe quel type de problème d'apprentissage
supervisé.
\begin{itemize}
    \item Préciser architecture : neurone, activation
    \item Préciser fonction de perte : doit être intimement connectée
          au savoir préalable (prior) sur la tâche et la cible.
    \item Optimisation : taux d'apprentissage, Adam vs SGD
\end{itemize}

\subsection{Réseaux récurrents}

\begin{itemize}
    \item RNN
    \item LSTM : pour quelle raison ?
\end{itemize}
\chapter{Génération automatique de résumés de textes}     % numéroté
\label{chap:generation_resumes}                           % étiquette pour renvois (à compléter!)

Quand on résume, on veut (1) compresser un ou des textes tout en
s'assurant de (2) conserver la majeure partie de l'information contenue.

On dit qu'il s'agit d'un dilemme compression-conservation.

Deux techniques principales actuellement utilisées: extractive et abstractive.

Nous nous intéresserons seulement au cas de la génération de résumés à partir
d'un seul document pour lequel nous possédons un seul résumé cible.

Notons que le cas où on génère un résumé à partir de plusieurs documents se ramène
naturellement au cas d'un seul document en considérant un nouveau document
représentant la concaténation de tous les documents en entrée.

\section{Formulation extractive}

On sélectionne des phrases du document initial.

Formulation simple à résoudre: la nombre de résumés dépend maintenant seulement du nombre
de phrases et on s'évite toute les difficultés en termes de syntaxe et de cohérence
d'avoir à générer du texte.

Les approches de l'état de l'art sont toutes basées sur une approche
encodeur-décodeur avec des réseaux de neurones.

Ces approches produisent en sortie une distribution sur les phrases
originales.

Le résumé est ensuite bâti à partir de la distribution prédite par le modèle.

Définition formelle: On dit qu'on a un modèle de génération de résumés
$(\pi, \phi)$.
Ici $\pi$ prend en entrée un document $d$ et retourne
$\pi(d)$, une distribution de probabilités de sélection sur les phrases
de $d$.
On a ensuite $\phi$, qui est un processus de génération de résumé
extractif à partir d'une distribution $\pi(d)$.
Deux exemples intuitifs de $\phi$ sont les processus voraces et stochastiques,
où on choisit les $n$ phrases avec la plus grande probabilité ou on en pige
$n$ sans remise de $\pi(d)$, respectivement.

Note: le modèle doit fondamentalement contenir $\phi$ et $\pi$ car
l'encoder-décodeur optimal $\pi^*$ dépend du processus $\phi$ employé
pour la génération de résumés.

\subsection{Approches supervisées}

La majorité des approches extractives sont supervisées.

Comme le résumé correspondant à un document n'est habituellement pas obtenu de
manière extractive (i.e. le résumé n'est pas une combinaison de phrases du document initial),
les approches supervisées doivent utilisent des cibles basées sur des heuristiques pour
leur entraînement.

Par exemple, \citep{10.5555/3298483.3298681} sélectionnent de manière vorace
des phrases une à la fois en fonction de leur similarité à un résumé pour
un document.

Après avoir sélectionné trois phrases, leur processus est arrêté et la cible pour le
document est fixée à un vecteur binaire où les index des trois phrases ont une valeur
de 1 et les autres index ont une valeur nulle.

Comme on est supervisé, le réseau entraîné ne peut être au mieux aussi bon que l'heuristique
utilisée pour générer les cibles utilisées.

La performance est tout de même très bonne; \citep{zhong-etal-2020-extractive} est le SOTA
actuel en extractive.

\subsection{Approches par renforcement}

\citep{dong2018banditsum,luo-etal-2019-reading}
sont des approches représentant l'état de l'art en la matière actuellement.

Au lieu d'utiliser des cibles binaires, les approches par renforcement visent à
optimiser directement une mesure de la similarité entre deux résumés: le ROUGE \citep{lin-2004-rouge}
(plus de détails plus bas).

\citep{dong2018banditsum,luo-etal-2019-reading} emploient le
même modèle de réseau de neurones.
Comme encodeur, deux LSTMs consécutifs, un travaillant au niveau
des mots et l'autre au niveau des phrases.
Comme décodeur, une couche pleinement connectée partagée pour toutes les
phrases et avec sortie réelle.
Un schéma et davantage de détails sur ce modèle neuronal sont disponibles
au chapitre \ref{chap:mcs}.
Nous reprendrons une architecture similaire de réseau de neurones pour toutes
les expérimentations.

Bémol à insérer : \citep{DBLP:journals/corr/PaulusXS17} entraînent par
renforcement sur la formulation abstractive mais finissent par ne pas
utiliser le modèle avec le meilleur score ROUGE.

\section{Formulation abstractive}

On écrit un résumé \textit{à la mitaine}.

Formulation difficile car nécessite de gérer la syntaxe et les fautes
d'orthographe en plus de la gestion du dilemme compression-conversation.

Récentes percées en NLG ont beaucoup boosté les performances ici.

\citep{2020t5, unilm, zhang2019pegasus} sont le SOTA en abstractif.

\section{Évaluation de la performance}

Comment distinguer quantitativement deux résumés candidats $s$ et $\hat{s}$ ?

Intuition: on se base sur les \ngrams d'un résumé cible. Plus le overlap
est grand entre les \ngrams d'un candidat et ceux de la cible, plus le résumé
a conservé l'information recherchée.

Un problème: si on se fie juste au rappel sur les \ngrams, le résumé optimal sera
de conserver tout le texte original. Pour incorporer la portion compression du dilemme
fondamental de la génération de résumé, on peut utiliser une métrique plus
appropriée comme le score F1 pour ajouter une pénalité sur les \ngrams présents dans
le candidat pas dans la cible.

Les considérations précédentes ont mené à la famille de métriques ROUGE \citep{lin-2004-rouge}.
Le score ROUGE correspond à assigner une valeur numérique à un résumé candidat à partir
de son F1-Score sur une tâche de classification sur les \ngrams d'un résumé cible.
En pratique, on utilise généralement $n={1,2}$ et on utilise aussi la plus longue
sous-séquence, nommée ROUGE-L.
On pose généralement

\begin{equation}
    \label{eq:ROUGE}
    R(s, \hat{s}) := \frac{1}{3} R_1(s, \hat{s}) + R_2(s, \hat{s}) + R_L(s, \hat{s}).
\end{equation}

On définit enfin la similarité entre un résumé candidat $s$ à partir d'un résumé cible $\hat{s}$
comme étant $R(s, \hat{s})$.

Une propriété intéressante du score ROUGE est son invariance à
l'ordre des phrases.

\todo{expliquer mérites et défauts de R1, R2 et RL et donc
    pourquoi leur moyenne est bonne. Range de R est [0,1]}

Benchmark classique: CNN/DailyMail \citep{DBLP:journals/corr/SeeLM17}.

À ajouter :

\begin{itemize}
    \item N articles (n train/valid/test)
    \item Moyennes mots, phrases texte/abstract
\end{itemize}

En fonction des stats, les modèles entraînés sur ce jeu de données produisent
pour la plupart des résumés de 3 phrases du document original.

C'est ce jeu de données qu'on utilisera pour tous les tests.

\subsection{Pré-calcul des scores ROUGE}

Les méthodes présentées dans les prochains chapitres utilisent toutes
un grand nombre de scores $R$ dans leur procédure d'entraînement.

Or, comme le calcul du ROUGE entre deux résumés est lent.
Aussi, il est possible de calculer, pour un document $d$ donné, tous les
résumés de 3 phrases possibles.

On a donc, dans un premier temps, précalculé tous les scores ROUGE
associés à tous les résumés pour chaque document du jeu de données.

\commentaire{Ce chapitre me semble un peu décousu, j'ai l'impression que
    plusieurs de mes sections ont des dépendances circulaires les unes envers les autres.}
\chapter{Estimation BAALALAL Monte-Carlo}
\label{chap:mcs}                   % étiquette pour renvois (à compléter!)

<Texte du chapitre ou de l'article.>

\section{Description}

\section{Apprentissage}

\subsection{Approche contextuelle}

Approches inspirées de UCT

\subsection{Approche linéaire}

Leaf-LinUCT and so on

\section{Résultats}

\chapter{Recherche arborescente}
\label{chap:mcts}                   % étiquette pour renvois (à compléter!)

<Texte du chapitre ou de l'article.>

\section{Description}

\section{Apprentissage}

\subsection{Approche contextuelle}

Approches inspirées de UCT

\subsection{Approche linéaire}

Leaf-LinUCT and so on

\section{Résultats}

\chapter{Recherche arborescente linéaire}
\label{chap:linmcts}                   % étiquette pour renvois (à compléter!)

\section{Estimation du score de tous les résumés}

On a des représentations vectorielles $\psi(s)$ des noeuds
de l'arbre de résumé.
Les représentations sont issues de l'encodeur.
On cherche à trouver le mapping linéaire $\theta$ minimisant

\begin{equation}
    \left( R(s, \hat(s)) - \langle \theta, \psi(s)\rangle \right)^2
\end{equation}


\section{Recherche d'une approximation du score de tous les résumés}

\commentaire{Juste un copier-coller d'un document que j'avais préparé à la fin de
    l'été.}

The experiments were done using 25 000 random training documents from the CNN/DailyMail dataset.
For the pre-training, a fully connected layer took as input 3 sentence embeddings and was tasked with predicting their relative ROUGE-\{1,2,L\} scores.
For each document in the pre-training phase, 250 summaries of 3 sentences were randomly sampled and their ROUGE scores were used as targets.
The pre-training batch size was of 64.

\tikzsetnextfilename{linMcts_pretraining}
\begin{center}
    \begin{tikzpicture}
        \begin{axis}[title={Importance of pre-training, $\alpha = 0.1$}, grid style={dashed,gray!50}, axis y line*=left, axis x line*=bottom, every axis plot/.append style={line width=1.5pt, mark size=0pt, font=\huge}, name=plot0, xshift=-.1\textwidth, y tick label style={/pgf/number format/fixed zerofill}, ylabel={Inferred ROUGE}, xlabel={n samples}, width=\textwidth, legend style={at={(1,0.1)},anchor=south east}, smooth, xmin=0, xmax=500, ymin=0.2, ymax=0.45]
            \addplot[blue, ylabel near ticks, line width=3pt] table[x=x0, y=y0, col sep=comma]{linmcts/linmctsExpResults/pre_training_experiment/alpha_0.1--1000--pretrained.csv};
            \addplot[red, ylabel near ticks, line width=3pt] table[x=x0, y=y0, col sep=comma]{linmcts/linmctsExpResults/pre_training_experiment/alpha_0.1--100--pretrained.csv};
            \addplot[green, ylabel near ticks, line width=3pt] table[x=x0, y=y0, col sep=comma]{linmcts/linmctsExpResults/pre_training_experiment/alpha_0.1--1000--raw.csv};
            \addplot[black, dotted, ylabel near ticks, line width=3pt] table[x=x0, y=y3, col sep=comma]{linmcts/linmctsExpResults/pre_training_experiment/alpha_0.1--1000--raw.csv};
            \legend{1000 warmup steps, 100 warmup steps, random, maximum}
        \end{axis}
    \end{tikzpicture}
\end{center}

\noindent\textbf{Analysis}
The pretraining works as intended.
After 100 warmup steps, the score prediction head has merely learned which range is adequate for ROUGE-\{1,2,L\}
After 1000 warmup steps, it has however begun to better correlate embeddings to their scores, enabling a much improved MCTS exploration.
Less massive batches could be interesting to explore (i.e. instead of sampling 250 summaries per document, move to a more reasonable 16 summaries in a more classical online learning fashion).


\tikzsetnextfilename{linMcts_exploration}
\begin{center}
    \begin{tikzpicture}
        \begin{axis}[title={Exploring the exploration, all 10 000 warmup steps}, grid style={dashed,gray!50}, axis y line*=left, axis x line*=bottom, every axis plot/.append style={line width=1.5pt, mark size=0pt, font=\huge}, name=plot0, xshift=-.1\textwidth, y tick label style={/pgf/number format/fixed zerofill}, ylabel={Inferred ROUGE}, xlabel={n samples}, width=\textwidth, legend style={at={(1,0.1)},anchor=south east}, smooth, xmin=0, xmax=500, ymin=0.2, ymax=0.45]
            \addplot[blue, ylabel near ticks, line width=3pt] table[x=x0, y=y0, col sep=comma]{linmcts/linmctsExpResults/alpha_experiment/alpha_0.1--1000--pretrained.csv};
            \addplot[red, ylabel near ticks, line width=3pt] table[x=x0, y=y0, col sep=comma]{linmcts/linmctsExpResults/alpha_experiment/alpha_1.0--1000--pretrained.csv};
            \addplot[green, ylabel near ticks, line width=3pt] table[x=x0, y=y0, col sep=comma]{linmcts/linmctsExpResults/alpha_experiment/alpha_0.01--1000--pretrained.csv};
            \addplot[cyan, ylabel near ticks, line width=3pt] table[x=x0, y=y0, col sep=comma]{linmcts/linmctsExpResults/alpha_experiment/alpha_10.0--1000--pretrained.csv};
            \addplot[black, dotted, ylabel near ticks, line width=3pt] table[x=x0, y=y3, col sep=comma]{linmcts/linmctsExpResults/pre_training_experiment/alpha_0.1--1000--raw.csv};
            \legend{$\alpha = 0.1$, $\alpha = 1.0$, $\alpha = 0.01$, $\alpha = 10.0$, maximum}
        \end{axis}
    \end{tikzpicture}
\end{center}

\noindent\textbf{Analysis}
There seems to be a sole winner of $\alpha=0.1$ as the best exploration parameter for our experiments.
It seems this parameter is absolutely crucial as it really influences the performance in both the earlier and later stages of the MCTS.


\tikzsetnextfilename{linMcts_less20}
\begin{tikzpicture}[baseline]
    \begin{axis}[grid style={dashed,gray!50}, axis y line*=left, axis x line*=bottom, every axis plot/.append style={line width=1.5pt, mark size=0pt, font=\Large}, width=0.45\textwidth,
            height=0.6\textwidth, name=plot0, xshift=-.1\textwidth, y tick label style={/pgf/number format/fixed zerofill},xmin=0.0, xmax=500.0, ymin=0.20, ymax=0.45, ylabel={Inferred ROUGE}, xlabel={n samples}, legend style={at={(1,0.05), nodes={scale=0.5, transform shape}},anchor=south east}, smooth, title={$|d| \leq 20$}]
        \addplot[yellow, ylabel near ticks, line width=3pt] table[x=x0, y=y0, col sep=comma]{linmcts/linmctsExpResults/document_length_experiment/alpha_0.01--1000--pretrained--c1.csv};
        \addplot[blue, ylabel near ticks, line width=3pt] table[x=x0, y=y0, col sep=comma]{linmcts/linmctsExpResults/document_length_experiment/alpha_0.1--1000--pretrained--c1.csv};
        \addplot[red, ylabel near ticks, line width=3pt] table[x=x0, y=y0, col sep=comma]{linmcts/linmctsExpResults/document_length_experiment/alpha_1.0--1000--pretrained--c1.csv};
        \addplot[black, dotted, ylabel near ticks, line width=3pt] table[x=x0, y=y3, col sep=comma]{linmcts/linmctsExpResults/document_length_experiment/alpha_0.1--1000--pretrained--c1.csv};
        \legend{$\alpha=0.01$, $\alpha=0.1$, $\alpha=1.0$, maximum}

    \end{axis}
\end{tikzpicture}
\tikzsetnextfilename{linMcts_20to35}
\begin{tikzpicture}[baseline]
    \begin{axis}[grid style={dashed,gray!50}, axis y line*=left, axis x line*=bottom, every axis plot/.append style={line width=1.5pt, mark size=0pt, font=\huge}, width=.45\textwidth,
            height=0.6\textwidth, name=plot0, xshift=-.1\textwidth, y tick label style={/pgf/number format/fixed zerofill},xmin=0.0, xmax=500.0, ymin=0.20, ymax=0.45, xlabel={n samples}, legend style={at={(0.9,0.1)},anchor=south east}, smooth, title={$20 < |d| < 35$}, ymajorticks=false]
        \addplot[yellow, ylabel near ticks, line width=3pt] table[x=x0, y=y0, col sep=comma]{linmcts/linmctsExpResults/document_length_experiment/alpha_0.01--1000--pretrained--c2.csv};
        \addplot[blue, ylabel near ticks, line width=3pt] table[x=x0, y=y0, col sep=comma]{linmcts/linmctsExpResults/document_length_experiment/alpha_0.1--1000--pretrained--c2.csv};
        \addplot[red, ylabel near ticks, line width=3pt] table[x=x0, y=y0, col sep=comma]{linmcts/linmctsExpResults/document_length_experiment/alpha_1.0--1000--pretrained--c2.csv};
        \addplot[black, dotted, ylabel near ticks, line width=3pt] table[x=x0, y=y3, col sep=comma]{linmcts/linmctsExpResults/document_length_experiment/alpha_0.1--1000--pretrained--c2.csv};
    \end{axis}
\end{tikzpicture}
\tikzsetnextfilename{linMcts_more35}
\begin{tikzpicture}[baseline]
    \begin{axis}[grid style={dashed,gray!50}, axis y line*=right, axis x line*=bottom, every axis plot/.append style={line width=1.5pt, mark size=0pt, font=\huge}, width=.4\textwidth,
            height=0.6\textwidth, name=plot0, xshift=-.1\textwidth, y tick label style={/pgf/number format/fixed zerofill},xmin=0.0, xmax=500.0, ymin=0.20, ymax=0.45, xlabel={n samples}, legend style={at={(1,0.1)},anchor=south east}, smooth, title={$35 \leq |d|$}]
        \addplot[yellow, ylabel near ticks, line width=3pt] table[x=x0, y=y0, col sep=comma]{linmcts/linmctsExpResults/document_length_experiment/alpha_0.01--1000--pretrained--c3.csv};
        \addplot[blue, ylabel near ticks, line width=3pt] table[x=x0, y=y0, col sep=comma]{linmcts/linmctsExpResults/document_length_experiment/alpha_0.1--1000--pretrained--c3.csv};
        \addplot[red, ylabel near ticks, line width=3pt] table[x=x0, y=y0, col sep=comma]{linmcts/linmctsExpResults/document_length_experiment/alpha_1.0--1000--pretrained--c3.csv};
        \addplot[black, dotted, ylabel near ticks, line width=3pt] table[x=x0, y=y3, col sep=comma]{linmcts/linmctsExpResults/document_length_experiment/alpha_0.1--1000--pretrained--c3.csv};
    \end{axis}
\end{tikzpicture}
\vspace{10pt}

\noindent\textbf{Analysis}
Let's first note that there are 8674, 10490 and 5796 documents for each of the respective graphs.
The longer the document, the highest is its maximum performing summary.
It seems like a single value of $\alpha$ is still the way to go, no matter the document length.
Smaller documents are easier to solve for the MCTS, much likely due to the exponentially smaller tree to explore.
The convergence is not only better but also faster for smaller documents.
This implies that one should attempt to scale the number of samples made by the MCTS with the number of sentences in a document.
A linear scaling seems legitimate, starting at 200 samples for documents of 10 sentences and ending at 500 for documents of 50 sentences i.e. $n = \ceil{7.5|d|} + 125$.

\section*{Conclusion}

\begin{itemize}
    \item \textbf{Pre-training formulation seems to be adequate.} The number of 1000 warmup steps is to be fixed and the number of sampled summaries per document should be tried at the lower value of 16.
    \item \textbf{Should explore values of $\alpha \in [0.05, 0.5]$}.
    \item Should try scaling number of MCTS samples per document with document length via $n = \ceil{7.5|d|} + 125$.
\end{itemize}

\section{Intégration}

Le linUCT génère des targets $\theta$, qu'on apprend par perte cosine.
À l'inférence, on utilise la linéarité du produit vectoriel
et le fait que nos représentations de résumés sont la somme
des représentations des phrases pour prédire les 3 phrases
dont la représentation a le produit vectoriel maximal avec la mapping
prédit.
\include{conclusion}

\appendix                       % annexes le cas échéant

% \include{annexe/a}                % annexe A

\bibliographystyle{abbrvnat}         % production de la bibliographie
\bibliography{biblio}

\end{document}
