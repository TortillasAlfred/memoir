%% GABARIT POUR MÉMOIRE STANDARD
%%
%% Consulter la documentation de la classe ulthese pour une
%% description détaillée de la classe, de ce gabarit et des options
%% disponibles.
%%
%% [Ne pas hésiter à supprimer les commentaires après les avoir lus.]
%%
%% Déclaration de la classe avec le type de grade
%%   [l'un de MATDR, MArch, MA, LLM, MErg, MMus, MPht, MSc, MScGeogr,
%%    MServSoc, MPsEd]
%% et les langues les plus courantes. Le français sera la langue par
%% défaut du document.
\documentclass[MSc,english,french]{ulthese}
  %% Encodage utilisé pour les caractères accentués dans les fichiers
  %% source du document. Les gabarits sont encodés en UTF-8. Inutile
  %% avec XeLaTeX, qui gère Unicode nativement.
  \ifxetex\else \usepackage[utf8]{inputenc} \fi

  %% Charger ici les autres paquetages nécessaires pour le document.
  %% Quelques exemples; décommenter au besoin.
  %\usepackage{amsmath}       % recommandé pour les mathématiques
  %\usepackage{icomma}        % gestion de la virgule dans les nombres

  %% Utilisation d'une autre police de caractères pour le document.
  %% - Sous LaTeX
  \usepackage{mathpazo}      % texte et mathématiques en Palatino
  \usepackage{mathptmx}      % texte et mathématiques en Times
  %% - Sous XeLaTeX
  %\setmainfont{TeX Gyre Pagella}      % texte en Pagella (Palatino)
  %\setmathfont{TeX Gyre Pagella Math} % mathématiques en Pagella (Palatino)
  %\setmainfont{TeX Gyre Termes}       % texte en Termes (Times)
  %\setmathfont{TeX Gyre Termes Math}  % mathématiques en Termes (Times)

  %% Options de mise en forme du mode français de babel. Consulter la
  %% documentation du paquetage babel pour les options disponibles.
  %% Désactiver (effacer ou mettre en commentaire) si l'option
  %% 'nobabel' est spécifiée au chargement de la classe.
  \frenchbsetup{%
    StandardItemizeEnv=true,       % format standard des listes
    ThinSpaceInFrenchNumbers=true, % espace fine dans les nombres
    og=«, fg=»                     % caractères « et » sont les guillemets
  }

  %% Style de la bibliographie.
 % \bibliographystyle{}

  %% Composition de la page frontispice. Remplacer les éléments entre < >.
  %% Supprimer les caractères < >. Couper un long titre ou un long
  %% sous-titre manuellement avec \\.
  \titre{Méthodes neuronales de Monte-Carlo pour la génération automatique de résumés de textes}
  % \titre{Ceci est un exemple de long titre \\
  %   avec saut de ligne manuel}
  % \soustitre{Sous-titre le cas échéant}
  % \soustitre{Ceci est un exemple de long sous-titre \\
  %   avec saut de ligne manuel}
  \auteur{Mathieu Godbout}
  \programme{Maîtrise en informatique}
  % \direction{<Prénom Nom>, <directeur ou directrice> de recherche}
  % \codirection{<Prénom Nom>, <codirecteur ou codirectrice> de recherche}
  \codirection{Luc Lamontagne, codirecteur de recherche \\
              Audrey Durand, codirectrice de recherche}

  %% Les commandes ci-dessous servent uniquement pour la création
  %% d'une page de titre (interdite lors du dépôt à la FESP).
  % \annee{<20xx>}

\begin{document}

\frontmatter                    % pages liminaires

\frontispice                    % production de la page frontispice

\chapter*{Résumé}               % ne pas numéroter
\label{chap:resume}             % étiquette pour renvois
\phantomsection\addcontentsline{toc}{chapter}{\nameref{chap:resume}} % inclure dans TdM

\begin{otherlanguage*}{french}
  <Texte du résumé en français. Obligatoire.>
\end{otherlanguage*}
                % résumé français
\chapter*{Abstract}             % ne pas numéroter
\label{chap:abstract}           % étiquette pour renvois
\phantomsection\addcontentsline{toc}{chapter}{\nameref{chap:abstract}} % inclure dans TdM

\begin{otherlanguage*}{english}
This thesis discusses the use of bandit methods to solve 
the problem of training extractive abstract generation models.
The extractive models, which build summaries by selecting sentences from an 
original document, are difficult to train because the target summary of a document is usually not built in an extractive way.
It is for this purpose that we propose to see the production of extractive summaries
as different bandit problems, for which there exist algorithms that can be leveraged for training summarization models.

In this paper, BanditSum is first presented, an approach drawn from 
the literature that sees the generation of the summaries 
of a set of documents as a contextual bandit problem.
Next, we introduce CombiSum, a new algorithm 
which formulates the generation of the summary of a 
single document as a combinatorial bandit.
By exploiting the combinatorial formulation,
CombiSum manages to incorporate the notion of the extractive potential 
of each sentence of a document in its training.
Finally, we propose LinCombiSum, the linear variant of CombiSum 
which exploits the similarities between sentences in a document 
and uses the linear combinatorial bandit formulation instead.
 
\end{otherlanguage*}
              % résumé anglais
\cleardoublepage

\maxtocdepth{subsection}
\tableofcontents                % production de la TdM
\cleardoublepage

\listoftables                   % production de la liste des tableaux
\cleardoublepage

\listoffigures                  % production de la liste des figures
\cleardoublepage

\dedicace{<Dédicace si désiré>}
\cleardoublepage

\epigraphe{<Texte de l'épigraphe>}{<Source ou auteur>}
\cleardoublepage

\chapter*{Remerciements}        % ne pas numéroter
\label{chap:remerciements}      % étiquette pour renvois
\phantomsection\addcontentsline{toc}{chapter}{\nameref{chap:remerciements}} % inclure dans TdM

Je tiens à remercier mon directeur de recherche, Luc Lamontagne,
pour ses précieux conseils tout au long de mon parcours à la maîtrise.
Sa bienveillance et sa franchise m'auront permis de progresser non 
seulement en tant que chercheur, mais aussi en tant que personne.
Un merci tout particulier à Audrey Durand, ma co-directrice, pour m'avoir initié 
au passionnant domaine des bandits et de l'apprentissage par renforcement.
À travers nos interactions, j'ai découvert tout le plaisir 
qui se cache derrière le processus scientifique bien fait et je suis
impatient de continuer notre collaboration.
Par leurs efforts conjoints, Luc et Audrey m'ont permis de me 
dépasser et de bâtir mon propre bagage scientifique.

Je ne peux passer sous le silence l'incroyable communauté de recherche 
du Groupe de Recherche en Apprentissage Automatique 
de l'université Laval (GRAAL) dans laquelle j'ai eu la chance de compléter 
ma maîtrise.
Tantôt collègues de travail, tantôt partenaires de célébrations, les 
personnes que j'ai rencontrées au GRAAL ont grandement contribué 
à faire de ma maîtrise une période de ma vie que je chérirai toujours.
Merci à David, Jean-Thomas, Nicolas, Jean-Samuel, Frédérik, Gaël
et tous les autres.

Un merci spécial va aussi à mes parents, Martine et Alain, qui, avec leur éternelle confiance en 
mes capacités et leur dévouement à mon épanouissement ont rendu possible
mon cheminement scolaire et personnel.
Je conserve le dernier de mes remerciements pour Samantha, sans qui je ne serais pas 
l'ombre de la personne que je suis aujourd'hui. 
Partenaire de mes joies comme de mes peines, elle a toujours su faire ressortir 
la meilleure version de moi-même.
Par son écoute, ses questions intéressées et son soutien incomparable,
elle a certainement participé à l'écriture de ce mémoire dans une bien plus grande
proportion qu'elle ne peut l'imaginer.

Je remercie aussi le Conseil de Recherches en Sciences Naturelles et en Génie du Canada
(CRSNG), le Fonds de recherche du Québec — Nature et Technologies (FRQ-NT)
et Intact~Corporation~financière pour leur soutien financier ayant permis l'élaboration de ce document.         % remerciements
\chapter*{Avant-propos}         % ne pas numéroter
\label{chap:avantpropos}        % étiquette pour renvois
\phantomsection\addcontentsline{toc}{chapter}{\nameref{chap:avantpropos}} % inclure dans TdM

<Texte de l'avant-propos. Obligatoire dans une thèse ou un mémoire par
articles.>
           % avant-propos

\mainmatter                     % corps du document

% !TeX root = ./main.tex
\chapter{Introduction}
\label{chap:introduction}       % étiquette pour renvois

<Texte de l'introduction. La thèse ou le mémoire devrait normalement
débuter par une introduction. Celle-ci est traitée comme un chapitre
normal, sauf qu'elle n'est pas numérotée.>

\section{Objectifs}

\section{Structure du mémoire}
\chapter{Apprentissage profond}     % numéroté
\label{chap:reseaux_neurones}                   % étiquette pour renvois (à compléter!)

<Texte du chapitre ou de l'article.>

\section{Perceptron multicouche}

\section{Réseaux à convolution}

\section{Réseaux récurrents}
% !TeX root = ./main.tex
\chapter{Méthodes de Monte-Carlo}     % numéroté
\label{chap:montecarlo}                   % étiquette pour renvois (à compléter!)

<Texte du chapitre ou de l'article.>

\section{Processus décisionnel de Markov}

\section{Estimation de Monte-Carlo}

Approximation statistique de procédés déterministes

\section{Recherche arborescente de Monte-Carlo}
\chapter{Génération automatique de résumés de textes}     % numéroté
\label{chap:generation_resumes}                           % étiquette pour renvois (à compléter!)

<Texte du chapitre ou de l'article.>

\section{Traitement automatique de la langue naturelle}

\section{Formulation extractive}

\section{Formulation abstractive}

\section{Évaluation de la performance}

\section{Jeux de données}
% !TeX root = ./main.tex
\chapter{Formulation en bandit contextuel}
\label{chap:bandits}                   % étiquette pour renvois (à compléter!)

<Texte du chapitre ou de l'article.>

\section{Description}

\section{Apprentissage}

\section{Résultats}

\chapter{Formulation séquentielle}
\label{chap:mcts}                   % étiquette pour renvois (à compléter!)

<Texte du chapitre ou de l'article.>

\section{Description}

\section{Apprentissage}

\subsection{Approche contextuelle}

Approches inspirées de UCT

\subsection{Approche linéaire}

Leaf-LinUCT and so on

\section{Résultats}

\chapter*{Conclusion}           % ne pas numéroter
\label{chap:conclusion}         % étiquette pour renvois
\phantomsection\addcontentsline{toc}{chapter}{\nameref{chap:conclusion}} % inclure dans TdM

<Texte de la conclusion. Une thèse ou un mémoire devrait normalement
se terminer par une conclusion placée avant les annexes, le cas
échéant. La conclusion est traitée comme un chapitre normal, sauf
qu'elle n'est pas numérotée.>


\appendix                       % annexes le cas échéant

% \chapter{<Titre de l'annexe>}     % numérotée
\label{chap:}                   % étiquette pour renvois (à compléter!)

<Texte de l'annexe.>
                % annexe A

% \bibliography{}                 % production de la bibliographie

\end{document}
