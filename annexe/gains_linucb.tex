\chapter{Gains computationnels LinUCB}     % numérotée
\label{chap:lincub_gains}                   % étiquette pour renvois (à compléter!)

Cet annexe vise à décrire comment il est possible d'alléger le coût computationnel
lié à LinUCB en exploitant la nature de la matrice $\mathbf{V}_t$ construite
selon \eqref{eq:linucb_iteratif}.
Commençons d'abord par énoncer la formule de Sherman-Morrison \citep{sherman1950adjustment}:

\textbf{Formule de Sherman-Morrison}
Soit $\mathbf{M} \in \mathbb{R}^{n \times n}$ une matrice carrée inversible et $u, v \in \mathbb{R}^n$
deux vecteurs.
Dans ce cas, $\mathbf{M} + uv^\top$ est inversible si et seulement si $1 + v^\top \mathbf{M}u \neq 0$.
On a alors
\begin{equation*}
    \left(\mathbf{M} + uv^\top \right)^{-1} = \mathbf{M}^{-1} - \frac{\mathbf{M}^{-1}uv^\top\mathbf{M}^{-1}}{1 + v^\top \mathbf{M}u}.
    \label{eq:sherman_morrison}
\end{equation*}

Pour le calcul de $\mathbf{V}_t$ dans LinUCB, la formule s'applique directement et
donne 
\begin{equation*}
\mathbf{V}_t^{-1} =
\left(\mathbf{V}_{t-1} + \tilde{a}_{t}^\top \tilde{a}_{t} \right)^{-1} =
\mathbf{V}_{t-1}^{-1} - \frac{\mathbf{V}_{t-1}^{-1}\tilde{a}_{t}\tilde{a}_{t}^\top\mathbf{V}_{t-1}^{-1}}{1 + \tilde{a}_{t}^\top \mathbf{V}_{t-1}\tilde{a}_{t}},
\end{equation*}
un calcul bien plus efficace que d'inverser la matrice $\mathbf{V}_t$ car il ne requiert 
que des produits matriciels et vectoriels.

On note aussi qu'une légère optimisation peut aussi être faite en constatant que 
le produit $\mathbf{V}^{-1}_{t-1}\tilde{a}_{t}$ est utilisé à plusieurs reprises.
On peut donc entreposer le produit en posant 
\begin{equation*}
W = \mathbf{V}^{-1}_{t-1}\tilde{a}_{t}.
\end{equation*}

Enfin, comme la matrice $\mathbf{V}$ sera toujours symétrique car elle est la 
somme de matrices symétriques, on aura aussi $\mathbf{V}^{-1}$ symétrique.
Cette symétrie peut aussi être exploitée pour donner 
\begin{equation*}
    \mathbf{V}_t^{-1} =
    \mathbf{V}_{t-1}^{-1} - \frac{WW^\top}{1 + \tilde{a}_{t}^\top W},
\end{equation*}
qui est la version utilisée dans notre implémentation de LinUCB.