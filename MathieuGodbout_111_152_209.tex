%% GABARIT POUR MÉMOIRE STANDARD
%%
%% Consulter la documentation de la classe ulthese pour une
%% description détaillée de la classe, de ce gabarit et des options
%% disponibles.
%%
%% [Ne pas hésiter à supprimer les commentaires après les avoir lus.]
%%
%% Déclaration de la classe avec le type de grade
%%   [l'un de MATDR, MArch, MA, LLM, MErg, MMus, MPht, MSc, MScGeogr,
%%    MServSoc, MPsEd]
%% et les langues les plus courantes. Le français sera la langue par
%% défaut du document.
\documentclass[MSc,english,french]{ulthese}
  %% Encodage utilisé pour les caractères accentués dans les fichiers
  %% source du document. Les gabarits sont encodés en UTF-8. Inutile
  %% avec XeLaTeX, qui gère Unicode nativement.
  \ifxetex\else \usepackage[utf8]{inputenc} \fi

  %% Charger ici les autres paquetages nécessaires pour le document.
  %% Quelques exemples; décommenter au besoin.
  \usepackage{amsmath}       % recommandé pour les mathématiques
  \usepackage{icomma}        % gestion de la virgule dans les nombres
  \usepackage{amssymb}
  %% Utilisation d'une autre police de caractères pour le document.
  %% - Sous LaTeX
  \usepackage{tgpagella}      % texte et mathématiques en Palatino
  % \usepackage{mathptmx}      % texte et mathématiques en Times
  %% - Sous XeLaTeX
  %\setmainfont{TeX Gyre Pagella}      % texte en Pagella (Palatino)
  %\setmathfont{TeX Gyre Pagella Math} % mathématiques en Pagella (Palatino)
  %\setmainfont{TeX Gyre Termes}       % texte en Termes (Times)
  %\setmathfont{TeX Gyre Termes Math}  % mathématiques en Termes (Times)

  %% Options de mise en forme du mode français de babel. Consulter la
  %% documentation du paquetage babel pour les options disponibles.
  %% Désactiver (effacer ou mettre en commentaire) si l'option
  %% 'nobabel' est spécifiée au chargement de la classe.
  \frenchbsetup{%
    StandardItemizeEnv=true,       % format standard des listes
    ThinSpaceInFrenchNumbers=true, % espace fine dans les nombres
    og=«, fg=»                     % caractères « et » sont les guillemets
  }

  %% Style de la bibliographie.  

  \titre{Approches par bandit pour la\\ génération automatique de résumés de textes}
  \auteur{Mathieu Godbout}
  \programme{Maîtrise en informatique}
  \codirection{Luc Lamontagne, directeur de recherche \\
              Audrey Durand, codirectrice de recherche}


\usepackage{tikz}
% \usepackage{ctable}
\usepackage{icomma}
\usepackage{graphicx}
\usetikzlibrary{intersections}
\usetikzlibrary{external}
\usetikzlibrary{shapes.misc, positioning, fit, arrows.meta, arrows, shapes, backgrounds, calc}
%\tikzexternalize[prefix=figures/]
%\tikzset{external/export=true}

\usepackage{pgfplots}
\usepackage{pgfplotstable}
\usepgfplotslibrary{colorbrewer}
\usepgfplotslibrary{fillbetween}
\pgfplotsset{compat=1.16, colormap/Blues, every axis/.append style={label style={font=\large}, tick label style={font=\large}}}

\usepackage{mathtools}
\DeclarePairedDelimiter{\ceil}{\lceil}{\rceil}
\DeclareMathOperator*{\argmax}{arg\,max}
\DeclareMathOperator*{\argmin}{arg\,min}


\usepackage[mathscr]{euscript}
\let\euscr\mathscr \let\mathscr\relax
\usepackage[scr]{rsfso}

\usepackage{xcolor}
\newcommand\todo[1]{\textcolor{red}{TODO: #1}}
\newcommand\question[1]{\textcolor{blue}{Question: #1}}
\newcommand\commentaire[1]{\textcolor{violet}{Commentaire: #1}}
\newcommand\ngrams{\textit{n}-grammes }

\usepackage{mathtools}

\DisemulatePackage{setspace}
\usepackage{setspace}

\addto\captionsfrench{\def\tablename{Tableau}}

  % French version of algorithmic

  \usepackage[noend]{algpseudocode}
  \usepackage{algorithm}
  \algnewcommand\AND{\textbf{et} }
	\renewcommand{\listalgorithmname}{Liste des algorithmes}
	\floatname{algorithm}{Algorithme}
	\renewcommand{\algorithmicreturn}{\textbf{retourner}}
	\renewcommand{\algorithmicprocedure}{\textbf{procédure}}
	\renewcommand{\algorithmicrequire}{\textbf{Entrée:}}
	\renewcommand{\algorithmicensure}{\textbf{Sortie:}}
	\renewcommand{\algorithmicend}{\textbf{fin}}
	\renewcommand{\algorithmicif}{\textbf{si}}
	\renewcommand{\algorithmicthen}{\textbf{alors}}
	\renewcommand{\algorithmicelse}{\textbf{sinon}}
	\renewcommand{\algorithmicfor}{\textbf{pour}}
	\renewcommand{\algorithmicforall}{\textbf{pour tout}}
	\renewcommand{\algorithmicdo}{\textbf{faire}}
	\renewcommand{\algorithmicwhile}{\textbf{tant que}}
	\newcommand{\algorithmicelsif}{\algorithmicelse\ \algorithmicif}
	\newcommand{\algorithmicendif}{\algorithmicend\ \algorithmicif}
	\newcommand{\algorithmicendfor}{\algorithmicend\ \algorithmicfor}

  \makeatletter
% start with some helper code
% This is the vertical rule that is inserted
\newcommand*{\algrule}[1][\algorithmicindent]{%
  \makebox[#1][l]{%
    \hspace*{.2em}% <------------- This is where the rule starts from
    \vrule height .4\baselineskip depth .2\baselineskip
  }
}

\newcount\ALG@printindent@tempcnta
\def\ALG@printindent{%
    \ifnum \theALG@nested>0% is there anything to print
    \ifx\ALG@text\ALG@x@notext% is this an end group without any text?
    % do nothing
    \else
    \unskip
    % draw a rule for each indent level
    \ALG@printindent@tempcnta=1
    \loop
    \algrule[\csname ALG@ind@\the\ALG@printindent@tempcnta\endcsname]%
    \advance \ALG@printindent@tempcnta 1
    \ifnum \ALG@printindent@tempcnta<\numexpr\theALG@nested+1\relax
    \repeat
    \fi
    \fi
}
% the following line injects our new indent handling code in place of the default spacing
\patchcmd{\ALG@doentity}{\noindent\hskip\ALG@tlm}{\ALG@printindent}{}{\errmessage{failed to patch}}
\patchcmd{\ALG@doentity}{\item[]\nointerlineskip}{}{}{} % no spurious vertical space
% end vertical rule patch for algorithmicx
\makeatother

\begin{document}

\frontmatter                    % pages liminaires

\frontispice                    % production de la page frontispice

\chapter*{Résumé}               % ne pas numéroter
\label{chap:resume}             % étiquette pour renvois
\phantomsection\addcontentsline{toc}{chapter}{\nameref{chap:resume}} % inclure dans TdM

\begin{otherlanguage*}{french}

Ce mémoire aborde l'utilisation des méthodes par bandit pour résoudre 
la problématique de l'entraînement de modèles de générations de résumés extractifs.
Les modèles extractifs, qui bâtissent des résumés en sélectionnant des phrases d'un 
document original, sont difficiles à entraîner car le résumé cible 
correspondant à un document n'est habituellement pas constitué de 
manière extractive.
C'est à cet effet que l'on propose de voir la production de résumés
extractifs comme différents problèmes de bandit, lesquels sont 
accompagnés d'algorithmes pouvant être utilisés pour l'entraînement.

On commence ce document en présentant BanditSum, une approche tirée 
de la litérature et qui voit la génération des résumés 
d'un ensemble de documents comme un problème de bandit 
contextuel.
Ensuite, on introduit CombiSum, un nouvel algorithme 
qui formule la génération du résumé d'un 
seul document comme un bandit combinatoire.
En exploitant la formule combinatoire,
CombiSum réussit à incorporer la notion du potentiel 
extractif de chaque phrase à son entraînement.
Enfin, on propose LinCombiSum, la variante linéaire de CombiSum 
qui exploite les similarités entre les phrases d'un document 
et emploie plutôt la formulation en bandit linéaire combinatoire.

\end{otherlanguage*}
                % résumé français
\chapter*{Abstract}             % ne pas numéroter
\label{chap:abstract}           % étiquette pour renvois
\phantomsection\addcontentsline{toc}{chapter}{\nameref{chap:abstract}} % inclure dans TdM

\begin{otherlanguage*}{english}
  This thesis discusses the use of bandit methods to solve 
the problem of training extractive abstract generation models.
The extractive models, which build summaries by selecting sentences from an 
original document, are difficult to train because the target summary of a document is usually not built in an extractive way.
It is for this purpose that we propose to see the production of an extractive summary
as different bandit problems, for which there exist algorithms that can be used for training.

In this document, we first see how the generation of the summary 
of any document can be formulated as a contextual bandit problem
to achieve excellent computational efficiency.
Next, we propose to see the generation of the abstract 
of a single document like a combinatorial bandit.
We present a method for obtaining targets 
for training a model from the resolution 
of the combinatorial bandit.
We present two algorithms, UCB and LinUCB, which can 
be used for this purpose.
UCB, the first algorithm presented, is characterized by 
its universality while LinUCB is computationally
more efficient, at the added cost of an assumption of the existence 
a linear relationship between a sentence and the quality of abstracts 
it produces. 
\end{otherlanguage*}
              % résumé anglais
\cleardoublepage

\maxtocdepth{subsection}
\tableofcontents                % production de la TdM
\cleardoublepage

\listoftables                   % production de la liste des tableaux
\cleardoublepage

\listoffigures                  % production de la liste des figures
\cleardoublepage

\listofalgorithms
\cleardoublepage

\dedicace{À Samantha.}
\cleardoublepage

% \epigraphe{"Éviter les situations dans lesquelles on peut faire des erreurs est peut-être la pire des erreurs."}{Peter McWilliams}
\epigraphe{To avoid situations in which you might make mistakes may be the biggest mistake of all.}{Peter McWilliams}
\cleardoublepage

\chapter*{Remerciements}        % ne pas numéroter
\label{chap:remerciements}      % étiquette pour renvois
\phantomsection\addcontentsline{toc}{chapter}{\nameref{chap:remerciements}} % inclure dans TdM

Je tiens à remercier mon directeur de recherche, Luc Lamontagne,
pour ses précieux conseils tout au long de mon parcours à la maîtrise.
Sa bienveillance et sa franchise m'auront permis de progresser non 
seulement en tant que chercheur, mais aussi en tant que personne.
Un merci tout particulier à Audrey Durand, ma co-directrice, pour m'avoir initié 
au passionnant domaine des bandits et de l'apprentissage par renforcement.
À travers nos interactions, j'ai découvert tout le plaisir 
qui se cache derrière le processus scientifique bien fait et je suis
impatient de continuer notre collaboration.
Par leurs efforts conjoints, Luc et Audrey m'ont permis de me 
dépasser et de bâtir mon propre bagage scientifique.

Je ne peux passer sous le silence l'incroyable communauté de recherche 
du Groupe de Recherche en Apprentissage Automatique 
de l'université Laval (GRAAL) dans laquelle j'ai eu la chance de compléter 
ma maîtrise.
Tantôt collègues de travail, tantôt partenaires de célébrations, les 
personnes que j'ai rencontrées au GRAAL ont grandement contribué 
à faire de ma maîtrise une période de ma vie que je chérirai toujours.
Merci à David, Jean-Thomas, Nicolas, Jean-Samuel, Frédérik, Gaël
et tous les autres.

Un merci spécial va aussi à mes parents, Martine et Alain, qui, avec leur éternelle confiance en 
mes capacités et leur dévouement à mon épanouissement ont rendu possible
mon cheminement scolaire et personnel.
Je conserve le dernier de mes remerciements pour Samantha, sans qui je ne serais pas 
l'ombre de la personne que je suis aujourd'hui. 
Partenaire de mes joies comme de mes peines, elle a toujours su faire ressortir 
la meilleure version de moi-même.
Par son écoute, ses questions intéressées et son soutien incomparable,
elle a certainement participé à l'écriture de ce mémoire dans une bien plus grande
proportion qu'elle ne peut l'imaginer.

Je remercie aussi le Conseil de Recherches en Sciences Naturelles et en Génie du Canada
(CRSNG), le Fonds de recherche du Québec — Nature et Technologies (FRQ-NT)
et Intact~Corporation~financière pour leur soutien financier ayant permis l'élaboration de ce document.         % remerciements

\mainmatter                     % corps du document

% !TeX root = ./main.tex
\chapter*{Introduction}
\label{chap:introduction}       % étiquette pour renvois

\question{
    À quel point est-ce que l'introduction doit déboucher exactement sur mon mémoire ?
    J'ai l'impression qu'à la fin de mon intro, le lecteur devrait être en contexte
    par rapport au titre, i.e. il devrait savoir qu'on va parler de (1) résumés, (2)
    méthodes de MC et (3) que tout ça va être fait avec des NN.
}

Ère du big data: on a une quantité immense de données, notamment textuelles.

Succès des dernières décennies nous montrent le potentiel qui peut se cacher
derrière une utilisation judicieuse de ces données.

Or, il n'est pas toujours possible d'incorporer toutes les données dans 
un processus de traitement.

Notamment, dans certains domaines, les humains sont encore requis dans la loop 
dans des applications de traitement de données textuelles en raison de 
législations ou d'assurance qualité (i.e. processus de réclamations en assurance).

Comment faire profiter du grand afflux de données dans des systèmes où l'humain (et 
sa capacité de traitement limitée) sont essentiels ?

Une option qui peut s'avérer facilement attrayante est l'idée de condenser un ou plusieurs
documents en un texte concis contenant seulement l'information requise pour procéder à la 
tâche à exécuter.

On dit alors que l'on fait appel à un système de génération automatique de résumés de textes.

Comment fonctionnent ces systèmes de génération de résumés ?

L'espace des résumés possibles est très vaste; comment faire pour en trouver un qui est
suffisamment bon dans un délai de temps assez raisonnable pour l'utilisation dans un contexte
du monde réel ?

Des outils probabilistes, connus sous le nom de méthodes de Monte-Carlo, ont été
spécifiquement conçus pour de tels contextes.

Ils se basent sur l'aléatoire pour parcourir efficacement des espaces potentiellement
très grands.

En les jumelant à des approches neuronales ayant démontré largement leur efficacité sur des données
textuelles, il est naturel de s'attendre à ce que ces outils permettent d'obtenir des 
systèmes de génération automatique qui produisent non seulement d'excellents résumés, mais qui
ne le font pas au détriment d'un temps d'entraînement déraisonnable.

\section*{Objectifs}

Dans ce mémoire, on explorera d'abord comment différentes méthodes de Monte-Carlo
peuvent être utilisées pour estimer des mesures liées à la qualité de résumés de textes
dont le calcul est difficile à première vue.

On verra ensuite comment ces approximations obtenues par des approches Monte-Carlo
peuvent être utilisées dans un système complet de génération de résumés.

Les différents systèmes explorés seront enfin comparés aux approches représentant
l'état de l'art en génération de résumés en fonction de leur efficacité d'entraînement
et de prédiction ainsi que selon la qualité des résumés produits.

\section*{Structure du mémoire}

Ce mémoire sera divisé en trois chapitres pour les trois méthodes de Monte-Carlo explorées:
échantillonnage, recherche arborescente et recherche arborescente linéaire.

Pour chacun des chapitres, on posera d'abord une fonction en lien avec la génération de résumé
qui est difficile à évaluer.

On montera ensuite comment une méthode Monte-Carlo
peut être utilisée pour l'approximer.

On évaluera empiriquement la rapidité
de la convergence de la méthode proposée.

On proposera ensuite un modèle neuronal
de génération automatique de textes utilisant la méthode MC à l'essai.

Des expériences seront ensuite effectuées pour évaluer la performance du modèle.

Les hyperparamètres de la portion MC seront choisis en fonction des tests
empiriques préliminaires.
% !TeX root = ./main.tex
\chapter{Prérequis}     % numéroté
\label{chap:montecarlo}                   % étiquette pour renvois (à compléter!)

Tout ceci est seulement en place à titre indicatif.
Toute information pertinente sera insérée ou retirée en fonction
du développement des chapitres de corps du mémoire.

\section{Méthodes de Monte-Carlo}

Approximation statistique de procédés déterministes

\subsection{Estimation de Monte-Carlo}

\subsection{Recherche arborescente de Monte-Carlo}

\section{Réseaux de neurones}

\subsection{Réseaux pleinement connectés}

Présentation haut niveau de leur utilité prouvée empiriquement
sur à peu près n'importe quel type de problème d'apprentissage
supervisé.
\begin{itemize}
    \item Préciser architecture : neurone, activation
    \item Préciser fonction de perte : doit être intimement connectée
          au savoir préalable (prior) sur la tâche et la cible.
    \item Optimisation : taux d'apprentissage, Adam vs SGD
\end{itemize}

\subsection{Réseaux récurrents}

\begin{itemize}
    \item RNN
    \item LSTM : pour quelle raison ?
\end{itemize}
\chapter{Bandit contextuel}
\label{chap:bandit_contextuel}

La première approche explorée est celle qui formule la génération de
résumés comme un bandit contextuel, telle qu'initialement proposée par \citep{dong2018banditsum}.
Dans ce chapitre, on présente la formulation en bandit contextuel, on présente 
comment elle s'applique à la génération de résumés et on valide son applicabilité 
sur un jeu de données de développement.
On montre ensuite comment la formulation en bandit contextuel peut
naturellement être utilisée avec l'algorithme REINFORCE \cite{williams1992simple}
pour obtenir une performance représentant l'état de l'art.

\section{Formulation contextuelle}

Un bandit contextuel est un problème de bandit où un contexte $c_t$ est perçu
avant de prendre une action $a_t$ et de percevoir une récompense $X_t$
possiblement dépendante du contexte $c_t$.
La présence de ce contexte permet d'apprendre à prédire des actions différentes
pour des entrées différentes.
En génération de résumés, cela correspond à l'hypothèse facilement vérifiable
que ce n'est pas nécessairement une bonne stratégie de toujours sélectionner
les mêmes index de phrases d'un document pour bâtir son résumé.

Concrètement, pour une paire document-résumé $(d, s)$, on définit le contexte
comme étant le document ($c=d$), les actions $\mathcal{A}$ sont les groupes non-ordonnés de 3 phrases
possibles à partir de $d$ et les récompenses représentent la valeur de $R(\hat{s}, s)$,
où $\hat{s}$ est le résumé bâti à partir de l'action retenue.
On ne s'intéresse pas à l'ordre des 3 phrases d'une action car, tel que mentionné
à la section \ref{sec:rouge}, le score $R$ utilisé est presque totalement invariant à l'ordre.
Ainsi, on choisit de générer le résumé $\hat{s}$ en fonction de $a \in \mathcal{A}$ en insérant les phrases
dans l'ordre dans lequel elles apparaissent dans le document original.


\section{Approximation de la performance par échantillonnage}

Selon le formalisme de génération de résumés défini à le section \ref{sec:extractive},
on peut donc dire que l'on a un bandit $\pi$ qui, pour un document $d$ en entrée, retourne
une distribution $\pi(d) \in [0,1]^{|d|}$.
Si l'on considère un bandit $\pi_\theta$ doté d'une certaine paramétrisation $\theta$,
l'algorithme REINFORCE (\ref{subsec:rl_summ}) peut être utilisé pour faire une ascension
de gradient pour la fonction de récompense $J(\theta)$ selon \eqref{eq:REINFORCE_sample}.
Rappelons ici que $J(\theta)$ représente la récompense espérée en pigeant un document $d$
et en utilisant $\pi_\theta$ pour générer son résumé.
En générant les résumés de manière stochastique, on permet à $J$ de tenir compte
de tous les résumés possibles de $d$ pondérés par leur probabilité selon $\pi_\theta(d)$,
permettant de distinguer des distributions qui accordent une plus grande probabilité aux
meilleurs résumés.
On emploie donc le processus $\xi$ de génération de résumé stochastique
qui pige sans répétition 3 phrases selon $\pi_\theta(d)$.

En pratique, les approximations de $\nabla J(\theta)$ générées en utilisant un seul
échantillon de $J$ sont très instables pour un processus stochastique $\psi$
et rendent l'apprentissage difficile.
Une solution simple et peu coûteuse pour stabiliser l'entraînement est de piger $N$
résumés $\hat{s_t}$ en fonction de $\xi(\pi)$ pour bâtir un meilleur estimateur selon

\begin{equation}
    \nabla J(\theta) = \frac{1}{N} \sum_{t=1}^N R(\hat{s_t}, s) \nabla \ln \xi\left( \pi_\theta(\hat{s_t})\right).
    \label{eq:REINFORCE_batch}
\end{equation}

\subsection{Expériences}

On s'intéresse à savoir combien d'échantillons sont requis pour obtenir une représentation 
adéquate de la valeur de $\nabla J(\theta)$.
On note d'abord que, dans l'équation \eqref{eq:REINFORCE_batch}, la dérivée de $\xi(\pi)$ est déterministe
et n'est pas sensible à la portion stochastique du processus de génération de résumés.
Pour mesurer la qualité de l'approximation, il faut donc seulement s'intéresser
à la différence entre la véritable valeur de $J(\theta)$ et son estimation 
basée sur des échantillons selon $\xi(\pi)$.
Enfin, comme la pige d'un document $d$ dans le calcul de $J$ est uniforme, on
peut simplement s'intéresser à l'estimation de $J$ sur un seul document à la fois.


On prend donc $J$ comme étant l'espérance de récompense de $\xi(\pi)$ sur une paire 
document résumé $(d, s)$ quelconque et on pose
 $\bar{J_N}(\theta) =  \frac{1}{N} \sum_{t=1}^N R(\hat{s_t}, s)$, son approximation par échantillonnage.
La qualité de l'échantillonnage dépend de la proximité entre $J$ et $\bar{J_N}$, que
nous nommons l'erreur d'échantillonnage $\Delta_N$ pour

\begin{equation}
    \Delta_N = \left| J(\theta) - \bar{J}_N(\theta) \right|.
\end{equation}

Le pseudocode permettant de générer l'approximation $\bar{J_N}$ se trouve 
dans l'algorithme \ref{alg:approx_J}.

\begin{algorithm}[!h]
    \caption{Calcul de $\bar{J_N}$}
    \begin{algorithmic}[1]
        \Require $\pi(d)$ (distribution de phrases), $N$ (nombre d'échantillons), $s$ (résumé cible).
        \For{$t=1, ..., N$}
        \State Piger $\hat{s_t}\sim \xi\left(\pi(d)\right)$
        \EndFor
        \State $\hat{J_N} = \frac{1}{N} \sum_{t=1}^N R(\hat{s_t}, s)$
    \end{algorithmic}
    \label{alg:approx_J}
\end{algorithm}

\subsubsection*{Pré-calcul des scores $R$}

Les méthodes présentées dans ce mémoire utilisent toutes
un grand nombre de scores $R$ dans leur procédure d'entraînement, dont le calcul
est plutôt lent.
Or, pour un document $d$ donné, on sait que tous les résumés que l'on doit considérer sont 
l'ensemble des combinaisons des phrases ordonnées selon leur position dans 
le document initial.
Il est donc possible de calculer le score $R(\hat{s}, s)$ de tous les résumés $\hat{s}$
dans un premier temps et de simplement aller lire la valeur dans 
un tableau correspondant lorsque nécessaire.
On a donc, dans un premier temps, précalculé tous les scores $R$
associés à tous les résumés pour chaque document du jeu de données.

\subsubsection*{Méthodologie}

On valide la rapidité de la convergence de $\bar{J}$ vers $J$ en les comparant
directement sur un sous-ensemble de 25 000 documents d'entraînement du jeu 
de données CNN/DailyMail.
Ce jeu de développement, qui sera réutilisé pour les chapitres suivants, 
se veut aussi varié que possible.
Notamment, il contient 8 674 documents de 20 phrases ou moins, 10 490
documents contenant entre 20 et 35 phrases exclusivement et 5 796 documents 
de 35 phrases ou plus.

On pige aléatoirement des distributions $\pi(d)$ sur les phrases de chacun des documents.
Pour assurer une bonne variété dans les distributions explorées, on choisit des
distributions représentant une moyenne pondérée entre une distribution uniforme
$U(d)$ et une distribution vorace $G(d)$ :

\begin{equation}
    \pi(d) = (1 - \tau) U(d) + \tau G(d).
\end{equation}

La distribution vorace employée $G(d)$ représente un vecteur de taille $|d|$ où trois indices 
pigés aléatoirement ont la valeur de $1/3$ et les autres indices sont nuls.
On explore $\tau$ variant de 0 à 1 en incréments de 0.01 et en repigeant les indices 
non-nuls de $G(d)$ à chaque fois.

Les expériences consistent à calculer la valeur de $\Delta_N$ pour différentes valeurs
de $N$ allant jusqu'à 100 sur tous les documents du jeu de développement.
L'approximation $\hat{J_N}$ est obtenue en suivant l'algorithme \ref{alg:approx_J}.
On calcule analytiquement la véritable valeur de $J$ grâce au fait que l'on possède tous les scores
$R$ de tous les résumés possibles pour les documents du jeu de données.

\subsection{Résultats}

Les résultats obtenus sont rapportés dans les figures \ref{fig:bandit_contextuel_overall}
et \ref{fig:bandit_contextuel_top3}.

Tout d'abord, sur la figure \ref{fig:bandit_contextuel_overall}, on remarque que le nombre de phrases dans un document n'est pas un
facteur déterminant sur la rapidité de convergence de l'approximation par échantillonnage.
Ce résultat n'est pas surprenant: on peut s'attendre à ce que la difficulté
de convergence soit davantage qualifiée par une mesure sur la distribution
des résumés.

\tikzsetnextfilename{bandit_contextuel_overall}
\begin{figure}[h!]
    \begin{center}
        \begin{tikzpicture}
            \begin{axis}[title={Erreur de l'échantillonnage selon la taille des documents}, grid style={dashed,gray!50}, axis y line*=left, axis x line*=bottom, every axis plot/.append style={line width=1.5pt, mark size=0pt, font=\huge}, name=plot0, xshift=-.1\textwidth, y tick label style={/pgf/number format/fixed zerofill}, ylabel={$\Delta_t$}, xlabel={Nombre d'échantillons}, width=0.95\textwidth, height=0.4\textwidth, smooth, xmin=0, xmax=50, y tick label style={/pgf/number format/fixed}, scaled y ticks = false, ytick={0.01, 0.02, 0.03, 0.04, 0.05}]
                \addplot[blue, ylabel near ticks, line width=3pt] table[x=x, y=y, col sep=comma]{bandit_contextuel/bandit_contextuelExpResults/overall/mean.csv};
                \addplot[red, ylabel near ticks, line width=3pt] table[x=x, y=y, col sep=comma]{bandit_contextuel/bandit_contextuelExpResults/overall/less20.csv};
                \addplot[green, ylabel near ticks, line width=3pt] table[x=x, y=y, col sep=comma]{bandit_contextuel/bandit_contextuelExpResults/overall/20to35.csv};
                \addplot[yellow, ylabel near ticks, line width=3pt] table[x=x, y=y, col sep=comma]{bandit_contextuel/bandit_contextuelExpResults/overall/more35.csv};
                \legend{Moyenne, $|d| \leq 20$, $20 < |d| < 35$, $35 \leq |d|$}
            \end{axis}
        \end{tikzpicture}
    \end{center}
    \caption{Impact de la taille du document en entrée sur l'erreur d'échantillonnage.}
    \label{fig:bandit_contextuel_overall}
\end{figure}

Une mesure naturelle sur la distribution des résumés est la valeur maximale de
la distribution des résumés.
La figure \ref{fig:bandit_contextuel_top3} nous montre une courbe à la tendance logarithmique,
où plus la probabilité maximale de la distribution augmente, plus rapide est la convergence
de notre processus d'échantillonnage.

\tikzsetnextfilename{bandit_contextuel_top3}
\begin{figure}[h!]
    \begin{center}
        \begin{tikzpicture}
            \begin{axis}[title={Erreur de l'échantillonnage selon la plus grande probabilité présente}, grid style={dashed,gray!50}, axis y line*=left, axis x line*=bottom, every axis plot/.append style={line width=1.5pt, mark size=0pt, font=\huge}, name=plot0, xshift=-.1\textwidth, y tick label style={/pgf/number format/fixed zerofill}, ylabel={$\Delta_t$}, xlabel={Probabilité maximale de $\xi(\pi(d))$}, width=0.95\textwidth, height=0.4\textwidth, smooth, yticklabel style={/pgf/number format/.cd,fixed,precision=3},, scaled y ticks = false, ymin=0.0, ytick={0.005, 0.010, 0.015, 0.020}, xmin=0.05]
                \addplot[blue, ylabel near ticks, line width=3pt] table[x=x, y=t8, col sep=comma]{bandit_contextuel/bandit_contextuelExpResults/top3/all.csv};
                \addplot[red, ylabel near ticks, line width=3pt] table[x=x, y=t16, col sep=comma]{bandit_contextuel/bandit_contextuelExpResults/top3/all.csv};
                \addplot[green, ylabel near ticks, line width=3pt] table[x=x, y=t32, col sep=comma]{bandit_contextuel/bandit_contextuelExpResults/top3/all.csv};
                \addplot[yellow, ylabel near ticks, line width=3pt] table[x=x, y=t64, col sep=comma]{bandit_contextuel/bandit_contextuelExpResults/top3/all.csv};
                \legend{$t=8$, $t=16$, $t=32$, $t=64$}
            \end{axis}
        \end{tikzpicture}
    \end{center}
    \caption{Impact de la probabilité maximale de la distribution des résumés sur l'erreur d'échantillonnage.}
    \label{fig:bandit_contextuel_top3}
\end{figure}

Globalement, on remarque aussi que la convergence est rapide.
Une différence de moins de 0.01 $R$ est définitivement suffisante pour justifier 
l'utilisation de $\hat{J_N}$ au lieu de $J$.
Aussi, les résultats indiquent que la rapidité de la convergence ralentit progressivement autour 
de $N=16$.
Comme il est plus coûteux en temps de calcul de faire plus d'échantillons, il
appraît naturel de considérer $N=16$ comme un excellent compromis entre 
le temps de calcul et la rapidité de l'estimation.
Notons que les articles de \citep{dong2018banditsum} et \citep{luo-etal-2019-reading},
qui utilisent cet échantillonnage, utilisent $N=20$ dans leurs expériences.

\section{BanditSum}

Maintenant que l'on sait que l'on peut avoir des approximations très justes pour
$J$ (donc $\nabla_\theta J$), on peut utiliser l'algorithme REINFORCE pour apprendre
les paramètres $\theta$ d'un système complet.
En effet, on peut considérer un réseau de neurones $h_\theta$ prenant en entrée un document et produisant
une distribution sur les phrases comme un bandit contextuel, que l'on peut apprendre
efficacement avec REINFORCE.

\subsection{Description}

Le système présenté est attributé à \citep{dong2018banditsum}, qui l'ont nommé 
BanditSum.
Dans BanditSum, on considère un réseau de neurones $h_\theta$ comme un bandit contextuel,
lequel a pour objectif de maximiser le score des résumés qu'il produit pour l'ensemble 
des documents $d$ d'un jeu de données.

L'architecture retenue pour $h$ est séparée en un encodeur de phrases et une tête 
de prédiction d'affinités.
Pour l'encodeur de phrases, un LSTM bidirectionnel à une couche prend en entrée
la séquence des plongements de mots de chaque phrase et retourne une représentation de 
chaque phrase par rapport aux mots qu'elle contient.
Ces représentations sont par la suite fournies à un LSTM bidirectionnel à deux couches qui 
produit une représentation $\tilde{d_i}$ de chaque phrase par rapport aux autres phrases du même 
document.
La tête de prédiction est constituée d'un réseau pleinement connecté prenant en entrée 
les représentations $\tilde{d_i}$ et produisant une affinité entre 0 et 1 pour chaque phrase.
Les affinités liées à chaque phrase sont enfin regroupées et divisées par leur somme 
pour produire la distribution $\pi_\theta(d)$ sur les phrases du document.
À l'entraînement, les paramètres $\theta$ sont appris via REINFORCE en générant 
les résumés avec le processus de génération stochastique $\xi$.
À l'inférence, c'est le processus vorace $\psi$ qui est employé, sélectionnant les 
3 phrases avec la plus grande affinité.

\todo{Figure de l'architecture de BanditSum.}

Banditsum incorpore aussi deux artefacts techniques pour faciliter la convergence.
Ceux-ci sont énumérés plus bas et accompagnés de l'intuition justifiant leur 
utilisation.

D'abord, BanditSum utilise la version de REINFORCE incorporant une \textbf{baseline} $b(d)$:
\begin{equation}
    \nabla J(\theta) = \frac{1}{N} \sum_{t=1}^N\left(R(\hat{s_t}, s) - b(d)\right)\nabla \ln \xi\left(\pi_\theta (\hat{s_t})\right).
    \label{eq:REINFORCE_baseline}
\end{equation}
La baseline retenue est $b(d) = R\left(\psi(\pi), s\right)$, représentant le score associé au résumé
le plus probable document $d$.
L'introduction de $b$ peut être vue comme assignant un scalaire positif 
aux résumés meilleurs que le résumé actuellement privilégié et un sclaire négatif à ceux qui 
sont moins bons.
En pratique, l'introduction de baseline est fréquemment utilisée pour réduire la variance 
élevée dans l'entraînement avec REINFORCE et permettre un apprentissage plus stable.

Le second artefact employé par BanditSum est celui de l'ajout d'une exploration artificielle
$\epsilon$ dans le processus de pige de résumé $\xi$.
Afin d'assurer une exploration satisfaisante des résumés possibles d'un document, le processus 
de génération stochastique $\xi$ qu'ils utilisent pige une phrase au hasard dans une 
proportion $\epsilon=0.1$ du temps.
L'introduction de $\epsilon$ a pour effet d'assurer que les résumés pigés selon $\xi(\pi, \epsilon)$ seront 
suffisamment variés pour assurer une bonne exploration de l'espace des résumés possibles lors de
l'entraînement.
Enfin, notons que la probabilité de générer un résumé $\hat{s}$, représentée par
 $\xi \left(\pi_\theta(\hat{s_t}), \epsilon \right)$ dans l'équation
 \ref{eq:REINFORCE_baseline}, s'écrit alors sous la forme close
\begin{equation}
    \xi \left(\pi_\theta(\hat{s_t}), \epsilon \right) = \displaystyle \prod_{i=1}^3 \left(\dfrac{\epsilon}{|d| -i + 1}  + \dfrac{(1 - \epsilon)\pi(d)_{s_i}}{\sum_{k=1}^{j-1} \pi(d)_{s_j}}\right),
    \label{eq:Xi}
\end{equation}
où $s_i$ représente l'index de la $i$-ème phrase du résumé $\hat{s}$. 
Ainsi, le gradient de \eqref{eq:Xi}, nécessaire dans la mise à jour des paramètres 
par REINFORCE, est facilement calculable avec les logiciels de différentiation automatique
habituellement utilisés pour les réseaux de neurones.

\subsection{Expériences}

On roule notre propre implémentation de BanditSum, dont le pseudocode se trouve à l'algorithme 
\ref{alg:Banditsum}, sur le jeu de données CNN/DailyMail complet.
Pour les expériences, on fait varier $N$, pour explorer
l'importance que peut avoir un meilleur estimé du gradient sur le temps de calcul
vs la rapidité de convergence.

\begin{algorithm}
    \caption{Système utilisant un bandit contextuel}
    \begin{algorithmic}[1]
        \Require  $\mathcal{D}$ (jeu de données), $h_\theta$ (modèle neuronal), $N$ (nombre d'échantillons), $\alpha$ (taux d'apprentissage), $\epsilon$ (taux d'exploration).
        \While{vrai}
        \State{$(d, s) \sim \mathcal{D}$} \Comment{On pige un document du jeu de données}
        \State $\pi = h_\theta(d)$
        \State Piger $N$ résumés $\hat{s_i} \sim \xi(\pi, \epsilon)$
        \State Calculer les $N$ scores associés $r_i = R(\hat{s_i}, s)$
        \State $b = R(\psi(\pi), s)$
        \State $\nabla = \frac{1}{N} \sum_{i=1}^N (r_i - b) \nabla_\theta \ln \xi \left(\pi(\hat{s_i}), \epsilon \right)$ \Comment{Mise à jour REINFORCE avec baseline}
        \State $\theta = \theta + \alpha \nabla$
        \EndWhile
    \end{algorithmic}
    \label{alg:Banditsum}
\end{algorithm}

\subsubsection*{Détails d'implémentation}

Nous reprenons exactement les mêmes paramètres d'entraînement que \citep{dong2018banditsum},
que nous énumérons.

Pour les LSTMs, la dimension cachée est fixée à 200.
La tête de prédiction est un réseau pleinement connecté avec une seule couche cachée
de taille 100, activation ReLU \citep{agarap2018learning} et sortie sigmoïde.

Les plongements de mots utilisés sont les plongements GLoVe \citep{pennington2014glove} de taille 100 
pré-entraînés sur la langue anglaise.
L'optimiseur employé est Adam \citep{kingma2014method}, pour lequel on fixe 
$\beta = [0, 0.999]$.
On utilise un taux d'apprentissage $\alpha=5e^{-5}$ et une pénalité sur la norme 
des poids $\theta$ de $1e^{-6}$.
Un \textit{gradient clipping} de 1 est aussi appliqué.
\todo{GPU et CPU utilisés.}

\subsection{Résultats}


Rapporter graphe de l'évolution de $R$ en validation selon le nombre de documents vus.
En validation et en test, on calcule le vrai ROUGE pour éviter toute possibilité 
d'erreur introduite dans le pré-calcul.
On garde quand même l'heuristique que l'ordre des phrases du résumé est le même que celui 
dans le document, car c'est l'ordre qui risque le moins d'introduire des incohérences
syntaxiques (références mal-définies).
Rapporter une courbe pour chaque $N$ parmi $\{8,16,32\}$, qui sont les valeurs les plus
pertinentes selon les expés plus haut.
Pour chacune des valeurs de $N$ à l'essai, on roule l'algorithme à 5 reprises 
en faisant varier le générateur de nombres aléatoires et on rapporte la 
moyenne et déviation standard.
Rapporter la courbe que BanditSum dit obtenir pour $N=20$.
Rouler 4 epochs, selon ce que BS rapporte comme meilleur modèle.

Dans un tableau, rapporter performance en test de Banditsum (nous) vs Banditsum (paper) et
SOTA extractif.
Rapporter ROUGE-1, ROUGE-2, ROUGE-L et leur moyenne.

Dans un autre tableau, rapporter temps d'entraînement et architecture de calcul 
utilisée pour montrer l'impact de $N$ sur le temps de calcul.
Discussion sur non-nécessité de vrai gradient de $J$ pour l'apprentissage, mais que
ça stabilise naturellement.

\section{Conclusion}

\todo{à faire une fois les résultats de BanditSUM en main}
\chapter{Bandit combinatoire}
\label{chap:bandit_combi}                   % étiquette pour renvois (à compléter!)

Un problème fondamental de l'apprentissage de 
systèmes de génération de résumés extractifs est l'indisponibilité
de cibles naturelles.
À cet effet, une façon de faire attrayante est de 
calculer un oracle (\ref{sec:extractive}) 
qui génère d'excellents résumés extractifs que 
l'on utilise comme cibles pour un entraînement.
Habituellement, ces cibles sont binaires et représentent 
l'inclusion ou non d'une phrase dans le résumé produit 
par l'oracle. 
L'utilisation de cibles binaires n'est pas sans faute.
Si 3 phrases peuvent constituer le résumé extractif optimal 
selon un oracle, il est bien possible que certaines 
autres phrases produisent tout de même d'excellents résumés 
et se voient assigner une cible nulle.

Nous proposons une approche basée sur un problème 
de bandit combinatoire permettant de générer des cibles 
plus riches que les cibles binaires.
En pénalisant moins drastiquement les phrases qui ne font 
pas partie du résumé extractif optimal mais qui 
génèrent d'excellents résumés extractifs, nos cibles 
permettent intuitivement de quantifier la \textit{qualité 
extractive} d'une phrase.

Dans ce chapitre, on présente comment la génération du résumé d'un document
peut être interprétée un problème de bandit combinatoire \citep{CESABIANCHI20121404}.
On monte ensuite comment l'algorithme Upper Confidence Bound (UCB) \citep{ucb} peut être utilisé 
pour résoudre ce type de problème et comment il peut être employé pour 
générer des cibles pour un système de génération de résumés.
L'applicabilité des cibles proposées est par la suite validée empiriquement 
à l'aide d'un jeu de données de développement.
Enfin, on présente comment ces cibles peuvent être utilisées pour 
l'apprentissage de systèmes de génération de résumés en proposant 
un nouvel algorithme que l'on nomme CombiSum.
On met CombiSum à l'épreuve sur notre cadre expérimental 
uniformisé pour juger de son efficacité.

\section{Formulation combinatoire}
\label{section:formulation_combi}

Le bandit combinatoire est une approche par bandit qui vise à étendre
la formulation multi-bras à des environnements où plusieurs actions
$a \in \mathcal{A}$ peuvent être sélectionnées
en même temps par l'apprenant.
Nous nous intéressons seulement à la variante dite multitâche \citep{banditalgs}, où
il y a un nombre fixé $m$ d'actions sélectionnées à
chaque pas de temps $t$, créant une \textit{super-action} $\mathcal{M} = \{a_1, ..., a_m\}$
pour laquelle l'environnement retourne une récompense $X_{\mathcal{M}, t}$.
L'objectif demeure la minimisation du pseudo-regret cumulatif \eqref{eq:regret_cumulatif}, 
mais celui-ci est désormais calculé entre
la \textit{super-action} optimale $\mathcal{M}^*$ et les \textit{super-actions}
choisies $\mathcal{M}_t$.

Tout l'intérêt de cette formulation vient de l'éventuelle relation entre les \textit{super-actions}.
Bien qu'il serait possible de formuler le bandit multitâche comme un bandit multi-bras où l'ensemble des
actions possibles est l'ensemble des \textit{super-actions} $\mathcal{M}$, cette formulation
risque d'être très peu efficace.
En effet, comme les actions $a \in \mathcal{A}$ sont partagées entre les diverses \textit{super-actions},
il est possible d'utiliser la récompense associée à une \textit{super-action} pour
déduire de l'information sur la récompense associée aux actions qui la composent.
Cette information sur les actions $a$ peut alors être utilisée pour guider le choix des
prochaines \textit{super-actions} et accélérer l'apprentissage.

\subsection{Application à la génération de résumé}

La génération du résumé d'un document $d$ peut être vue comme un bandit combinatoire où
les actions de base sont les phrases $d_i$ et les \textit{super-actions} $\mathcal{M}$
sont les résumés de 3 phrases possibles.
En choisissant la \textit{super-action} $\mathcal{M}_{\hat{s}}$ associée au résumé $\hat{s}$,
l'apprenant reçoit $\text{ROUGE}(\mathcal{M}_{\hat{s}}, s)$ comme récompense et peut mettre à jour ses
croyances sur les phrases $d_i$ de $\hat{s}$.
Pour alléger la notation, nous dirons désormais que la \textit{super-action} 
$\mathcal{M} = \{a_{i_1},a_{i_2},a_{i_3}\}$ est un résumé extractif 
où l'action $a_i$ correspond à la phrase $d_i$.

Notons d'abord que, comme on ne tient pas compte de l'ordre dans le cadre 
expérimental uniformisé décrit à la section \ref{section:problematique},
il est naturel de considérer que chaque phrase d'un résumé contribue à part égale
au score observé.
Implicitement, cela revient à faire l'hypothèse que la récompense associée à un résumé
$\mathcal{M}$ est la moyenne de la récompense associable à chaque phrase.
Or, comme le score ROUGE est basé sur un score F1 sur le résumé complet, cette hypothèse
n'est pas strictement vraie mais elle est intuitivement valide.
En effet, les phrases qui résument bien le document généreront des résumés aux scores
plus élevés et donc leur présence à elle seule doit contribuer à augmenter le score
associé à un résumé.

Maintenant que l'on a établi le cadre théorique entourant le problème de bandit 
combinatoire, une question demeure: comment peut-on le résoudre ?
La prochaine section répond à cette question en démontrant 
comment l'algorithme UCB peut être légèrement modifié pour 
être appliqué au bandit combinatoire.

\subsection{UCB pour bandit combinatoire}

Tel que décrit à la section \ref{subsec:ucb}, Upper Confidence Bound (UCB) est une approche permettant 
de minimiser le pseudo-regret cumulatif pour les problèmes de bandit stochastique.
Pour ce faire, UCB maintient des bornes supérieures UCB$_a(t)$ pour chaque action 
$a \in \mathcal{A}$ et chaque pas de temps $t$

\begin{equation}
    \text{UCB}_a(t) = \bar{x}_a(t) + \beta \sqrt{\frac{2\ln t}{n_a(t)}},
    \label{eq:UCB}
\end{equation}
où $\bar{x}_a(t)$ est la moyenne des récompenses reçues jusqu'au temps $t$ par l'action $a$,
$n_a(t)$ est le nombre de fois que l'action $a$ a été sélectionnée et $\beta$ est un
hyperparamètre régulant l'exploration.
Quand une action n'a pas encore été sélectionnée i.e. $n_a(t)=0$, on 
lui assigne UCB$_a = \infty$, forçant l'apprenant à sélectionner au moins 
une fois chaque action avant d'exploiter les moyennes $\bar{x}_a(t)$
dans son choix d'action.

Bien que UCB est conçu pour les problèmes de bandits stochastiques, 
l'approche peut être 
naturellement étendue au problème de bandit combinatoire.
En effet, il suffit alors de choisir à chaque pas de temps $t$ la \textit{super-action}
maximisant la borne supérieure définie par UCB sur les actions $a \in \mathcal{A}$.
Dans notre cas, cela revient à sélectionner la \textit{super-action} composée des trois actions
avec la borne supérieure maximale, i.e.

\begin{equation}
    \mathcal{M}_t = \text{3-}\argmax_{a \in \mathcal{A}} \text{UCB}_a(t),
    \label{eq:ucb_combi_rule}
\end{equation}

où UCB$_a(t)$ est défini selon \eqref{eq:UCB}.

La prochaine section décrit le premier volet de notre contribution:
l'utilisation de UCB pour la génération de cibles extractives 
d'un document. 

\section{Génération de cibles par UCB}
\label{sec:cibles_ucb}

Pour entraîner le réseau de neurones décrit à la section
\ref{subsec:archi}, il est possible d'employer 
des cibles extractives $y$ associées à chaque document $d$.
Intuitivement, ces cibles doivent représenter 
dans quelle mesure chacune des phrases $d_i$
correspond à une bonne phrase d'un point de vue extractif.
Naturellement, on prend $y_i \in [0, 1]$\footnote{Ici $y_i$ 
représente la cible associée à la phrase $d_i$. On utilisera aussi parfois $y_a$
pour représenter la cible associée à la phrase $a$.}, où un score 
près de 1 indique que la phrase $d_i$ permet de générer 
des résumés extractifs de haute qualité.

Sur le jeu de données CNN/DailyMail \citep{hermann2015teaching},
les cibles habituellement retenues sont produites par 
l'oracle proposé par \citet{10.5555/3298483.3298681} et décrit 
à la section \ref{sec:extractive}.
Mentionnons simplement que l'oracle est une heuristique qui 
génère un résumé extractif $\dot{s}$ de 3 phrases en sélectionnant 
une à une les phrases d'un document maximisant le score 
ROUGE.
Comme l'ordre des phrases n'impacte presque pas le score ROUGE (\ref{subsec:eval}),
le résumé choisi par l'oracle 
peut être vu comme un résumé de 3 phrases presque optimal
d'un document.
Ainsi, \citet{10.5555/3298483.3298681} proposent d'utiliser 
les cibles binaires indiquant la présence ou non d'une phrase 
dans le résumé $\dot{s}$ produit par l'oracle, i.e.

\begin{equation}
    y_i = \mathbb{I}\big[d_i \in \dot{s}\big],
    \label{eq:cibles_binaires}
\end{equation}
pour $\mathbb{I}$ la fonction identité retournant 1 quand 
son prédicat est vrai et 0 sinon.

En raison de leur nature binaire, les cibles obtenues selon \eqref{eq:cibles_binaires}
peuvent potentiellement être mal spécifiées.
En effet, si le résumé $\dot{s}$ produit par l'oracle est 
presque optimal au sens extractif, cela n'empêche pas 
que d'autres résumés $\dot{s}'$ peuvent obtenir un score ROUGE
très similaire. 
Or, les phrases de $\dot{s}'$ qui ne font pas partie de $\dot{s}$
se verront attribuer une cible de 0, alors qu'elles présentent 
tout de même une excellente \textit{extractivité}.

\subsection{Motivation}

Nous proposons un nouveau processus de génération de cibles 
permettant d'alléger ce problème en quantifiant 
directement dans quelle mesure une phrase est susceptible de générer 
des bons résumés extractifs.
Pour un document $d$, le processus exécute $T$ pas de temps UCB \eqref{eq:ucb_combi_rule}
sur le bandit combinatoire représentant la génération du résumé de $d$.
En minimisant le pseudo-regret, UCB identifie graduellement les 
bons résumés et, plus particulièrement, maintient une 
moyenne $\bar{x}_a$ des scores ROUGE reçus en sélectionnant la phrase $a$.

Étant donné un nombre de pas temps $T$ infini, UCB 
identifierait éventuellement le résumé $\mathcal{M}^*$ optimal.
À ce moment, les moyennes $\bar{x}_a$ convergeraient éventuellement à $R^*$, 
le meilleur score de résumé extractif d'un document, pour les phrases $a$ faisant
partie du résumé optimal.
Or, tout l'intérêt repose dans les moyennes $\bar{x}_a$ pour les 
phrases ne faisant pas partie du résumé optimal.
Comme UCB va progressivement choisir de meilleurs résumés en s'assurant 
de bien explorer l'espace des résumés possibles, la moyenne $\bar{x}_a$ associée 
à une phrase $a$ pourra servir d'estimateur de sa qualité extractive.

Intuitivement, la structure inhérente de la génération 
de résumés extractifs donne lieu à bon nombre de phrases aux 
potentiels extractifs similaires (quelques mots correspondant au résumé cible) 
et peu d'excellentes phrases, difficiles à discerner.
En minimisant le regret, UCB va donc naturellement obtenir plus d'échantillons de résumés contenant 
les excellentes phrases.
Or, comme les excellentes phrases sont limitées, il faut échantillonner plus de résumés les contenant
avant d'obtenir une bonne estimation de leur valeur, contrairement aux phrases insatisfaisantes 
dont la valeur est rapidement estimée après quelques échantillons.
Il est donc justifié d'utiliser les quantités mesurées par UCB pour obtenir de l'information 
sur toutes les phrases d'un document, pas seulement pour trouver le résumé optimal.

Concrètement, nous proposons d'utiliser les moyennes $\bar{x}_a$ produites 
par un UCB pour générer les cibles $y_a$ selon le processus suivant.
On commence par appliquer une mise à l'échelle min-max des moyennes $\bar{x}_a$.
Cette mise à l'échelle produit des moyennes normalisées $\bar{x}_a' \in [0, 1]$
en conservant proportions originales entre les moyennes.
Les $\bar{x}_a'$ sont ensuite utilisé pour produire les cibles $y_a$ selon

\begin{equation}
    y_a = 10^{-10(1 - \bar{x}_a')}.
    \label{eq:cibles_ucb}
\end{equation}

L'intuition derrière l'équation \eqref{eq:cibles_ucb} est qu'une phrase 
avec une moyenne 10 \% inférieure à une autre devrait se voir attribuer une cible 
10 fois moins élevée. 
Cette mise à l'échelle peut sembler drastique mais il faut se souvenir 
que l'on ne souhaite accorder d'importance qu'aux phrases qui produisent 
d'excellents résumés.
Enfin, grâce à la mise à l'échelle min-max préalable, une cible de 1 
sera toujours associée pour la meilleure phrase selon UCB.

\subsection{Expériences}

Si l'intuition de l'utilisation des quantités calculées par UCB comme cibles est naturelle
pour le cas où le nombre de pas de temps $T$ effectué est immense, il demeure néanmoins important
de vérifier comment ces cibles sont adéquates sur un nombre de pas de temps plus restreint.
Mais, comment savoir quand on approche de ce stade d'optimalité et que 
l'on peut considérer les cibles comme adéquates ?
Réponse: quand le score du résumé le plus prometteur selon les moyennes $\bar{x}_a$, nommons le $R_t$,
commence à converger vers le score $R^*$ du résumé extractif optimal.
En mesurant la sous-optimalité $\Delta_t = R^* - R_t$ des cibles générées
par UCB, on peut donc savoir à partir de quel moment les cibles seraient satisfaisantes
pour être utilisées en entraînement.

On reprend donc le jeu de développement présenté à la section \ref{subsec:jeu_donnees} 
pour mener des expériences sur la rapidité de l'obtention de cibles satisfaisante par UCB.
Les résultats des figures \ref{fig:bandit_combi_alpha}
et \ref{fig:bandit_combi_doc_len} sont obtenus en calculant les $\Delta_t$ à partir de
l'algorithme \ref{alg:cible_ucb}.
Notons que, pour mettre UCB à l'échelle de chaque document $d$, 
l'exploration que l'on utilise est plutôt guidée par $\beta' = \frac{\beta}{|d|}$.
Il est aussi à noter que les différences rapportées entre les différentes 
courbes sur les figures ne sont pas statistiquement significatives.
Pour éviter d'encombrer inutilement les figures, on rapporte alors 
seulement les moyennes observées.
Les versions incorporant la déviation standard des différentes courbes 
se trouvent à l'annexe \ref{chap:variance_graphs}.

\begin{algorithm}
    \setstretch{1.3}
    \caption{UCB combinatoire pour génération de résumé}
    \begin{algorithmic}[1]
        \Require $d$ (document), $s$ (résumé cible), $\beta$ (paramètre d'exploration), $T$ (nombre de pas de temps).
        \State Initialiser $\bar{x}_a =0$ et $n_a = 0$ pour $a \in d$
        \For{$t=1, ..., T$}
        \For{$a \in d$}
        \State UCB$_a = \bar{x}_a + \frac{\beta}{|d|} \sqrt{\frac{2 \ln t}{n_a}}$
        \EndFor
        \State $\mathcal{M}_t =$ 3-$\argmax$ UCB$_a$
        \State $r_t = \text{ROUGE}(\mathcal{M}_t, s)$
        \State Mettre à jour $\bar{x}_a$ et $n_a$ pour {$a \in \mathcal{M}_t$}
        \EndFor
    \end{algorithmic}
    \label{alg:cible_ucb}
\end{algorithm}

\subsection{Résultats}
\label{subsec:ucb_resultats}

% La figure \ref{fig:bandit_combi_q_argmax} présente l'importance du critère de sélection
% utilisé pour trouver le résumé optimal selon UCB au temps $t$.
% Deux approches sont envisageables: considérer les phrases $i$ où la moyenne des récompenses reçues
% $\bar{x}_i$ est maximale ou encore celles où le nombre de sélections $n_i$ est maximal.
% Les résultats rapportés donnent un net avantage aux résumés sélectionnés selon la moyenne des
% récompenses perçues.
% C'est normal: les $n_i$ prennent beaucoup plus de temps à distinguer les bonnes phrases
% car ils sont issus d'une distribution uniforme pour la sélection des $|d|$ premières phrases.
% Remarquons aussi que les deux approches convergent éventuellement au même résumé optimal,
% comme c'était à prévoir.
% Enfin, la conclusion retenue est que les $\bar{x}_i$ forment de meilleurs cibles que les $n_i$.
% Pour les prochaines figures, les résultats rapportés seront donc seulement ceux obtenus 
% selon le critère de sélection basés sur $\bar{x}_i$.

% \tikzsetnextfilename{bandit_combi_q_argmax}
% \begin{figure}[h!]
%     \begin{center}
%         \begin{tikzpicture}
%             \begin{axis}[title={Critère de sélection du résumé retenu}, grid style={dashed,gray!50}, axis y line*=left, axis x line*=bottom, every axis plot/.append style={line width=1.5pt, mark size=0pt, font=\huge}, name=plot0, y tick label style={/pgf/number format/fixed zerofill}, ylabel={$\Delta_t$}, xlabel={$t$}, width=0.95\textwidth, height=0.4\textwidth, smooth, xmin=0, xmax=100, y tick label style={/pgf/number format/fixed}]
%                 \addplot[green, ylabel near ticks, line width=3pt] table[x=t, y=arg_1e1, col sep=comma]{bandit_combi/bandit_combiExpResults/alpha/all.csv};
%                 \addplot[blue, ylabel near ticks, line width=3pt] table[x=t, y=q_1e1, col sep=comma]{bandit_combi/bandit_combiExpResults/alpha/all.csv};
%                 \legend{$n_i$, $\bar{x_i}$}
%             \end{axis}
%         \end{tikzpicture}
%     \end{center}
%     \caption{Impact du critère de sélection utilisé pour choisir le résumé retenu au temps $t$.}
%     \label{fig:bandit_combi_q_argmax}
% \end{figure}

La figure \ref{fig:bandit_combi_alpha} présente comment l'hyperparamètre $\beta$ influence la
convergence de UCB.
On remarque que $\beta=10$ (courbe verte sur la figure) semble être le bon compromis entre exploration et exploitation,
réussissant à converger à $\Delta_t \approx 0.05$ après 250 pas de temps de UCB.

\tikzsetnextfilename{bandit_combi_alpha}
\begin{figure}[h!]
    \begin{center}
        \begin{tikzpicture}
            \begin{axis}[title={Impact de l'exploration sur la progression de la sous-optimalité}, grid style={dashed,gray!50}, axis y line*=left, axis x line*=bottom, every axis plot/.append style={line width=1.5pt, mark size=0pt, font=\huge}, name=plot0, y tick label style={/pgf/number format/fixed zerofill}, ylabel={$\Delta_t$}, xlabel={$t$}, width=0.95\textwidth, height=0.4\textwidth, smooth, xmin=0, xmax=250, y tick label style={/pgf/number format/fixed}]
                \addplot[red, ylabel near ticks, line width=3pt] table[x=t, y=q_1e0, col sep=comma]{bandit_combi/bandit_combiExpResults/alpha/all.csv};
                \addplot[green, ylabel near ticks, line width=3pt] table[x=t, y=q_1e1, col sep=comma]{bandit_combi/bandit_combiExpResults/alpha/all.csv};
                \addplot[blue, ylabel near ticks, line width=3pt] table[x=t, y=q_1e2, col sep=comma]{bandit_combi/bandit_combiExpResults/alpha/all.csv};
                \legend{$\beta = 1$, $\beta = 10$, $\beta = 100$}
            \end{axis}
        \end{tikzpicture}
    \end{center}
    \caption{Impact du paramètre d'exploration $\beta$ utilisé par UCB sur la convergence.
    $\Delta_t$ représente la différence entre le score ROUGE du meilleur résumé selon UCB et 
    celui généré par l'oracle.}
    \label{fig:bandit_combi_alpha}
\end{figure}

La figure \ref{fig:bandit_combi_doc_len} évalue dans quelle mesure la taille des documents
influe sur la sous-optimalité des cibles générées par UCB.
On remarque d'abord que $\beta=10$ (courbe verte) est la meilleure configuration pour toutes les tailles
de document.
Il est donc adéquat de prendre la même valeur de $\beta$ pour tous les documents.

Aussi, on remarque que l'apprentissage converge plus tôt pour les documents qui sont
plus courts.
Il serait donc pertinent de faire croître le nombre de pas de temps de UCB en fonction du nombre de
phrases dans un document.
À cette fin, on remarque que la performance stagne autour de 100 pas de temps pour les documents 
courts mais continue à augmenter jusqu'à 250 pas de temps pour les longs documents.

\begin{figure}[h!]
    \tikzsetnextfilename{bandit_combi_less20}
    \begin{tikzpicture}[baseline]
        \begin{axis}[grid style={dashed,gray!50}, axis y line*=left, axis x line*=bottom, every axis plot/.append style={line width=1.5pt, mark size=0pt, font=\Large}, width=0.35\textwidth,
                height=0.4\textwidth, name=plot0, y tick label style={/pgf/number format/fixed zerofill}, xmin=0.0, xmax=250.0, ymin=0.01, ymax=0.30, ylabel={$\Delta_t$}, xlabel={$t$}, smooth, title={$|d| \leq 20$}]
            \addplot[red, ylabel near ticks, line width=3pt] table[x=t, y=less20_1e0, col sep=comma]{bandit_combi/bandit_combiExpResults/doc_len/all.csv};
            \addplot[green, ylabel near ticks, line width=3pt] table[x=t, y=less20_1e1, col sep=comma]{bandit_combi/bandit_combiExpResults/doc_len/all.csv};
            \addplot[blue, ylabel near ticks, line width=3pt] table[x=t, y=less20_1e2, col sep=comma]{bandit_combi/bandit_combiExpResults/doc_len/all.csv};
            \legend{$\beta=1$, $\beta=10$, $\beta=100$}
        \end{axis}
    \end{tikzpicture}
    \tikzsetnextfilename{bandit_combi_20to35}
    \begin{tikzpicture}[baseline]
        \begin{axis}[grid style={dashed,gray!50}, axis y line*=left, axis x line*=bottom, every axis plot/.append style={line width=1.5pt, mark size=0pt, font=\huge}, width=.35\textwidth,
                height=0.4\textwidth, name=plot0, xshift=-.1\textwidth, y tick label style={/pgf/number format/fixed zerofill},xmin=0.0, xmax=250.0, ymin=0.01, ymax=0.30, xlabel={$t$}, legend style={at={(0.9,0.1)},anchor=south east}, smooth, title={$20 < |d| < 35$}, ymajorticks=false]
            \addplot[red, ylabel near ticks, line width=3pt] table[x=t, y=20to35_1e0, col sep=comma]{bandit_combi/bandit_combiExpResults/doc_len/all.csv};
            \addplot[green, ylabel near ticks, line width=3pt] table[x=t, y=20to35_1e1, col sep=comma]{bandit_combi/bandit_combiExpResults/doc_len/all.csv};
            \addplot[blue, ylabel near ticks, line width=3pt] table[x=t, y=20to35_1e2, col sep=comma]{bandit_combi/bandit_combiExpResults/doc_len/all.csv};
        \end{axis}
    \end{tikzpicture}
    \tikzsetnextfilename{bandit_combi_more35}
    \begin{tikzpicture}[baseline]
        \begin{axis}[grid style={dashed,gray!50}, axis y line*=right, axis x line*=bottom, every axis plot/.append style={line width=1.5pt, mark size=0pt, font=\huge}, width=.35\textwidth,
                height=0.4\textwidth, name=plot0, xshift=-.1\textwidth, y tick label style={/pgf/number format/fixed zerofill}, xmin=0.0, xmax=250.0, ymin=0.01, ymax=0.30, xlabel={$t$}, legend style={at={(1,0.1)},anchor=south east}, smooth, title={$35 \leq |d|$}]
            \addplot[red, ylabel near ticks, line width=3pt] table[x=t, y=more35_1e0, col sep=comma]{bandit_combi/bandit_combiExpResults/doc_len/all.csv};
            \addplot[green, ylabel near ticks, line width=3pt] table[x=t, y=more35_1e1, col sep=comma]{bandit_combi/bandit_combiExpResults/doc_len/all.csv};
            \addplot[blue, ylabel near ticks, line width=3pt] table[x=t, y=more35_1e2, col sep=comma]{bandit_combi/bandit_combiExpResults/doc_len/all.csv};
        \end{axis}
    \end{tikzpicture}
    \caption{Impact de la taille du document sur la convergence de UCB.
    $\Delta_t$ représente la différence entre le score ROUGE du meilleur résumé selon UCB et 
    celui généré par l'oracle.}
    \label{fig:bandit_combi_doc_len}
\end{figure}


\section{CombiSum}
\label{sec:combisum}

On propose maintenant un algorithme basé sur les cibles 
générées par UCB: CombiSum.
Conformément au cadre expérimental uniformisé présenté à la section 
\ref{section:problematique},
CombiSum réutilise exactement la même architecture neuronale que BanditSum.
Conformément aux méthodes de l'état de l'art sur l'apprentissage 
de cibles \citep{liu2019text}, la mise à jour des paramètres est faite selon la 
perte basée sur l'entropie croisée binaire entre une cible $y$
et un vecteur d'affinités produit $\hat{y} = \pi_\theta(d)$:

\begin{equation}
    l(\hat{y}, y) = \frac{1}{|y|} \sum_{i=1}^{|y|}\big[y_i \log(\hat{y}_i) + (1 - y_i) \log(1 - \hat{y}_i) \big].
    \label{eq:bce}
\end{equation}

En se basant sur les résultats de la section précédente, on explore deux 
fonctions $T(d)$ déterminant le nombre $T$ de pas de temps utilisés pour 
générer les cibles UCB pour un document $d$.
Premièrement, on considère une version fixe $T_f(d)=100$, correspondant à une 
bonne estimation pour toutes les tailles de document.
On considère aussi une version $T_l(d)=2|d| + 50$ augmentant linéairement selon 
le nombre de phrases d'un document.
Comme notre régularisation limite les documents à 50 phrases ou moins, 
$T_l(d)$ varie entre 50 et 150, avec une moyenne (sur les tailles de document) à 
100.
Ainsi, $T_f$ et $T_l$ utiliseront le même nombre de pas de temps en moyenne 
et il sera possible de valider laquelle des deux fonctions permet 
le meilleur entraînement.

\subsection{Expériences}

Une description détaillée de CombiSum est fournie à l'algorithme \ref{alg:systeme_ucb}.
On effectue un entraînement sur le jeu de données CNN/DailyMail en 
parcourant dans son entièreté le jeu de d'entraînement à 5 reprises
et en utilisant des \textit{minibatches} de taille $B=64$.
Notre implémentation est faite selon les détails expérimentaux décrits à la section 
\ref{subsec:archi}.

\begin{algorithm}
    \setstretch{1.3}
    \caption{CombiSum}
    \begin{algorithmic}[1]
        \Require  $\mathcal{D}$ (jeu de données), $T$ (fonction pour le nombre de pas de temps), $\alpha$ (taux d'apprentissage), $\beta$ (taux d'exploration UCB), $B$ (taille de minibatch).
        \While{vrai}
        \State{batch $\sim \mathcal{D}^B$} \Comment{On pige la minibatch du jeu de données}
        \State $\nabla = \mathbf{0}$
        \ForAll{$(d,s) \in $ batch}
        \State $\hat{y} = \pi_\theta(d)$
        \State Exécuter $T(d)$ pas de temps de UCB et obtenir $\bar{x}$.
        \State Générer les cibles $y$ à partir de $\bar{x}$ \Comment{Selon \eqref{eq:cibles_ucb}}
        \State $\nabla = \nabla + \nabla_\theta l(\hat{y}, y)$ \Comment{Selon \eqref{eq:bce}}
        \EndFor
        \State $\theta = \theta - \alpha \nabla$
        \EndWhile
    \end{algorithmic}
    \label{alg:systeme_ucb}
\end{algorithm}

Pour les expériences, on s'intéresse à
l'impact de la fonction $T$ utilisée.
On expérimente donc avec $T = T_f$ et $T=T_l$,
tel que décrits plus haut.
En guise de référence, on entraîne aussi un modèle 
en utilisant les cibles binaire générées par l'oracle.
Étant donné le facteur aléatoire présent dans l'entraînement, nous effectuons 
5 entraînements distincts pour chaque $T$ et présentons la moyenne
des résultats obtenus.

\subsection{Résultats}

Les résultats obtenus sont présentés dans la figure \ref{fig:cs_learning_curve} 
et le tableau \ref{tab:ROUGE_UCB}.
Ici, UCB linéaire et fixe font référence respectivement à l'utilisation
à l'utilisation $T_l$ et $T_f$ pour déterminer le nombre de pas de temps
effectués par UCB.

Tout d'abord, la figure \ref{fig:cs_learning_curve} illustre l'évolution 
de la performance sur le jeu de validation des modèles selon 
les cibles utilisées.
On constate d'abord que la performance en validation avec les cibles 
UCB atteint rapidement un plateau à partir duquel elle 
ne varie plus.
Ce plateau n'affecte toutefois pas l'entraînement avec les cibles binaires,
avec lesquelles l'apprentissage continue de varier jusqu'à
éventuellement dépasser légèrement la performance des cibles UCB.

\tikzsetnextfilename{combisum_learning_curve}
\begin{figure}[h!]
    \begin{center}
        \begin{tikzpicture}
            \begin{axis}[title={Performance sur le jeu de validation selon la cible utilisée}, grid style={dashed,gray!50}, axis y line*=left, axis x line*=bottom, every axis plot/.append style={line width=1.5pt, mark size=0pt, font=\huge}, name=plot0, y tick label style={/pgf/number format/fixed zerofill}, ylabel={ROUGE}, xlabel={Nombre de mises à jour (en milliers)}, width=0.95\textwidth, height=0.4\textwidth, smooth, xmin=0.5, legend style={at={(0.9,0.1)},anchor=south east}, legend cell align={left}, xmax=22.5, y tick label style={/pgf/number format/fixed}]
                \addplot[blue, ylabel near ticks, line width=2pt] table[x=t, y=sit_fix, col sep=comma]{bandit_combi/graham_results/cs_learning_curve.csv};
                \addplot[forget plot, name path=upperblue, draw=none] table[x=t, y=sit_fix_+, col sep=comma]{bandit_combi/graham_results/cs_learning_curve.csv};
                \addplot[forget plot, name path=lowerblue, draw=none] table[x=t, y=sit_fix_-, col sep=comma]{bandit_combi/graham_results/cs_learning_curve.csv};
                \addplot[red, ylabel near ticks, line width=2pt] table[x=t, y=sit_linear, col sep=comma]{bandit_combi/graham_results/cs_learning_curve.csv};
                \addplot[forget plot, name path=upperred, draw=none] table[x=t, y=sit_linear_+, col sep=comma]{bandit_combi/graham_results/cs_learning_curve.csv};
                \addplot[forget plot, name path=lowerred, draw=none] table[x=t, y=sit_linear_-, col sep=comma]{bandit_combi/graham_results/cs_learning_curve.csv};
                \addplot[green, ylabel near ticks, line width=2pt] table[x=t, y=binary_linear, col sep=comma]{bandit_combi/graham_results/cs_learning_curve.csv};
                \addplot[forget plot, name path=uppergreen, draw=none] table[x=t, y=binary_linear_+, col sep=comma]{bandit_combi/graham_results/cs_learning_curve.csv};
                \addplot[forget plot, name path=lowergreen, draw=none] table[x=t, y=binary_linear_-, col sep=comma]{bandit_combi/graham_results/cs_learning_curve.csv};
                \addplot[fill=blue!20, fill opacity=0.5] fill between[of=upperblue and lowerblue];
                \addplot[fill=red!20, fill opacity=0.5] fill between[of=upperred and lowerred];
                \addplot[fill=green!20, fill opacity=0.5] fill between[of=uppergreen and lowergreen];
                \legend{UCB fixe, UCB linéaire, Oracle}
            \end{axis}
        \end{tikzpicture}
    \end{center}
    \caption{Courbe d'apprentissage de CombiSum selon la cible UCB utilisée.
             Un intervalle de confiance à 95 \% calculé à partir de 5 entraînements distincts
             est rapporté pour chaque cible.}
    \label{fig:cs_learning_curve}
\end{figure}

Le tableau \ref{tab:ROUGE_UCB} présente la performance 
obtenue par nos modèles et par la référence Lead-3 sur le jeu de test.
Pour chaque entraînement effectué, la performance en test 
rapportée est celle obtenue avec les paramètres $\theta$
permettant d'obtenir la meilleure performance en validation.

\begin{table}[!h]
    \centering
    \def\arraystretch{1.8}
    \resizebox{0.8\textwidth}{!}{%
    \begin{tabular}{ccccc}
    \specialrule{.2em}{.1em}{.1em}
    \multicolumn{1}{c}{\textbf{Cible utilisée}} & \multicolumn{1}{c}{\textbf{ROUGE-1}} & \multicolumn{1}{c}{\textbf{ROUGE-2}} & \multicolumn{1}{c}{\textbf{ROUGE-L}} & \multicolumn{1}{c}{\textbf{ROUGE}} \\ \specialrule{.2em}{.1em}{.1em}
    Lead-3 & 39.59  & 17.68 & 36.21 & 31.16\\ \specialrule{.1em}{.05em}{.05em}
    UCB fixe & $40.04 \pm 0.11$  & $\mathbf{18.18 \pm 0.14}$  & $36.46 \pm 0.15$ & $\mathbf{31.56 \pm 0.13}$\\
    UCB linéaire & $40.07 \pm 0.09$  & $\mathbf{18.21 \pm 0.10}$ & $36.49 \pm 0.10$ & $\mathbf{31.59 \pm 0.10}$ \\ 
    Oracle & $\mathbf{40.47 \pm 0.27}$ & $\mathbf{18.21 \pm 0.15}$ & $\mathbf{36.93 \pm 0.24}$ & $\mathbf{31.87 \pm 0.22}$ \\\specialrule{.2em}{.1em}{.1em}
    \end{tabular}
    }
    \caption{Performance sur le jeu de test.}
    \label{tab:ROUGE_UCB}
\end{table}

Un premier constat qui se pose est que, peu importe la cible 
utilisée, les modèles produits se comportent de manière 
extrêmement similaire sur le jeu de test.
Aussi, la performance obtenue n'est que 
très légèrement supérieure à Lead-3.

En somme, les résultats obtenus par CombiSum sont décevants,
ne se distinguant que très légèrement de la référence Lead-3.
Nous élaborons davantage sur les potentielles 
justifications de cette performance et discutons 
d'une piste de solution envisageable au chapitre \ref{chap:analyse_convergemce}.

\section{Conclusion}

Dans ce chapitre, nous avons proposé de nouvelles cibles plus riches
pour l'entraînement d'un modèle de génération de résumés.
Nos nouvelles cibles sont basées sur une version 
de UCB adaptée au bandit combinatoire, dont nous avons validé 
l'applicabilité sur notre jeu de développement.
Enfin, nous avons présenté comment les cibles 
peuvent être utilisées dans un algorithme 
que nous avons nommé CombiSum.

Le prochain chapitre présente une autre 
formulation bandit pouvant être utilisée pour 
générer des cibles riches: le bandit combinatoire 
linéaire.
En mode linéaire, le bandit combinatoire peut être résolu 
encore plus rapidement et le prochain chapitre 
représente donc une version potentiellement plus 
efficace des cibles présentées dans ce chapitre.
\chapter{Bandit combinatoire linéaire}
\label{chap:bandit_combi_lin}                   % étiquette pour renvois (à compléter!)

L'approche combinatoire pure présentée au chapitre précédent présente un inconvénient
notoire: elle n'utilise pas de relation de similarité entre les phrases pour propager
de l'information entre deux phrases similaires.
Il est possible de faire cela sous la formulation du bandit combinatoire linéaire,
où on possède des représentations vectorielles de chacune des actions et on
fait l'hypothèse qu'il existe un vecteur propre au problème qui relie
les représentations de phrases à leur récompense espérée.

Dans ce chapitre, on présente la formulation en bandit combinatoire linéaire et 
comment elle peut s'appliquer au processus de génération du résumé d'un document.
On aborde l'algorithme LinUCB \citep{chu2011contextual} qui minimise 
le regret sur le bandit combinatoire linéaire et on suggère comment il peut
aussi être utilisé pour générer des cibles pour un système de génération de résumés.
On valide encore une fois l'applicabilité des cibles proposées à partir 
d'un jeu de données de développement.
Enfin, on présente comment ces cibles peuvent être intégrées dans un système complet de
génération de résumés, que l'on compare aux performances de l'état-de-l'art et aux approches par bandits
présentées précédemment.

\section{Formulation combinatoire linéaire}

De manière analogue à comment nous avons présenté la formulation combinatoire,
débutons par considérer le cas multi-bras linéaire avant d'incorporer la portion 
combinatoire.
La formulation en bandit linéaire reprend exactement le même
cadre formel que la bandit mutli-bras stochastique vu à la section \ref{sec:bandits}.
La seule nouvelle notion est la présence de représentations vectorielles $\tilde{a}_i \in \mathbb{R}^n$
pour les actions du problème et l'introduction de l'hypothèse linéaire.
Cette dernière suppose que, pour les vecteurs $\tilde{a}_i$, tous soumis à $\lVert \tilde{a}_i \rVert \leq 1$,
il existe un vecteur unique $\omega_*$ avec $\lVert \omega_* \rVert \leq 1$ de récompense permettant 
de relier chaque à sa récompense espérée $\mu_i$, i.e. 
$\langle \omega_*, \tilde{a}_i\rangle \approx \mu_i$ pour tout $i$.
Lorsque l'on se place dans la version combinatoire du bandit linéaire,
l'hypothèse linéaire est alors simplement appliquée à des représentations vectorielles 
des \textit{super-bras} $\tilde{\mathcal{M}}$.


\section{Linear Upper Confidence Bound (LinUCB)}

L'algorithme LinUCB \citep{chu2011contextual} représente une application 
du principe de l'optimisme face à l'incertitude sur le problème de bandit linéaire
qui vise encore une fois la minimisation du regret cumulatif.
Rappelons que ce principe consiste à avoir, pour chaque action $i$, un estimé optimiste UCB$_i$
de sa valeur espérée $\mu_i$, lequel est plus élevé que $\mu_i$ avec forte probabilité.
Dans le contexte linéaire, comme on souhaite faire usage des représentations vectorielles 
des actions, on cherche à avoir UCB$_i$ qui fait usage des relations linéaires entre les 
actions.
Nous présentons plus bas l'intuition derrière la construction de bonnes bornes supérieures
UCB$_i$ en omettant certains détails techniques pour alléger la lecture.
Le lecteur plus curieux pourra se référer à \citep{chu2011contextual} ou \citep{abbasi2011improved}
pour des descriptions techniques complètes.

Une première intuition clé pour bâtir une bonne borne supérieure est celle d'exploiter 
les observations faites aux rondes précédentes pour bâtir un estimé $\hat{\omega}_N$ de 
$\omega_*$.
Cet estimé devrait naturellement viser 
à combiner aussi bien que possible les paires bras-récompenses $(\tilde{a}_{i_t}, X_T)$
observées aux rondes précédentes, i.e.
\begin{equation}
    \langle \hat{\omega}_N, \tilde{a}_{i_t}\rangle \approx X_t.    
    \label{eq:omega_n_assumption}
\end{equation}

L'équation \eqref{eq:omega_n_assumption} peut être vue comme un problème 
de moindres carrés, où l'on souhaite trouver
\begin{equation}
    \hat{\omega}_N = \argmin_{\omega \in \mathbb{R}^n} \sum_{t=1}^{N-1} \left( \langle \omega, \tilde{a}_{i_t} \rangle - X_t \right)^2.
    \label{eq:lstsq}
\end{equation}

Or, le problème \eqref{eq:lstsq} n'admet pas nécessairement de solution exacte ou unique.
LinUCB emploie donc la régularisation de Tikhonov \citep{tikhonov1963solution} 
avec $\lambda=1$ pour garantir une solution et son unicité.
Le centre $\hat{\omega}_N$ est alors choisi selon la solution analytique connue du
problème

\begin{equation}
    \hat{\omega}_N = \mathbf{V_N}^{-1}\mathbf{b_N}, \quad \text{pour} \quad \mathbf{V_N} = (\mathbf{A^\top_N} \mathbf{A_N} + \lambda \mathbf{I}) \quad \text{et} \quad \mathbf{b_N}=\mathbf{A_N^\top} \mathbf{X_N},
    \label{eq:omega_N}
\end{equation}

où $\mathbf{A_N}$ est la matrice dont les lignes sont les actions $\tilde{a}_{i_t}$ sélectionnées,
$\mathbf{X_N}$ est le vecteur composé des récompenses $X_t$ et $\mathbf{I}$ est la matrice identité.

En pratique, $\mathbf{V_N}$ et $\mathbf{b_N}$ peuvent tous deux être calculés de manière 
itérative selon 

\begin{equation}
\mathbf{V_N} = \lambda \mathbf{I} + \displaystyle \sum_{t=1}^N \tilde{a}_{i_t} \tilde{a}_{i_t}^\top, \qquad \mathbf{b_N} = \displaystyle \sum_{t=1}^N \tilde{a}_{i_t} X_t.
\label{eq:linucb_iteratif}
\end{equation}

Maintenant que l'on possède une bonne estimation $\hat{\omega}_N$ qui exploite l'expérience 
des rondes précédentes, on cherche à introduire un terme garantissant une exploration 
satisfaisante des actions possibles.
LinUCB propose à cet effet d'utiliser 
\begin{equation*}
    \sqrt{\beta\tilde{a}_i^\top \mathbf{V_N}^{-1} \tilde{a}_i},
    \label{eq:explo_linucb}
\end{equation*}
un terme qui base l'exploration liée à une action $i$ à sa différence par rapport 
aux actions précédentes représentées par $\mathcal{A_N}$ et un hyperparamètre 
d'exploration $\beta$.
Les détails techniques complets permettant d'arriver au terme pour l'exploration sont 
donnés dans \citep{abbasi2011improved}.

En regroupant les termes d'exploitation et d'exploration, on obtient encore une fois 
une borne supérieure résolvant de manière élégante le compromis exploration-exploitation: 

\begin{equation}
    \text{UCB}_i = \langle \omega_N^\top, a_i \rangle +  \sqrt{\beta\tilde{a}_i^\top \mathbf{V_N}^{-1} \tilde{a}_i},
    \label{eq:linucb}
\end{equation}

où $\beta$ représente encore un hyperparamètre balançant l'exploration.

\subsubsection*{Connexions avec UCB}

Nous prenons un moment ici pour mettre l'emphase sur les connexions entre UCB et 
LinUCB, qui peut être vu comme une généralisation linéaire directe de UCB.
En effet, si l'on considère les vecteurs d'action correspondant aux vecteurs unitaires,
i.e. $\tilde{a}_i = e_i$, et une régularisation de Tikhonov avec $\lambda = 0$, on 
a que $\mathbf{V_N}$ est une matrice diagonale, où la $i$-ème entrée est le nombre 
de fois où l'action $i$ a été sélectionnée.
À ce moment, le terme $\tilde{a}_i^\top \mathbf{V_N}^{-1} \tilde{a}_i$ n'est 
donc que l'inverse du nombre $n_i$ de fois que l'action $i$ a été sélectionnée et le 
terme d'exploration devient
\begin{equation*}
\sqrt{\frac{\beta}{n_i(N)}},
\end{equation*}
un terme proportionnel à celui présent dans UCB si l'on pose $\beta = 2\log(N)$.

Aussi, $\hat{\omega}_N$ devient simplement le vecteur où l'entrée $i$ correspond 
à la moyenne des récompenses perçues pour l'action $i$.
Le produit vectoriel $\langle \hat{\omega}_N, e_i \rangle$ ne fait donc qu'aller
chercher la moyenne $\bar{x}_i$ utilisée dans UCB.

Ainsi, si l'on se positionne dans le cas où aucune information n'est partagée par 
les vecteurs d'action $\tilde{a}_i$, LinUCB est équivalent à UCB.
On peut donc considérer que, si l'hypothèse linéaire est respectée, LinUCB 
obtiendra toujours une performance supérieure ou égale à UCB.

\subsection{Représentations vectorielles de phrases}

Les représentations vectorielles utilisées pour les phrases sont basées sur l'observation
que les calculs de ROUGE sont essentiellement basés sur les comptes de \ngrams.
On choisit donc de générer, pour chaque phrase, un vecteur parcimonieux
représentant le nombre de fois qu'un $n$-gramme donné y est présent.
Pour que les relations entre les phrases soient incorporées dans ces vecteurs,
on associe à chaque $n$-gramme du document un index unique $i$.
On obtient donc, pour chaque phrase, un vecteur parcimonieux de taille $N$,
pour le nombre $N$ de \ngrams dans le document, où l'indice $i$ correspond 
au nombre d'occurrences du $n$-gramme $i$ dans la phrase.
Dans nos expériences, on limite l'exploration des \ngrams aux unigrammes,
afin de garder une taille raisonnable pour les vecteurs parcimonieux générés.

On bâtit ensuite une matrice parcimonieuse $M$ dont les rangées sont les vecteurs parcimonieux 
de chaque phrase du document.
La dimension de la matrice $M$ est ensuite réduite via une analyse sémantique latente \citep{10.1145/291128.291131}, 
applicable naturellement dans notre cas de matrice parcimonieuse de grande taille.
Le processus retourne $|d|$ vecteurs de taille $|d|$, que l'on normalise pour avoir 
une norme unitaire et qu'on utilise pour les 
représentations vectorielles $\tilde{a}_i$.
On note ici que, si aucun $n$-gramme n'était partagé entre les phrases,
leur représentation vectorielle $\tilde{a}_i$ serait alors simplement 
$e_i$ et, comme mentionné plus haut, LinUCB serait équivalent à UCB.

Comme pour les scores $R$, les vecteurs $\tilde{a}_i$ associés à un document peuvent 
être pré-calculés pour ne pas avoir à les recalculer inutilement à chaque fois 
que l'on traite un document.

\subsection{Application à la génération de résumés}

Pour appliquer LinUCB au contexte combinatoire, il suffit de disposer 
de représentations vectorielles $\tilde{\mathcal{M}}$ des super-bras.
Dans le contexte de la génération de résumés, on se base encore une fois 
sur la contribution égale de chacune des phrases d'un résumé pour obtenir 
les représentations vectorielles des résumés.
Pour ce faire, on prend la représentation vectorielle 
de chacune des phrases et que la représentation d'un résumé 
de 3 phrases est alors représentée par la moyenne des phrases
qu'il contient

\begin{equation}
    \tilde{\mathcal{M}} = \frac{1}{3}\sum_{i \in \mathcal{M}} \tilde{a}_i.
\end{equation}

Ainsi, à chaque ronde $t$, la version combinatoire de LinUCB sélectionnera 

\begin{equation*}
\tilde{\mathcal{M}}_t = \text{3-}\argmax_{i \in \mathcal{A}} \text{UCB}_i(t),
\end{equation*}
pour UCB$_i$ tel que défini à l'équation \eqref{eq:linucb}.

L'objectif est encore une fois d'utiliser LinUCB pour générer des cibles.
Dans notre cas, comme LinUCB ne garantit plus l'exploration de chacune des 
actions, on utilise alors les approximations fournies par $\langle \omega_N, \tilde{a}_i \rangle$
comme cibles au lieu de $\hat{x}_i$.
On mesure la qualité de la cible générée par LinUCB par le score $R_t=R(\mathcal{M}, s)$ 
du meilleur résumé $\mathcal{M}$ pris en sélectionnant les 3 actions dont la valeur 
$\langle \omega_t, \tilde{a}_i \rangle$ est maximale.

\subsection{Introduction de connaissances a priori}

On note aussi que, comme pour le bandit combinatoire pur, des connaissance a priori $P$
peuvent être insérées dans le calcul de borne supérieure de LinUCB, qui devient alors 

\begin{equation}
    \text{UCB}_i = \langle \omega_N^\top, a_i \rangle +  P_i\sqrt{\beta\tilde{a}_i^\top \mathbf{V_N}^{-1} \tilde{a}_i}.
    \label{eq:linucb_prior}
\end{equation}

\subsection{Expériences}

Similairement au chapitre sur la formulation combinatoire pure,
on mesure la sous-optimalité $\Delta_t = R^* - R_t$ des cibles générées
par LinUCB.
On reprend donc le jeu de développement de 25 000 documents pour mener des expériences 
sur la qualité des cibles générées par LinUCB.

Les figures \ref{fig:bandit_combi_lin_alpha} et \ref{fig:bandit_combi_lin_doc_len} sont obtenues 
avec un prior uniforme alors que les figures \ref{fig:bandit_combi_lin_prior_tau} 
et \ref{fig:bandit_combi_lin_prior_choice} expérimentent avec les mêmes 
configurations de priors décrite à la section \ref{sec:experiences_ucb_priors}.
Le pseudocode permettant de recréer les expériences se trouve à l'algorithme 
\ref{alg:cible_linucb}.

Notre implémentation actuelle diffère légèrement du pseudocode présenté 
en utilisant la nature de la matrice $\mathbf{V}$ pour éviter d'avoir 
à calculer son inverse à chaque ronde et être computationnellement plus efficace.
Les détails techniques de cette manipulation sont disponibles dans l'annexe 
\ref{chap:lincub_gains} pour le lecteur intéressé.

\begin{algorithm}
    \caption{LinUCB combinatoire pour génération de résumé}
    \begin{algorithmic}[1]
        \Require $d$ (document), $s$ (résumé cible), $\beta$ (paramètre d'exploration), $T$ (nombre de rondes de UCB), $\tilde{a}_i$ (représentations des phrases).
        \State {$\mathbf{V} = \mathbf{I}$}
        \State {$\mathbf{b} = \mathbf{0}$}
        \For{$t=1, ..., T$}
        \State $\omega_t = \mathbf{V}^{-1}\mathbf{b}$
        \For{$i=1,..., |d|$}
        \State UCB$_i = \langle \omega_t^\top, \tilde{a}_i \rangle + \sqrt{\beta\tilde{a}_i^\top \mathbf{V}^{-1} \tilde{a}_i}$
        \EndFor
        \State $\mathcal{M}_t =$ 3-$\argmax$ UCB$_i$
        \State $X_t = R(\mathcal{M}_t, s)$
        \ForAll{$i \in \mathcal{M}_t$}
        \State {$\mathbf{V} = \mathbf{V} + \tilde{a}_i \tilde{a}_i^\top$}
        \State {$\mathbf{b} = \mathbf{b} + \tilde{a}_i X_t $}
        \EndFor
        \EndFor
    \end{algorithmic}
    \label{alg:cible_linucb}
\end{algorithm}


\subsection{Résultats}

La figure \ref{fig:bandit_combi_lin_alpha} présente comment l'hyperparamètre $\beta$ influence la
convergence de LinUCB.
On remarque d'abord que, comme il fallait s'y attendre, LinUCB converge permet d'obtenir 
nettement plus rapidement des cibles de bonne qualité de UCB.
En effet, l'erreur $\Delta_t$ de près 0.05 observée avec UCB après 250 rondes est plutôt 
observable après 50 rondes de LinUCB.
On remarque aussi $\beta=10^8$ (courbe bleue sur la figure) semble être un bon choix,
réussissant à converger légèrement plus rapidement que $\beta=10^7$.

\commentaire{Honnêtement ici j'ai essayé des valeurs de $\beta$ entre $10^5 et 10^10$ et le
constat est que la convergence est exécrable en bas de $10^7$ et similaire après.
C'est difficle de rapporter plus de valeurs de $\beta$ car elles sont soit aberrantes soit 
elles se confondent les unes avec les autres.}

Il est à noter que les différences rapportées entre les différentes 
courbes sur les figures ne sont pas statistiquement significatives.
Pour éviter d'encombrer inutilement les figures, on rapporte alors 
seulement les moyennes observées.
Les versions incorporant la déviation standard des différentes courbes 
se trouvent à l'annexe \ref{chap:variance_graphs}.

\tikzsetnextfilename{bandit_combi_lin_alpha}
\begin{figure}[h!]
    \begin{center}
        \begin{tikzpicture}
            \begin{axis}[title={Impact de l'exploration}, grid style={dashed,gray!50}, axis y line*=left, axis x line*=bottom, every axis plot/.append style={line width=1.5pt, mark size=0pt, font=\huge}, name=plot0, y tick label style={/pgf/number format/fixed zerofill}, ylabel={$\Delta_t$}, xlabel={$t$}, width=0.95\textwidth, height=0.4\textwidth, smooth, xmin=0, xmax=100, y tick label style={/pgf/number format/fixed}, ytick={0.00, 0.05, 0.10, 0.15, 0.20, 0.25}]
                \addplot[green, ylabel near ticks, line width=3pt] table[x=t, y=10000000.0_, col sep=comma]{bandit_combi_lin/bandit_combi_linExpResults/alpha/all.csv};
                \addplot[blue, ylabel near ticks, line width=3pt] table[x=t, y=100000000.0_, col sep=comma]{bandit_combi_lin/bandit_combi_linExpResults/alpha/all.csv};
                \addplot[black, ylabel near ticks, line width=3pt] table[x=t, y=q_1e1, col sep=comma]{bandit_combi/bandit_combiExpResults/alpha/all.csv};
                \legend{$\beta = 10^7$, $\beta = 10^8$, UCB}
            \end{axis}
        \end{tikzpicture}
    \end{center}
    \caption{Impact du paramètre d'exploration $\beta$ utilisé par LinUCB sur la convergence.}
    \label{fig:bandit_combi_lin_alpha}
\end{figure}

La figure \ref{fig:bandit_combi_lin_doc_len} évalue dans quelle mesure la taille des documents
influe sur la sous-optimalité des cibles générées par UCB.
La valeur $\beta=10^8$ (courbe bleue) semble encore une fois optimale et ce, pour toutes les tailles 
de document.
Similairement à ce qui avait été observé pour les cibles générées par UCB, il semble que le 
rouler LinUCB avec plus de rondes est bénéfique pour les documents plus longs.
Notamment, il semble qu'un minimum de 20 rondes soit raisonnable et que les documents plus longs
pourraient bénéficier d'aller jusqu'à 100 rondes.

\begin{figure}[h!]
    \tikzsetnextfilename{bandit_combi_lin_less20}
    \begin{tikzpicture}[baseline]
        \begin{axis}[grid style={dashed,gray!50}, axis y line*=left, axis x line*=bottom, every axis plot/.append style={line width=1.5pt, mark size=0pt, font=\Large}, width=0.34\textwidth,
                height=0.4\textwidth, name=plot0, y tick label style={/pgf/number format/fixed zerofill}, xmin=0.0, xmax=100.0, ymin=0.01, ymax=0.30, ylabel={$\Delta_t$}, xlabel={$t$}, smooth, title={$|d| \leq 20$}]
            \addplot[red, ylabel near ticks, line width=3pt] table[x=t, y=less20_10000000.0_, col sep=comma]{bandit_combi_lin/bandit_combi_linExpResults/doc_len/all.csv};
            \addplot[blue, ylabel near ticks, line width=3pt] table[x=t, y=less20_100000000.0_, col sep=comma]{bandit_combi_lin/bandit_combi_linExpResults/doc_len/all.csv};
            \legend{$\beta = 10^7$, $\beta = 10^8$}
        \end{axis}
    \end{tikzpicture}
    \tikzsetnextfilename{bandit_combi_lin_20to35}
    \begin{tikzpicture}[baseline]
        \begin{axis}[grid style={dashed,gray!50}, axis y line*=left, axis x line*=bottom, every axis plot/.append style={line width=1.5pt, mark size=0pt, font=\huge}, width=.34\textwidth,
                height=0.4\textwidth, name=plot0, xshift=-.1\textwidth, y tick label style={/pgf/number format/fixed zerofill},xmin=0.0, xmax=100.0, ymin=0.01, ymax=0.30, xlabel={$t$}, legend style={at={(0.9,0.1)},anchor=south east}, smooth, title={$20 < |d| < 35$}, ymajorticks=false]
            \addplot[red, ylabel near ticks, line width=3pt] table[x=t, y=20to35_10000000.0_, col sep=comma]{bandit_combi_lin/bandit_combi_linExpResults/doc_len/all.csv};
            \addplot[blue, ylabel near ticks, line width=3pt] table[x=t, y=20to35_100000000.0_, col sep=comma]{bandit_combi_lin/bandit_combi_linExpResults/doc_len/all.csv};
        \end{axis}
    \end{tikzpicture}
    \tikzsetnextfilename{bandit_combi_lin_more35}
    \begin{tikzpicture}[baseline]
        \begin{axis}[grid style={dashed,gray!50}, axis y line*=right, axis x line*=bottom, every axis plot/.append style={line width=1.5pt, mark size=0pt, font=\huge}, width=.34\textwidth,
                height=0.4\textwidth, name=plot0, xshift=-.1\textwidth, y tick label style={/pgf/number format/fixed zerofill}, xmin=0.0, xmax=100.0, ymin=0.01, ymax=0.30, xlabel={$t$}, legend style={at={(1,0.1)},anchor=south east}, smooth, title={$35 \leq |d|$}]
            \addplot[red, ylabel near ticks, line width=3pt] table[x=t, y=more35_10000000.0_, col sep=comma]{bandit_combi_lin/bandit_combi_linExpResults/doc_len/all.csv};
            \addplot[blue, ylabel near ticks, line width=3pt] table[x=t, y=more35_100000000.0_, col sep=comma]{bandit_combi_lin/bandit_combi_linExpResults/doc_len/all.csv};
        \end{axis}
    \end{tikzpicture}
    \caption{Impact de la taille du document sur la convergence.}
    \label{fig:bandit_combi_lin_doc_len}
\end{figure}

La figure \ref{fig:bandit_combi_lin_prior_tau} présente comment la similarité 
entre le prior utilisé et la distribution uniforme sur les phrases 
(paramétrée par $\tau$) impacte la qualité des cibles générées.
Les résultats démontrent que l'introduction de priors ne permet absolument pas de générer 
de meilleures cibles, car le prior uniforme (ligne noire) obtient des résultats
considérablement meilleurs.

Il est difficile de s'expliquer comment l'introduction de priors pouvait être si 
bénéfique à la convergence de UCB mais ne fait que nuire à LinUCB.
On émet l'hypothèse que cette différence est dûe à la nature de l'exploitation
dans LinUCB.
En effet, comme LinUCB bâtit un estimé $\hat{\omega}_N$ utilisé pour estimer les valeurs 
de toutes les actions à partir des rondes précédentes, le fait d'introduire un prior 
peut faire en sorte que $\hat{\omega}_N$ produit un piètre estimé pour les phrases
avec un prior faible.
Or, comme un prior faible leur est associé, le terme d'exploration ne permet pas de les 
visiter suffisamment souvent pour générer des cibles de meilleure qualité.

\tikzsetnextfilename{bandit_combi_lin_prior_tau}
\begin{figure}[h!]
    \begin{center}
        \begin{tikzpicture}
            \begin{axis}[title={Impact de $\tau$ $(\beta = 10^8$, résumé médian)}, grid style={dashed,gray!50}, axis y line*=left, axis x line*=bottom, every axis plot/.append style={line width=1.5pt, mark size=0pt, font=\huge}, name=plot0, y tick label style={/pgf/number format/fixed zerofill}, ylabel={$\Delta_t$}, xlabel={$t$}, width=0.95\textwidth, height=0.4\textwidth, smooth, xmin=1, xmax=100, ymin=0.0, y tick label style={/pgf/number format/fixed}]
                \addplot[black, ylabel near ticks, line width=3pt] table[x=t, y=best_0.0_, col sep=comma]{bandit_combi_lin/bandit_combi_linExpResults/priors/all.csv};
                \addplot[red, ylabel near ticks, line width=3pt] table[x=t, y=med_0.1_, col sep=comma]{bandit_combi_lin/bandit_combi_linExpResults/priors/all.csv};
                \addplot[blue, ylabel near ticks, line width=3pt] table[x=t, y=med_0.2_, col sep=comma]{bandit_combi_lin/bandit_combi_linExpResults/priors/all.csv};
                \legend{$\tau=0.0$, $\tau=0.1$, $\tau=0.2$}
            \end{axis}
        \end{tikzpicture}
    \end{center}
    \caption{Impact de la similarité de distributions a priori avec une distribution uniforme, selon le nombre $t$ de rondes effectuées.}
    \label{fig:bandit_combi_lin_prior_tau}
\end{figure}

La figure \ref{fig:bandit_combi_prior_choice} présente l'impact de la qualité du prior sur l'évolution
de $\Delta_t$.
Sans grande surprise, les priors basés sur le pire résumé et le résumé médian génèrent 
de piètre cibles alors que les priors basés sur le meilleur résumé convergent extrêmement 
rapidement.

\tikzsetnextfilename{bandit_combi_lin_prior_choice}
\begin{figure}[h!]
    \begin{center}
        \begin{tikzpicture}
            \begin{axis}[title={Impact de la qualité de la distribution \textit{a priori} $(\beta = 10^8, \tau=0.1)$}, grid style={dashed,gray!50}, axis y line*=left, axis x line*=bottom, every axis plot/.append style={line width=1.5pt, mark size=0pt, font=\huge}, name=plot0, y tick label style={/pgf/number format/fixed zerofill}, ylabel={$\Delta_t$}, xlabel={$t$}, width=0.95\textwidth, height=0.4\textwidth, smooth, xmin=1, xmax=100, ymin=0.0, y tick label style={/pgf/number format/fixed}]
                \addplot[black, ylabel near ticks, line width=3pt] table[x=t, y=best_0.0_, col sep=comma]{bandit_combi_lin/bandit_combi_linExpResults/priors/all.csv};
                \addplot[red, ylabel near ticks, line width=3pt] table[x=t, y=best_0.1_, col sep=comma]{bandit_combi_lin/bandit_combi_linExpResults/priors/all.csv};
                \addplot[blue, ylabel near ticks, line width=3pt] table[x=t, y=med_0.1_, col sep=comma]{bandit_combi_lin/bandit_combi_linExpResults/priors/all.csv};
                \addplot[green, ylabel near ticks, line width=3pt] table[x=t, y=worst_0.1_, col sep=comma]{bandit_combi_lin/bandit_combi_linExpResults/priors/all.csv};
                \legend{Uniforme, Meilleur, Médian, Pire}
            \end{axis}
        \end{tikzpicture}
    \end{center}
    \caption{Impact de la similarité de distributions à priori avec une distribution uniforme, selon le nombre $t$ de rondes effectuées.}
    \label{fig:bandit_combi_lin_prior_choice}
\end{figure}



\section{LinCombiSum}
\label{section:lincombisum}

On propose maintenant un système de génération de résumés complet nommé LinCombiSum,
qui se veut identique à CombiSum (\ref{sec:combisum}) mais qui utilise LinUCB
au lieu de UCB pour générer ses cibles.
Aussi, comme les expériences ont démontré que l'introduction de connaissances
a priori ne permet pas d'améliorer les cibles générées par LinUCB, on considère 
seulement les cibles avec prior uniforme.
Pour LinCombiSum, les cibles supervisées utilisées sont donc les valeurs
$\langle \hat{\omega}, \tilde{a}_i \rangle$ retournées par LinUCB, 
que l'on peut naturellement apprendre via une fonction de perte quadratique.
À l'inférence, le résumé est encore une fois généré en regroupant les 3 phrases pour lesquelles 
la prédiction de LinCombiSum est maximale.

\subsection{Expériences}

On roule le système sur le jeu de données CNN/DailyMail.
Le pseudocode décrivant le système se trouve dans l'algorithme \ref{alg:systeme_ucb}.

\begin{algorithm}
    \caption{LinCombiSum}
    \begin{algorithmic}[1]
        \Require  $\mathcal{D}$ (jeu de données), $h_\theta$ (modèle neuronal), $T$ (nombre de rondes UCB), $\alpha$ (taux d'apprentissage), $\beta$ (taux d'exploration UCB), $B$ (taille de minibatch).
        \While{vrai}
        \State{batch $\sim \mathcal{D}^B$} \Comment{On pige la minibatch du jeu de données}
        \State $\nabla = \mathbf{0}$
        \ForAll{$(d,s) \in $ batch}
        \State $x = h_\theta(d)$
        \State Obtenir les représentations $\tilde{a}_i$ des phrases de $d$
        \State Générer les cibles $\hat{x}_i = \langle \hat{\omega}_T, \tilde{a}_i \rangle$ après $T$ rondes de LinUCB
        \State $\nabla = \nabla + \nabla_\theta \lVert x - \hat{x} \rVert^2$
        \EndFor
        \State $\theta = \theta - \alpha \nabla$
        \EndWhile
    \end{algorithmic}
    \label{alg:systeme_ucb}
\end{algorithm}

Pour les expériences, on teste de faire varier le nombre de rondes de LinUCB utilisées en fonction 
du nombre de phrases ou de le fixer à $T=50$ selon les expériences de LinUCB.

\subsection{Résultats}

\todo{Idem à CombiSum}

\section{Conclusion}

\todo{Idem à CombiSum}

\chapter*{Conclusion}           % ne pas numéroter
\label{chap:conclusion}         % étiquette pour renvois
\phantomsection\addcontentsline{toc}{chapter}{\nameref{chap:conclusion}} % inclure dans TdM

\section*{Synthèse}

\todo{Comparaison des méthodes entre elles et avec l'état de l'art.
    Les angles à aborder sont la perfomance $R$, la vitesse en temps de calcul
    ainsi que la vitesse en nombre de documents.}

\section*{Travaux futurs}

On résout seulement une instance très limitée de bandit combinatoire: celle où la taille est
fixée.
On pourrait étudier comment des formulations plus souples comme \citep{luo-etal-2019-reading}
seraient envisageables avec les formulations par bandit explorées dans les deux derniers chapitres.
Autre point: on n'a utilisé aucun algo de bandit combinatoire dédié, on a juste utilisé
une généralisation des algos de MAB habituels pour les transformer en mode combinatoire
comme on avait une formulation facile.
Ce serait intéressant de voir la différence de performance quand on utilise les algorithmes
dédiés spécifiquement à cette tâche.


\appendix                       % annexes le cas échéant

\chapter{Gains computationnels LinUCB}     % numérotée
\label{chap:lincub_gains}                   % étiquette pour renvois (à compléter!)

Cet annexe vise à décrire comment il est possible d'alléger le coût computationnel
lié à LinUCB en exploitant la nature de la matrice $\mathbf{V}_t$ construite
selon \eqref{eq:linucb_iteratif}.
Commençons d'abord par énoncer la formule de Sherman-Morrison \citep{sherman1950adjustment}:

\textbf{Formule de Sherman-Morrison}
Soit $\mathbf{M} \in \mathbb{R}^{n \times n}$ une matrice carrée inversible et $u, v \in \mathbb{R}^n$
deux vecteurs.
Dans ce cas, $\mathbf{M} + uv^\top$ est inversible si et seulement si $1 + v^\top \mathbf{M}u \neq 0$.
On a alors
\begin{equation}
    \left(\mathbf{M} + uv^\top \right)^{-1} = \mathbf{M}^{-1} - \frac{\mathbf{M}^{-1}uv^\top\mathbf{M}^{-1}}{1 + v^\top \mathbf{M}u}.
    \label{eq:sherman_morrison}
\end{equation}

Pour le calcul de $\mathbf{V}_t$ dans LinUCB, la formule s'applique directement et
donne 
\begin{equation}
\mathbf{V}_t^{-1} =
\left(\mathbf{V}_{t-1} + \tilde{a}_{i_t}^\top \tilde{a}_{i_t} \right)^{-1} =
\mathbf{V}_{t-1}^{-1} - \frac{\mathbf{V}_{t-1}^{-1}\tilde{a}_{i_t}\tilde{a}_{i_t}^\top\mathbf{V}_{t-1}^{-1}}{1 + \tilde{a}_{i_t}^\top \mathbf{V}_{t-1}\tilde{a}_{i_t}},
\end{equation}
un calcul bien plus efficace que d'inverser la matrice $\mathbf{V}_t$ car il ne requiert 
que des produits matriciels et vectoriels.

On note aussi qu'une légère optimisation peut aussi être faite en constatant que 
le produit $\mathbf{V}^{-1}_{t-1}\tilde{a}_{i_t}$ est utilisé à plusieurs reprises.
On peut donc entreposer le produit en posant 
\begin{equation*}
W = \mathbf{V}^{-1}_{t-1}\tilde{a}_{i_t}.
\end{equation*}

Enfin, comme la matrice $\mathbf{V}$ sera toujours symétrique car elle est la 
somme de matrices symétriques, on aura aussi $\mathbf{V}^{-1}$ symétrique.
Cette symétrie peut aussi être exploitée pour donner 
\begin{equation}
    \mathbf{V}_t^{-1} =
    \mathbf{V}_{t-1}^{-1} - \frac{WW^\top}{1 + \tilde{a}_{i_t}^\top W},
\end{equation}
qui est la version utilisée dans notre implémentation de LinUCB.
\chapter{Graphiques avec variance}     % numérotée
\label{chap:variance_graphs}                   % étiquette pour renvois (à compléter!)

On présente ici, chapitre par chapitre, les versions des graphiques 
incorporant les déviations standard.

\section*{Graphiques du chapitre \ref{chap:bandit_contextuel}}

\tikzsetnextfilename{bandit_contextuel_overall_std}
\begin{figure}[ht!]
    \begin{center}
        \begin{tikzpicture}
            \begin{axis}[grid style={dashed,gray!50}, axis y line*=left, axis x line*=bottom, every axis plot/.append style={line width=1.5pt, mark size=0pt, font=\huge}, name=plot0, xshift=0\textwidth, ylabel={$\Delta_N$}, xlabel={Nombre $N$ d'échantillons}, width=0.95\textwidth, height=0.4\textwidth, smooth, xmin=1.1, xmax=50, ymin=0, ymax=5, y tick label style={/pgf/number format/fixed}, scaled y ticks = false, ytick={1, 2, 3, 4, 5}]
                \addplot[red, ylabel near ticks, line width=1.5pt] table[x expr=\thisrow{t} + 1, y expr=\thisrow{less20_} * 100, col sep=comma]{bandit_contextuel/bandit_contextuelExpResults/doc_len/all.csv};
                \addplot[green, ylabel near ticks, line width=1.5pt] table[x expr=\thisrow{t} + 1, y expr=\thisrow{20to35_} * 100, col sep=comma]{bandit_contextuel/bandit_contextuelExpResults/doc_len/all.csv};
                \addplot[blue, ylabel near ticks, line width=1.5pt] table[x expr=\thisrow{t} + 1, y expr=\thisrow{more35_} * 100, col sep=comma]{bandit_contextuel/bandit_contextuelExpResults/doc_len/all.csv};
                \addplot[forget plot, name path=upperred, draw=none] table[x expr=\thisrow{t} + 1, y expr=\thisrow{less20_+} * 100, col sep=comma]{bandit_contextuel/bandit_contextuelExpResults/doc_len/all.csv};
                \addplot[forget plot, name path=lowerred, draw=none] table[x expr=\thisrow{t} + 1, y expr=\thisrow{less20_-} * 100, col sep=comma]{bandit_contextuel/bandit_contextuelExpResults/doc_len/all.csv};
                \addplot[forget plot, name path=uppergreen, draw=none] table[x expr=\thisrow{t} + 1, y expr=\thisrow{20to35_+} * 100, col sep=comma]{bandit_contextuel/bandit_contextuelExpResults/doc_len/all.csv};
                \addplot[forget plot, name path=lowergreen, draw=none] table[x expr=\thisrow{t} + 1, y expr=\thisrow{20to35_-} * 100, col sep=comma]{bandit_contextuel/bandit_contextuelExpResults/doc_len/all.csv};
                \addplot[forget plot, name path=upperblue, draw=none] table[x expr=\thisrow{t} + 1, y expr=\thisrow{more35_+} * 100, col sep=comma]{bandit_contextuel/bandit_contextuelExpResults/doc_len/all.csv};
                \addplot[forget plot, name path=lowerblue, draw=none] table[x expr=\thisrow{t} + 1, y expr=\thisrow{more35_-} * 100, col sep=comma]{bandit_contextuel/bandit_contextuelExpResults/doc_len/all.csv};
                \addplot[fill=blue!20, fill opacity=0.5] fill between[of=upperblue and lowerblue];
                \addplot[fill=red!20, fill opacity=0.5] fill between[of=upperred and lowerred];
                \addplot[fill=green!20, fill opacity=0.5] fill between[of=uppergreen and lowergreen];
                \legend{$|d| \leq 20$, $20 < |d| < 35$, $35 \leq |d|$}
            \end{axis}
        \end{tikzpicture}
    \end{center}
    \caption[Erreur d'échantillonnage en fonction de la taille de document avec intervalle de confiance de 95 \%]
    {Impact de la taille du document en entrée sur l'erreur d'échantillonnage \eqref{eq:erreur_echantillonnage}
    (plus bas est meilleur).
    La zone pâle représente un intervalle de confiance de 95 \% sur la valeur de chacune des courbes.}
\end{figure}

\tikzsetnextfilename{bandit_contextuel_top3_std}
\begin{figure}[ht!]
    \begin{center}
        \begin{tikzpicture}
            \begin{axis}[legend cell align={left}, grid style={dashed,gray!50}, axis y line*=left, axis x line*=bottom, every axis plot/.append style={line width=1.5pt, mark size=0pt, font=\huge}, name=plot0, xshift=-0\textwidth, y tick label style={
                /pgf/number format/precision=1,
                /pgf/number format/fixed}, ylabel={$\Delta_N$}, xlabel={Probabilité maximale de $\xi(p(d))$}, width=0.95\textwidth, height=0.4\textwidth, smooth, scaled y ticks = false, ymin=0.0, ymax=5.0, ytick={1, 2, 3, 4, 5}, xmin=0.06, xmax=1.01]
                \addplot[red, ylabel near ticks, line width=1.5pt] table[x=t, y expr=\thisrow{t8_} * 100, col sep=comma]{bandit_contextuel/bandit_contextuelExpResults/top3/all.csv};
                \addplot[green, ylabel near ticks, line width=1.5pt] table[x=t, y expr=\thisrow{t16_} * 100, col sep=comma]{bandit_contextuel/bandit_contextuelExpResults/top3/all.csv};
                \addplot[blue, ylabel near ticks, line width=1.5pt] table[x=t, y expr=\thisrow{t32_} * 100, col sep=comma]{bandit_contextuel/bandit_contextuelExpResults/top3/all.csv};
                \addplot[yellow, ylabel near ticks, line width=1.5pt] table[x=t, y expr=\thisrow{t64_} * 100, col sep=comma]{bandit_contextuel/bandit_contextuelExpResults/top3/all.csv};
                \addplot[forget plot, name path=upperred, draw=none] table[x=t, y expr=\thisrow{t8_+} * 100, col sep=comma]{bandit_contextuel/bandit_contextuelExpResults/top3/all.csv};
                \addplot[forget plot, name path=lowerred, draw=none] table[x=t, y expr=\thisrow{t8_-} * 100, col sep=comma]{bandit_contextuel/bandit_contextuelExpResults/top3/all.csv};
                \addplot[forget plot, name path=uppergreen, draw=none] table[x=t, y expr=\thisrow{t16_+} * 100, col sep=comma]{bandit_contextuel/bandit_contextuelExpResults/top3/all.csv};
                \addplot[forget plot, name path=lowergreen, draw=none] table[x=t, y expr=\thisrow{t16_-} * 100, col sep=comma]{bandit_contextuel/bandit_contextuelExpResults/top3/all.csv};
                \addplot[forget plot, name path=upperblue, draw=none] table[x=t, y expr=\thisrow{t32_+} * 100, col sep=comma]{bandit_contextuel/bandit_contextuelExpResults/top3/all.csv};
                \addplot[forget plot, name path=lowerblue, draw=none] table[x=t, y expr=\thisrow{t32_-} * 100, col sep=comma]{bandit_contextuel/bandit_contextuelExpResults/top3/all.csv};
                \addplot[forget plot, name path=upperyellow, draw=none] table[x=t, y expr=\thisrow{t64_+} * 100, col sep=comma]{bandit_contextuel/bandit_contextuelExpResults/top3/all.csv};
                \addplot[forget plot, name path=loweryellow, draw=none] table[x=t, y expr=\thisrow{t64_-} * 100, col sep=comma]{bandit_contextuel/bandit_contextuelExpResults/top3/all.csv};
                \addplot[fill=blue!20, fill opacity=0.5] fill between[of=upperblue and lowerblue];
                \addplot[fill=red!20, fill opacity=0.5] fill between[of=upperred and lowerred];
                \addplot[fill=green!20, fill opacity=0.5] fill between[of=uppergreen and lowergreen];
                \addplot[fill=yellow!20, fill opacity=0.5] fill between[of=upperyellow and loweryellow];
                \legend{$N=8$, $N=16$, $N=32$, $N=64$}
            \end{axis}
        \end{tikzpicture}
    \end{center}
    \caption[Erreur d'échantillonnage selon la plus grande probabilité de génération avec intervalle de confiance de 95 \%]
    {Impact de la probabilité maximale de la distribution des résumés sur l'erreur 
    d'échantillonnage \eqref{eq:erreur_echantillonnage} (plus bas est meilleur).
    La zone pâle représente un intervalle de confiance de 95 \% sur la valeur de chacune des courbes.}
\end{figure}

\clearpage

\section*{Graphiques du chapitre \ref{chap:bandit_combi}}

\tikzsetnextfilename{bandit_combi_alpha_std}
\begin{figure}[ht!]
    \begin{center}
        \begin{tikzpicture}
            \begin{axis}[legend cell align={left},  grid style={dashed,gray!50}, axis y line*=left, axis x line*=bottom, every axis plot/.append style={line width=1.5pt, mark size=0pt, font=\huge}, name=plot0, 
                ylabel={$\Delta_T$}, xlabel={$T$}, width=0.95\textwidth, height=0.4\textwidth, smooth, xmin=1, xmax=250, y tick label style={/pgf/number format/fixed},  
                ytick={5, 10, 15, 20, 25}, ymax=25, ymin=0.0]
                \addplot[red, ylabel near ticks, line width=1.5pt] table[x expr=\thisrow{t} + 1, y expr=\thisrow{q_1e0} * 100, col sep=comma]{bandit_combi/bandit_combiExpResults/alpha/all.csv};
                \addplot[green, ylabel near ticks, line width=1.5pt] table[x expr=\thisrow{t} + 1, y expr=\thisrow{q_1e1} * 100, col sep=comma]{bandit_combi/bandit_combiExpResults/alpha/all.csv};
                \addplot[blue, ylabel near ticks, line width=1.5pt] table[x expr=\thisrow{t} + 1, y expr=\thisrow{q_1e2} * 100, col sep=comma]{bandit_combi/bandit_combiExpResults/alpha/all.csv};
                \addplot[forget plot, name path=upperred, draw=none] table[x expr=\thisrow{t} + 1, y expr=\thisrow{q_1e0+} * 100, col sep=comma]{bandit_combi/bandit_combiExpResults/alpha/all.csv};
                \addplot[forget plot, name path=lowerred, draw=none] table[x expr=\thisrow{t} + 1, y expr=\thisrow{q_1e0-} * 100, col sep=comma]{bandit_combi/bandit_combiExpResults/alpha/all.csv};
                \addplot[forget plot, name path=uppergreen, draw=none] table[x expr=\thisrow{t} + 1, y expr=\thisrow{q_1e1+} * 100, col sep=comma]{bandit_combi/bandit_combiExpResults/alpha/all.csv};
                \addplot[forget plot, name path=lowergreen, draw=none] table[x expr=\thisrow{t} + 1, y expr=\thisrow{q_1e1-} * 100, col sep=comma]{bandit_combi/bandit_combiExpResults/alpha/all.csv};
                \addplot[forget plot, name path=upperblue, draw=none] table[x expr=\thisrow{t} + 1, y expr=\thisrow{q_1e2+} * 100, col sep=comma]{bandit_combi/bandit_combiExpResults/alpha/all.csv};
                \addplot[forget plot, name path=lowerblue, draw=none] table[x expr=\thisrow{t} + 1, y expr=\thisrow{q_1e2-} * 100, col sep=comma]{bandit_combi/bandit_combiExpResults/alpha/all.csv};
                \addplot[fill=blue!20, fill opacity=0.5] fill between[of=upperblue and lowerblue];
                \addplot[fill=red!20, fill opacity=0.5] fill between[of=upperred and lowerred];
                \addplot[fill=green!20, fill opacity=0.5] fill between[of=uppergreen and lowergreen];
                \legend{$\beta = 1$, $\beta = 10$, $\beta = 100$}
            \end{axis}
        \end{tikzpicture}
    \end{center}
    \caption[Impact du paramètre d'exploration $\beta$ sur la convergence de CUCB avec intervalle de confiance de 95 \%]
    {Impact du paramètre d'exploration $\beta$ utilisé par CUCB sur la convergence.
    $\Delta_T$ (plus bas est meilleur) représente la différence entre le score ROUGE du meilleur résumé
    extractif d'un document et celui suggéré CUCB après $T$ pas de temps.
    La zone pâle représente un intervalle de confiance de 95 \% sur la valeur de chacune des courbes.}
\end{figure}

\begin{figure}[ht!]
    \tikzsetnextfilename{bandit_combi_less20_std}
    \begin{tikzpicture}[baseline]
        \begin{axis}[grid style={dashed,gray!50}, axis y line*=left, axis x line*=bottom,legend cell align={left},  every axis plot/.append style={line width=1.5pt, mark size=0pt, font=\Large}, width=0.35\textwidth,
                height=0.4\textwidth, name=plot0, xmin=1, xmax=250.0, ymin=1, ymax=25, ylabel={$\Delta_T$}, xlabel={$T$}, smooth, title={$|d| \leq 20$}, xtick={50, 100, 150, 200, 250}, ytick={5,10,15,20,25}]
            \addplot[red, ylabel near ticks, line width=1.5pt] table[x expr=\thisrow{t} + 1, y expr=\thisrow{less20_1e0} * 100, col sep=comma]{bandit_combi/bandit_combiExpResults/doc_len/all.csv};
            \addplot[green, ylabel near ticks, line width=1.5pt] table[x expr=\thisrow{t} + 1, y expr=\thisrow{less20_1e1} * 100, col sep=comma]{bandit_combi/bandit_combiExpResults/doc_len/all.csv};
            \addplot[blue, ylabel near ticks, line width=1.5pt] table[x expr=\thisrow{t} + 1, y expr=\thisrow{less20_1e2} * 100, col sep=comma]{bandit_combi/bandit_combiExpResults/doc_len/all.csv};
            \addplot[forget plot, name path=upperred, draw=none] table[x expr=\thisrow{t} + 1, y expr=\thisrow{less20_1e0+} * 100, col sep=comma]{bandit_combi/bandit_combiExpResults/doc_len/all.csv};
            \addplot[forget plot, name path=lowerred, draw=none] table[x expr=\thisrow{t} + 1, y expr=\thisrow{less20_1e0-} * 100, col sep=comma]{bandit_combi/bandit_combiExpResults/doc_len/all.csv};
            \addplot[forget plot, name path=uppergreen, draw=none] table[x expr=\thisrow{t} + 1, y expr=\thisrow{less20_1e1+} * 100, col sep=comma]{bandit_combi/bandit_combiExpResults/doc_len/all.csv};
            \addplot[forget plot, name path=lowergreen, draw=none] table[x expr=\thisrow{t} + 1, y expr=\thisrow{less20_1e1-} * 100, col sep=comma]{bandit_combi/bandit_combiExpResults/doc_len/all.csv};
            \addplot[forget plot, name path=upperblue, draw=none] table[x expr=\thisrow{t} + 1, y expr=\thisrow{less20_1e2+} * 100, col sep=comma]{bandit_combi/bandit_combiExpResults/doc_len/all.csv};
            \addplot[forget plot, name path=lowerblue, draw=none] table[x expr=\thisrow{t} + 1, y expr=\thisrow{less20_1e2-} * 100, col sep=comma]{bandit_combi/bandit_combiExpResults/doc_len/all.csv};
            \addplot[fill=blue!20, fill opacity=0.5] fill between[of=upperblue and lowerblue];
            \addplot[fill=red!20, fill opacity=0.5] fill between[of=upperred and lowerred];
            \addplot[fill=green!20, fill opacity=0.5] fill between[of=uppergreen and lowergreen];
            \legend{$\beta=1$, $\beta=10$, $\beta=100$}
        \end{axis}
    \end{tikzpicture}
    \tikzsetnextfilename{bandit_combi_20to35_std}
    \begin{tikzpicture}[baseline]
        \begin{axis}[grid style={dashed,gray!50}, axis y line*=left, axis x line*=bottom, every axis plot/.append style={line width=1.5pt, mark size=0pt, font=\huge}, width=.35\textwidth,
                height=0.4\textwidth, name=plot0, xshift=0\textwidth, xmin=1, xmax=250.0, ymin=1, ymax=25, xlabel={$T$}, legend style={at={(0.9,0.1)},anchor=south east}, smooth, title={$20 < |d| < 35$}, ymajorticks=false, xtick={50, 100, 150, 200, 250}, ytick={5,10,15,20,25}]
            \addplot[red, ylabel near ticks, line width=1.5pt] table[x expr=\thisrow{t} + 1, y expr=\thisrow{20to35_1e0} * 100, col sep=comma]{bandit_combi/bandit_combiExpResults/doc_len/all.csv};
            \addplot[green, ylabel near ticks, line width=1.5pt] table[x expr=\thisrow{t} + 1, y expr=\thisrow{20to35_1e1} * 100, col sep=comma]{bandit_combi/bandit_combiExpResults/doc_len/all.csv};
            \addplot[blue, ylabel near ticks, line width=1.5pt] table[x expr=\thisrow{t} + 1, y expr=\thisrow{20to35_1e2} * 100, col sep=comma]{bandit_combi/bandit_combiExpResults/doc_len/all.csv};
            \addplot[forget plot, name path=upperred, draw=none] table[x expr=\thisrow{t} + 1, y expr=\thisrow{20to35_1e0+} * 100, col sep=comma]{bandit_combi/bandit_combiExpResults/doc_len/all.csv};
            \addplot[forget plot, name path=lowerred, draw=none] table[x expr=\thisrow{t} + 1, y expr=\thisrow{20to35_1e0-} * 100, col sep=comma]{bandit_combi/bandit_combiExpResults/doc_len/all.csv};
            \addplot[forget plot, name path=uppergreen, draw=none] table[x expr=\thisrow{t} + 1, y expr=\thisrow{20to35_1e1+} * 100, col sep=comma]{bandit_combi/bandit_combiExpResults/doc_len/all.csv};
            \addplot[forget plot, name path=lowergreen, draw=none] table[x expr=\thisrow{t} + 1, y expr=\thisrow{20to35_1e1-} * 100, col sep=comma]{bandit_combi/bandit_combiExpResults/doc_len/all.csv};
            \addplot[forget plot, name path=upperblue, draw=none] table[x expr=\thisrow{t} + 1, y expr=\thisrow{20to35_1e2+} * 100, col sep=comma]{bandit_combi/bandit_combiExpResults/doc_len/all.csv};
            \addplot[forget plot, name path=lowerblue, draw=none] table[x expr=\thisrow{t} + 1, y expr=\thisrow{20to35_1e2-} * 100, col sep=comma]{bandit_combi/bandit_combiExpResults/doc_len/all.csv};
            \addplot[fill=blue!20, fill opacity=0.5] fill between[of=upperblue and lowerblue];
            \addplot[fill=red!20, fill opacity=0.5] fill between[of=upperred and lowerred];
            \addplot[fill=green!20, fill opacity=0.5] fill between[of=uppergreen and lowergreen];
            \end{axis}
    \end{tikzpicture}
    \tikzsetnextfilename{bandit_combi_more35_std}
    \begin{tikzpicture}[baseline]
        \begin{axis}[grid style={dashed,gray!50}, axis y line*=right, axis x line*=bottom, every axis plot/.append style={line width=1.5pt, mark size=0pt, font=\huge}, width=.35\textwidth,
                height=0.4\textwidth, name=plot0, xshift=0\textwidth, xmin=1, xmax=250.0, ymin=1, ymax=25, xlabel={$T$}, legend style={at={(1,0.1)},anchor=south east}, smooth, title={$35 \leq |d|$}, xtick={50, 100, 150, 200, 250}, ytick={5,10,15,20,25}]
            \addplot[red, ylabel near ticks, line width=1.5pt] table[x expr=\thisrow{t} + 1, y expr=\thisrow{more35_1e0} * 100, col sep=comma]{bandit_combi/bandit_combiExpResults/doc_len/all.csv};
            \addplot[green, ylabel near ticks, line width=1.5pt] table[x expr=\thisrow{t} + 1, y expr=\thisrow{more35_1e1} * 100, col sep=comma]{bandit_combi/bandit_combiExpResults/doc_len/all.csv};
            \addplot[blue, ylabel near ticks, line width=1.5pt] table[x expr=\thisrow{t} + 1, y expr=\thisrow{more35_1e2} * 100, col sep=comma]{bandit_combi/bandit_combiExpResults/doc_len/all.csv};
            \addplot[forget plot, name path=upperred, draw=none] table[x expr=\thisrow{t} + 1, y expr=\thisrow{more35_1e0+} * 100, col sep=comma]{bandit_combi/bandit_combiExpResults/doc_len/all.csv};
            \addplot[forget plot, name path=lowerred, draw=none] table[x expr=\thisrow{t} + 1, y expr=\thisrow{more35_1e0-} * 100, col sep=comma]{bandit_combi/bandit_combiExpResults/doc_len/all.csv};
            \addplot[forget plot, name path=uppergreen, draw=none] table[x expr=\thisrow{t} + 1, y expr=\thisrow{more35_1e1+} * 100, col sep=comma]{bandit_combi/bandit_combiExpResults/doc_len/all.csv};
            \addplot[forget plot, name path=lowergreen, draw=none] table[x expr=\thisrow{t} + 1, y expr=\thisrow{more35_1e1-} * 100, col sep=comma]{bandit_combi/bandit_combiExpResults/doc_len/all.csv};
            \addplot[forget plot, name path=upperblue, draw=none] table[x expr=\thisrow{t} + 1, y expr=\thisrow{more35_1e2+} * 100, col sep=comma]{bandit_combi/bandit_combiExpResults/doc_len/all.csv};
            \addplot[forget plot, name path=lowerblue, draw=none] table[x expr=\thisrow{t} + 1, y expr=\thisrow{more35_1e2-} * 100, col sep=comma]{bandit_combi/bandit_combiExpResults/doc_len/all.csv};
            \addplot[fill=blue!20, fill opacity=0.5] fill between[of=upperblue and lowerblue];
            \addplot[fill=red!20, fill opacity=0.5] fill between[of=upperred and lowerred];
            \addplot[fill=green!20, fill opacity=0.5] fill between[of=uppergreen and lowergreen];
            \end{axis}
    \end{tikzpicture}
    \caption[Impact de la taille du document sur la convergence de CUCB avec intervalle de confiance de 95 \%]
    {Impact de la taille du document sur la convergence de CUCB.
    $\Delta_T$ (plus bas est meilleur) représente la différence entre le score ROUGE du meilleur résumé
    extractif d'un document et celui suggéré CUCB après $T$ pas de temps.
    La zone pâle représente un intervalle de confiance de 95 \% sur la valeur de chacune des courbes.}
\end{figure}

\clearpage

\section*{Graphiques du chapitre \ref{chap:bandit_combi_lin}}

\tikzsetnextfilename{bandit_combi_lin_alpha_std}
\begin{figure}[ht!]
    \begin{center}
        \begin{tikzpicture}
            \begin{axis}[grid style={dashed,gray!50}, axis y line*=left, axis x line*=bottom, every axis plot/.append style={line width=1.5pt, mark size=0pt, font=\huge}, name=plot0, ylabel={$\Delta_T$}, xlabel={$T$}, width=0.95\textwidth, height=0.4\textwidth, smooth, xmin=1, xmax=99, y tick label style={/pgf/number format/fixed}, ytick={5, 10, 15, 20, 25}, ymax=25, ymin=0.0]
                \addplot[red, ylabel near ticks, line width=1.5pt] table[x expr=\thisrow{t} + 1, y expr=\thisrow{1000000_} * 100, col sep=comma]{bandit_combi_lin/bandit_combi_linExpResults/alpha/all.csv};
                \addplot[green, ylabel near ticks, line width=1.5pt] table[x expr=\thisrow{t} + 1, y expr=\thisrow{10000000.0_} * 100, col sep=comma]{bandit_combi_lin/bandit_combi_linExpResults/alpha/all.csv};
                \addplot[blue, ylabel near ticks, line width=1.5pt] table[x expr=\thisrow{t} + 1, y expr=\thisrow{100000000.0_} * 100, col sep=comma]{bandit_combi_lin/bandit_combi_linExpResults/alpha/all.csv};
                \addplot[black, dashed, ylabel near ticks, line width=1.5pt] table[x=t, y expr=\thisrow{q_1e1} * 100, col sep=comma]{bandit_combi/bandit_combiExpResults/alpha/all.csv};
                \addplot[forget plot, name path=upperred, draw=none] table[x expr=\thisrow{t} + 1, y expr=\thisrow{1000000_+} * 100, col sep=comma]{bandit_combi_lin/bandit_combi_linExpResults/alpha/all.csv};
                \addplot[forget plot, name path=lowerred, draw=none] table[x expr=\thisrow{t} + 1, y expr=\thisrow{1000000_-} * 100, col sep=comma]{bandit_combi_lin/bandit_combi_linExpResults/alpha/all.csv};
                \addplot[forget plot, name path=upperblue, draw=none] table[x expr=\thisrow{t} + 1, y expr=\thisrow{100000000.0_+} * 100, col sep=comma]{bandit_combi_lin/bandit_combi_linExpResults/alpha/all.csv};
                \addplot[forget plot, name path=lowerblue, draw=none] table[x expr=\thisrow{t} + 1, y expr=\thisrow{100000000.0_-} * 100, col sep=comma]{bandit_combi_lin/bandit_combi_linExpResults/alpha/all.csv};
                \addplot[forget plot, name path=uppergreen, draw=none] table[x expr=\thisrow{t} + 1, y expr=\thisrow{10000000.0_+} * 100, col sep=comma]{bandit_combi_lin/bandit_combi_linExpResults/alpha/all.csv};
                \addplot[forget plot, name path=lowergreen, draw=none] table[x expr=\thisrow{t} + 1, y expr=\thisrow{10000000.0_-} * 100, col sep=comma]{bandit_combi_lin/bandit_combi_linExpResults/alpha/all.csv};
                \addplot[fill=blue!20, fill opacity=0.5] fill between[of=upperblue and lowerblue];
                \addplot[fill=red!20, fill opacity=0.5] fill between[of=upperred and lowerred];
                \addplot[fill=green!20, fill opacity=0.5] fill between[of=uppergreen and lowergreen];
                \legend{$\beta=10^6$, $\beta = 10^7$, $\beta = 10^8$, CUCB}
            \end{axis}
        \end{tikzpicture}
    \end{center}
    \caption[Impact du paramètre d'exploration $\beta$ sur la convergence de LinCUCB avec intervalle de confiance de 95 \%]
    {Impact du paramètre d'exploration $\beta$ utilisé par LinCUCB sur la convergence.
    $\Delta_T$ (plus bas est meilleur) représente la différence entre le score ROUGE du meilleur résumé
    extractif d'un document et celui suggéré CUCB après $T$ pas de temps.
    La zone pâle représente un intervalle de confiance de 95 \% sur la valeur de chacune des courbes.}
\end{figure}

\begin{figure}[ht!]
    \tikzsetnextfilename{bandit_combi_lin_less20_std}
    \begin{tikzpicture}[baseline]
        \begin{axis}[grid style={dashed,gray!50}, axis y line*=left, axis x line*=bottom, every axis plot/.append style={line width=1.5pt, mark size=0pt, font=\Large}, width=0.34\textwidth,
                height=0.4\textwidth, name=plot0,  xmin=1.0, xmax=100.0, ymin=1, ymax=25, ylabel={$\Delta_T$}, xlabel={$T$}, smooth, title={$|d| \leq 20$}, ytick={5, 10, 15, 20, 25}]
            \addplot[red, ylabel near ticks, line width=1.5pt] table[x expr=\thisrow{t} + 1, y expr=\thisrow{less20_1000000_} * 100, col sep=comma]{bandit_combi_lin/bandit_combi_linExpResults/doc_len/all.csv};
            \addplot[green, ylabel near ticks, line width=1.5pt] table[x expr=\thisrow{t} + 1, y expr=\thisrow{less20_10000000.0_} * 100, col sep=comma]{bandit_combi_lin/bandit_combi_linExpResults/doc_len/all.csv};
            \addplot[blue, ylabel near ticks, line width=1.5pt] table[x expr=\thisrow{t} + 1, y expr=\thisrow{less20_100000000.0_} * 100, col sep=comma]{bandit_combi_lin/bandit_combi_linExpResults/doc_len/all.csv};
            \addplot[forget plot, name path=upperred, draw=none] table[x expr=\thisrow{t} + 1, y expr=\thisrow{less20_1000000_+} * 100, col sep=comma]{bandit_combi_lin/bandit_combi_linExpResults/doc_len/all.csv};
            \addplot[forget plot, name path=lowerred, draw=none] table[x expr=\thisrow{t} + 1, y expr=\thisrow{less20_1000000_-} * 100, col sep=comma]{bandit_combi_lin/bandit_combi_linExpResults/doc_len/all.csv};
            \addplot[forget plot, name path=upperblue, draw=none] table[x expr=\thisrow{t} + 1, y expr=\thisrow{less20_100000000.0_+} * 100, col sep=comma]{bandit_combi_lin/bandit_combi_linExpResults/doc_len/all.csv};
            \addplot[forget plot, name path=lowerblue, draw=none] table[x expr=\thisrow{t} + 1, y expr=\thisrow{less20_100000000.0_-} * 100, col sep=comma]{bandit_combi_lin/bandit_combi_linExpResults/doc_len/all.csv};
            \addplot[forget plot, name path=uppergreen, draw=none] table[x expr=\thisrow{t} + 1, y expr=\thisrow{less20_10000000.0_+} * 100, col sep=comma]{bandit_combi_lin/bandit_combi_linExpResults/doc_len/all.csv};
            \addplot[forget plot, name path=lowergreen, draw=none] table[x expr=\thisrow{t} + 1, y expr=\thisrow{less20_10000000.0_-} * 100, col sep=comma]{bandit_combi_lin/bandit_combi_linExpResults/doc_len/all.csv};
            \addplot[fill=blue!20, fill opacity=0.5] fill between[of=upperblue and lowerblue];
            \addplot[fill=red!20, fill opacity=0.5] fill between[of=upperred and lowerred];
            \addplot[fill=green!20, fill opacity=0.5] fill between[of=uppergreen and lowergreen];
            \legend{$\beta=10^6$, $\beta = 10^7$, $\beta = 10^8$}
        \end{axis}
    \end{tikzpicture}
    \tikzsetnextfilename{bandit_combi_lin_20to35_std}
    \begin{tikzpicture}[baseline]
        \begin{axis}[grid style={dashed,gray!50}, axis y line*=left, axis x line*=bottom, every axis plot/.append style={line width=1.5pt, mark size=0pt, font=\huge}, width=.34\textwidth,
                height=0.4\textwidth, name=plot0, xshift=0\textwidth, xmin=1.0, xmax=100.0, ymin=1, ymax=25, xlabel={$T$}, legend style={at={(0.9,0.1)},anchor=south east}, smooth, title={$20 < |d| < 35$}, ymajorticks=false]
            \addplot[red, ylabel near ticks, line width=1.5pt] table[x expr=\thisrow{t} + 1, y expr=\thisrow{20to35_1000000_} * 100, col sep=comma]{bandit_combi_lin/bandit_combi_linExpResults/doc_len/all.csv};
            \addplot[green, ylabel near ticks, line width=1.5pt] table[x expr=\thisrow{t} + 1, y expr=\thisrow{20to35_10000000.0_} * 100, col sep=comma]{bandit_combi_lin/bandit_combi_linExpResults/doc_len/all.csv};
            \addplot[blue, ylabel near ticks, line width=1.5pt] table[x expr=\thisrow{t} + 1, y expr=\thisrow{20to35_100000000.0_} * 100, col sep=comma]{bandit_combi_lin/bandit_combi_linExpResults/doc_len/all.csv};
            \addplot[forget plot, name path=upperred, draw=none] table[x expr=\thisrow{t} + 1, y expr=\thisrow{20to35_1000000_+} * 100, col sep=comma]{bandit_combi_lin/bandit_combi_linExpResults/doc_len/all.csv};
            \addplot[forget plot, name path=lowerred, draw=none] table[x expr=\thisrow{t} + 1, y expr=\thisrow{20to35_1000000_-} * 100, col sep=comma]{bandit_combi_lin/bandit_combi_linExpResults/doc_len/all.csv};
            \addplot[forget plot, name path=upperblue, draw=none] table[x expr=\thisrow{t} + 1, y expr=\thisrow{20to35_100000000.0_+} * 100, col sep=comma]{bandit_combi_lin/bandit_combi_linExpResults/doc_len/all.csv};
            \addplot[forget plot, name path=lowerblue, draw=none] table[x expr=\thisrow{t} + 1, y expr=\thisrow{20to35_100000000.0_-} * 100, col sep=comma]{bandit_combi_lin/bandit_combi_linExpResults/doc_len/all.csv};
            \addplot[forget plot, name path=uppergreen, draw=none] table[x expr=\thisrow{t} + 1, y expr=\thisrow{20to35_10000000.0_+} * 100, col sep=comma]{bandit_combi_lin/bandit_combi_linExpResults/doc_len/all.csv};
            \addplot[forget plot, name path=lowergreen, draw=none] table[x expr=\thisrow{t} + 1, y expr=\thisrow{20to35_10000000.0_-} * 100, col sep=comma]{bandit_combi_lin/bandit_combi_linExpResults/doc_len/all.csv};
            \addplot[fill=blue!20, fill opacity=0.5] fill between[of=upperblue and lowerblue];
            \addplot[fill=red!20, fill opacity=0.5] fill between[of=upperred and lowerred];
            \addplot[fill=green!20, fill opacity=0.5] fill between[of=uppergreen and lowergreen];
        \end{axis}
    \end{tikzpicture}
    \tikzsetnextfilename{bandit_combi_lin_more35_std}
    \begin{tikzpicture}[baseline]
        \begin{axis}[grid style={dashed,gray!50}, axis y line*=right, axis x line*=bottom, every axis plot/.append style={line width=1.5pt, mark size=0pt, font=\huge}, width=.34\textwidth,
                height=0.4\textwidth, name=plot0, xshift=0\textwidth,  xmin=1.0, xmax=100.0, ymin=1, ymax=25, xlabel={$T$}, legend style={at={(1,0.1)},anchor=south east}, smooth, title={$35 \leq |d|$}, ytick={5, 10, 15, 20, 25}]
            \addplot[red, ylabel near ticks, line width=1.5pt] table[x expr=\thisrow{t} + 1, y expr=\thisrow{more35_1000000_} * 100, col sep=comma]{bandit_combi_lin/bandit_combi_linExpResults/doc_len/all.csv};
            \addplot[green, ylabel near ticks, line width=1.5pt] table[x expr=\thisrow{t} + 1, y expr=\thisrow{more35_10000000.0_} * 100, col sep=comma]{bandit_combi_lin/bandit_combi_linExpResults/doc_len/all.csv};
            \addplot[blue, ylabel near ticks, line width=1.5pt] table[x expr=\thisrow{t} + 1, y expr=\thisrow{more35_100000000.0_} * 100, col sep=comma]{bandit_combi_lin/bandit_combi_linExpResults/doc_len/all.csv};
            \addplot[forget plot, name path=upperred, draw=none] table[x expr=\thisrow{t} + 1, y expr=\thisrow{more35_1000000_+} * 100, col sep=comma]{bandit_combi_lin/bandit_combi_linExpResults/doc_len/all.csv};
            \addplot[forget plot, name path=lowerred, draw=none] table[x expr=\thisrow{t} + 1, y expr=\thisrow{more35_1000000_-} * 100, col sep=comma]{bandit_combi_lin/bandit_combi_linExpResults/doc_len/all.csv};
            \addplot[forget plot, name path=upperblue, draw=none] table[x expr=\thisrow{t} + 1, y expr=\thisrow{more35_100000000.0_+} * 100, col sep=comma]{bandit_combi_lin/bandit_combi_linExpResults/doc_len/all.csv};
            \addplot[forget plot, name path=lowerblue, draw=none] table[x expr=\thisrow{t} + 1, y expr=\thisrow{more35_100000000.0_-} * 100, col sep=comma]{bandit_combi_lin/bandit_combi_linExpResults/doc_len/all.csv};
            \addplot[forget plot, name path=uppergreen, draw=none] table[x expr=\thisrow{t} + 1, y expr=\thisrow{more35_10000000.0_+} * 100, col sep=comma]{bandit_combi_lin/bandit_combi_linExpResults/doc_len/all.csv};
            \addplot[forget plot, name path=lowergreen, draw=none] table[x expr=\thisrow{t} + 1, y expr=\thisrow{more35_10000000.0_-} * 100, col sep=comma]{bandit_combi_lin/bandit_combi_linExpResults/doc_len/all.csv};
            \addplot[fill=blue!20, fill opacity=0.5] fill between[of=upperblue and lowerblue];
            \addplot[fill=red!20, fill opacity=0.5] fill between[of=upperred and lowerred];
            \addplot[fill=green!20, fill opacity=0.5] fill between[of=uppergreen and lowergreen];
        \end{axis}
    \end{tikzpicture}
    \caption[Impact de la taille du document sur la convergence de LinCUCB avec intervalle de confiance de 95 \%]
    {Impact de la taille du document sur la convergence de LinCUCB.
    $\Delta_T$ (plus bas est meilleur) représente la différence entre le score ROUGE du meilleur résumé
    extractif d'un document et celui suggéré CUCB après $T$ pas de temps.
    La zone pâle représente un intervalle de confiance de 95 \% sur la valeur de chacune des courbes.}
\end{figure}

\bibliographystyle{abbrvnat}         % production de la bibliographie
\bibliography{biblio}

\end{document}
