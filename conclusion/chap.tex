\chapter*{Conclusion}           % ne pas numéroter
\label{chap:conclusion}         % étiquette pour renvois
\phantomsection\addcontentsline{toc}{chapter}{\nameref{chap:conclusion}} % inclure dans TdM

Dans ce mémoire, on a abordé la problématique de 
l'entraînement de modèles de génération de résumés de documents.
On a considéré les résumés dits extractifs, qui sont bâtis 
en sélectionnant des phrases du document initial.
Pour mener nos expérimentations, on s'est intéressés 
au jeu de données du CNN/DailyMail \citep{hermann2015teaching}, 
regroupant plus de 300 000 articles de journaux et leurs résumés.
Notre intérêt était de démontrer comment les approches par bandit 
\citep{Robbins:1952} peuvent être mises à profit dans des algorithmes d'entraînement 
pour la génération de résumés.

Au chapitre \ref{chap:bandit_contextuel}, on a d'abord détaillé 
la problématique à l'étude ainsi que le cadre uniformisé 
utilisé tout au long du document pour comparer la performance
des algorithmes considérés.
Le cadre uniformisé retenu est de considérer seulement les résumés extractifs de 
3 phrases et d'utiliser le même réseau de neurones que BanditSum \citep{dong2018banditsum}.
On a aussi mis sur pied un jeu de développement constitué de 25 000 
documents sur le jeu de données du CNN/DailyMail qui est utilisé pour 
déterminer l'applicabilité des formulations bandits considérées.
Ensuite, on a présenté BanditSum en détails, qui voit la génération 
des résumés sur un jeu de données en entier comme un bandit contextuel.
BanditSum exploite la formulation contextuelle en utilisant l'algorithme REINFORCE \citep{williams1992simple} 
pour entraîner un système de génération de résumés optimisant la qualité des résumés 
générés en espérance.
À partir du jeu de développement, on a établi de manière rigoureuse 
que le nombre d'échantillons $N=20$ utilisé par BanditSum pour REINFORCE 
est approprié.
Enfin, on a présenté qu'avec un modèle contenant 25 fois de paramètres 
et avec un entraînement sur 10 fois moins de documents, BanditSum 
obtient une performance seulement légèrement inférieure aux meilleurs 
algorithmes de l'état de l'art, établissant sa supériorité 
en matière de temps de calcul.

Au chapitre \ref{chap:bandit_combi}, on s'est intéressé
à un second niveau d'application pour le problème de bandit: 
l'application à la génération du résumé 
d'un seul document.
On a présenté comment la génération du résumé d'un document 
peut être vue comme un problème de bandit combinatoire \citep{pmlr-v28-chen13a}.
On a ensuite introduit la notion de potentiel extractif d'une phrase,
utilisable comme cible quantifiable pour l'affinité d'un système de génération 
de résumés à inclure une phrase dans le résumé d'un document.
On a détaillé l'algorithme CUCB \citep{pmlr-v28-chen13a}
permettant de résoudre le bandit combinatoire proposé et pouvant 
être utilisé pour estimer le potentiel extractif de chaque phrase 
d'un document.
Les cibles basées sur le potentiel extractif des phrases 
sont utilisées par un nouvel algorithme, nommé CombiSum, 
que l'on a mis à l'essai sur le jeu de données CNN/DailyMail.
Nos résultats ont montré que les cibles proposées ne permettent 
pas un meilleur entraînement que les cibles basées sur un oracle 
\citep{10.5555/3298483.3298681}, car les modèles produits 
convergent tous vers une performance très similaire à l'heuristique 
de référence Lead-3 \citep{10.5555/3298483.3298681}.

Au chapitre \ref{chap:bandit_combi_lin}, on a montré comment 
il est possible d'incorporer les similarités entre les phrases 
d'un document pour voir la génération du résumé d'un document 
comme un bandit linéaire combinatoire \citep{NEURIPS2018_207f8801}.
On a présenté l'algorithme LinCUCB, une variante de CUCB 
qui permet d'exploiter l'hypothèse linéaire pour résoudre 
le bandit linéaire combinatoire à un coût de calcul amoindri.
On a utilisé LinCUCB pour générer des cibles basées sur le potentiel 
extractif des phrases d'un document dans un processus plus efficace 
en termes de coût de calcul que celui basé sur CUCB.
L'algorithme LinCombiSum, utilisant les cibles obtenues à partir de LinCUCB,
a été présenté et mis à l'épreuve sur le jeu de données du CNN/DailyMail.
Les modèles produits par LinCombiSum obtiennent encore une fois une performance 
très similaire à l'heuristique Lead-3.
On a analysé plus en détails la convergence de CombiSum, LinCombiSum et un
modèle entraîné selon les cibles issues d'un oracle et prouvé que 
les modèles produits à partir de toutes les cibles génèrent 
leurs résumés d'une manière très similaire à l'heuristique Lead-3.
On a émis l'hypothèse que cette convergence est dûe à la nature 
de la distribution de l'information contenue dans les articles de journaux ainsi 
qu'à l'expressivité limitée du modèle neuronal employé.

En somme, les travaux présentés ont permis d'établir comment les formulations en bandit 
contextuel, combinatoire et linéaire combinatoire peuvent être utilisées 
pour donner naissance à des approches par bandit pour l'entraînement 
de modèle de génération résumés.
L'utilisation des bandits aura permis aux algorithmes abordés une efficacité accrue,
mais surtout une façon directe d'incorporer la différence de qualité entre 
deux résumés dans la procédure d'entraînement.

\section*{Contributions}

Les contributions principales des travaux présentés sont les suivantes :

\begin{itemize}
    \item \textbf{Algorithme d'entraînement CombiSum.}
    En décrivant le problème de la génération du résumé extractif d'un 
    document comme un bandit combinatoire, CombiSum permet un changement de perspective
    important sur la production de cibles d'entraînement pour un document.
    En effet, la formulation combinatoire alloue l'incorporation d'information 
    relative aux résumés non-optimaux d'un document dans la génération de cibles,
    en contraste avec les cibles binaires actuelles basées sur le meilleur résumé selon un oracle.
    \item \textbf{Algorithme d'entraînement LinCombiSum.}
    LinCombiSum exploite la notion de similarité entre les phrases pour visualiser 
    le problème de la génération du résumé extractif d'un document comme un bandit 
    linéaire combinatoire.
    D'une manière similaire à ce qui est fait par CombiSum, LinCombiSum utilise 
    la formulation en bandit linéaire combinatoire pour générer des cibles d'entraînement
    plus riches pour chaque document.
    La génération des cibles est toutefois obtenue à un coût de calcul moins 
    élevé grâce à l'incorporation de similarités entre les phrases d'un même 
    document pour faciliter la résolution du problème de bandit.
\end{itemize}

\section*{Défis et perspectives}

Plusieurs modifications pourraient être apportées à nos travaux afin 
de les améliorer.

Un premier angle à partir duquel il serait possible d'améliorer nos travaux
est celui de la relaxation des contraintes sur les résumés extractifs considérés.
Notre étude s'arrête au cas où l'on génère un résumé à partir 
d'un groupe non-ordonné de 3 phrases.
Il serait intéressant d'incorporer des formulations plus souples 
qui permettent de sélectionner un nombre variable de phrases d'un document.
Une première piste en ce sens serait de considérer une critère d'arrêt 
flexible dans le processus de génération de résumés à partir d'affinités,
comme le font \citet{luo-etal-2019-reading}. 
On pourrait aussi considérer le cas où les phrases sont sélectionnées 
les unes après les autres, permettant notamment au modèle d'apprendre 
directement à éviter la répétition.
Cette dernière approche, s'apparentant à l'apprentissage par renforcement \citep{Sutton1998},
est d'ailleurs déjà explorée en partie par \citet{zhong-etal-2020-extractive}. 

Une seconde région où il serait possible de faire progresser notre réflexion est 
celle des algorithmes utilisés pour la génération de cibles.
En effet, les algorithmes CUCB et LinCUCB employés sont 
conçus pour le bandit combinatoire général et n'exploitent donc pas 
directement la variante multitâche à laquelle on s'intéresse.
Des algorithmes comme KMTL-UCB \citep{deshmukh2017multi} et MT-LinUCB \citep{soare2014multi},
conçus spécialement pour le bandit multitâche, pourraient notamment être envisagés
pour la génération de cibles d'entraînement.

Une troisième façon de bonifier nos travaux serait de voir comment les approches 
par bandit présentées performent pour l'entraînement de modèles 
bien plus imposants, comme celui utilisé par \citet{liu2019text}.
En effet, l'architecture de réseau de neurones utilisée par BanditSum 
et reprise dans nos travaux est très peu expressive en comparaison avec les modèles
de l'état de l'art.
Bien que cela permette d'établir l'efficacité computationnelle 
de BanditSum, on croit aussi qu'il peut s'agir de la raison 
pour laquelle la performance avec nos nouvelles cibles a
obtenu des résultats aussi peu convaincants.