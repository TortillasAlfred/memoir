\chapter{Échantillonnage}
\label{chap:mcs}                   % étiquette pour renvois (à compléter!)

Idée: si je possède une distribution sur les phrases, j'aimerais savoir
quelle est la performance qui lui est associée (processus de génération
stochastique seulement).

Concrètement, on cherche donc à approximer

\begin{equation}
    f = \underset{{s \sim \phi(\pi(d))}}{\mathbb{E}} \,\left[R(s, \hat{s})\right].
\end{equation}

Cette fonction ne peut pas être calculée analytiquement dans un temps
raisonnable car le nombre de résumés possibles à considérer est trop grand
(exponentiel en nombre de phrases).

Or, pour estimer $f$, on n'a pas \textbf{besoin} de vérifier tous les résumés
possibles: on a seulement besoin de ceux qui sont les plus susceptibles selon
$\phi(\pi(d))$.
À mentionner: la différence de probabilité entre deux résumés finit par outrepasser
la différence de $R$ entre les deux mêmes résumés, c'est aussi pour ça que ça marche.

On procède donc avec une approche nommé Échantillonnage de Monte-Carlo, où on
calcule plutôt

\begin{equation}
    \tilde{f} = \frac{1}{N} \sum_{t=1}^N R(s_t, \hat{s}),
\end{equation}

avec les résumés $s_t$ échantillonnés selon $\phi(\pi(d))$. Notons que $\tilde{f}$
est un estimateur non-biaisé de $f$.

\section{Expériences sur l'échantillonnage}

Pour les tests, on prend 10 000 documents d'entraînement du jeu de données.

On pige aléatoirement des distributions $\pi(d)$ qui sont une moyenne pondérée
entre une distribution uniforme $U(d)$ et une distribution \textit{3-hot} $G(d)$
sous la forme

\begin{equation}
    \pi(d) = (1 - \tau) U(d) + \tau G(d).
\end{equation}

On explore $\tau \in \{\frac{n}{100}: 0 \leq n \leq 100\}$ en repigeant les 3
index non-nuls de $G(d)$ à chaque fois. On ajoute un bruit gaussien sur chaque $\pi(d)$.

On calcule analytiquement $f$ grâce au fait que l'on possède tous les scores
ROUGE de tous les résumés possibles pour les documents du jeu de données.
On pige 1000 échantillons de $\phi(\pi(d))$, où $\phi$ est le processus de
pige sans remise de 3 phrases.
On calcule ainsi $\tilde{f}_t$ pour $t$ jusqu'à 1000.

\section{Résultats échantillonnage}

\commentaire{Pour les tableaux, je veux surtout votre input sur (1) la pertinence
    de chacun, (2) la manière de les présenter (j'ai les données brutes, je peux
    facilement réorganiser comment je présente les donnnées) et (3) s'il y a des
    expériences supplémentaires que vous jugez pertinentes.}

\begin{tikzpicture}
    \begin{axis}[title={Erreur de l'échantillonnage}, grid style={dashed,gray!50}, axis y line*=left, axis x line*=bottom, every axis plot/.append style={line width=1.5pt, mark size=0pt, font=\huge}, name=plot0, xshift=-.1\textwidth, y tick label style={/pgf/number format/fixed zerofill}, ylabel={Erreur absolue * 100}, xlabel={Nombre d'échantillons}, width=\textwidth, smooth, xmin=0, xmax=50]
        \addplot[blue, ylabel near ticks, line width=3pt] table[x=x, y=y, col sep=comma]{mcs/mcsExpResults/overall/mean.csv};
        \addplot[red, ylabel near ticks, line width=3pt] table[x=x, y=y, col sep=comma]{mcs/mcsExpResults/overall/less20.csv};
        \addplot[green, ylabel near ticks, line width=3pt] table[x=x, y=y, col sep=comma]{mcs/mcsExpResults/overall/20to35.csv};
        \addplot[yellow, ylabel near ticks, line width=3pt] table[x=x, y=y, col sep=comma]{mcs/mcsExpResults/overall/more35.csv};
        \legend{Moyenne, $|d| \leq 20$, $20 < |d| < 35$, $35 \leq |d|$}
    \end{axis}
\end{tikzpicture}

\begin{tikzpicture}
    \begin{axis}[title={Erreur de l'échantillonnage selon l'entropie}, grid style={dashed,gray!50}, axis y line*=left, axis x line*=bottom, every axis plot/.append style={line width=1.5pt, mark size=0pt, font=\huge}, name=plot0, xshift=-.1\textwidth, y tick label style={/pgf/number format/fixed zerofill}, ylabel={Erreur absolue * 100}, xlabel={Entropie}, width=\textwidth, smooth]
        \addplot[blue, ylabel near ticks, line width=3pt] table[x=x, y=t8, col sep=comma]{mcs/mcsExpResults/entropy/all.csv};
        \addplot[red, ylabel near ticks, line width=3pt] table[x=x, y=t16, col sep=comma]{mcs/mcsExpResults/entropy/all.csv};
        \addplot[green, ylabel near ticks, line width=3pt] table[x=x, y=t32, col sep=comma]{mcs/mcsExpResults/entropy/all.csv};
        \addplot[yellow, ylabel near ticks, line width=3pt] table[x=x, y=t64, col sep=comma]{mcs/mcsExpResults/entropy/all.csv};
        \legend{$t=8$, $t=16$, $t=32$, $t=64$}
    \end{axis}
\end{tikzpicture}

\begin{tikzpicture}
    \begin{axis}[title={Erreur de l'échantillonnage selon la plus grande probabilité présente}, grid style={dashed,gray!50}, axis y line*=left, axis x line*=bottom, every axis plot/.append style={line width=1.5pt, mark size=0pt, font=\huge}, name=plot0, xshift=-.1\textwidth, y tick label style={/pgf/number format/fixed zerofill}, ylabel={Erreur absolue * 100}, xlabel={Probabilité maximale de $\phi(\pi(d))$}, width=\textwidth, smooth]
        \addplot[blue, ylabel near ticks, line width=3pt] table[x=x, y=t8, col sep=comma]{mcs/mcsExpResults/top3/all.csv};
        \addplot[red, ylabel near ticks, line width=3pt] table[x=x, y=t16, col sep=comma]{mcs/mcsExpResults/top3/all.csv};
        \addplot[green, ylabel near ticks, line width=3pt] table[x=x, y=t32, col sep=comma]{mcs/mcsExpResults/top3/all.csv};
        \addplot[yellow, ylabel near ticks, line width=3pt] table[x=x, y=t64, col sep=comma]{mcs/mcsExpResults/top3/all.csv};
        \legend{$t=8$, $t=16$, $t=32$, $t=64$}
    \end{axis}
\end{tikzpicture}

Analyse:

\begin{itemize}
    \item Absolument aucun impact de la taille des documents
    \item Bizarre le creux à 1 pour l'entropie.
    \item Semblerait que $t=16$ est un bon compromis pour être rapide mais pas trop faire d'erreurs. C'est ce que BanditSUM ont aussi.
\end{itemize}

\section{Intégration dans un modèle complet}

Pour un modèle avec des paramètres $\theta$, on souhaite maximiser
la fonction objectif

\begin{equation}
    J(\theta) = f = \underset{{s \sim \phi(\pi(d))}}{\mathbb{E}} \,\left[R(s, \hat{s})\right].
\end{equation}

Comme on a vu qu'on peut approximer $f$ efficacement, on peut se limiter
à faire une ascension de gradient sur l'estimateur non-biaisé

\begin{equation}
    \tilde{J}(\theta) = \frac{1}{N} \sum_{t=1}^N R(s_t, \hat{s}).
\end{equation}

On peut alors utiliser l'ascension de gradient et mettre à jour les poids $\theta$
selon la règle

\begin{equation*}
    \theta = \theta + \nabla_\theta \tilde{J}(\theta).
\end{equation*}

C'est exactement ce que fait BanditSUM \citep{dong2018banditsum}.

\question{Ici, je reprends la logique derrière les approches policy gradient (RENFORCE
    notamment) en RL. J'ai décidé que ce serait pas souhaitable de m'étaler sur quoique ce soit
    RL dans mon mémoire alors je sais pas à quel point je devrais prendre un moment
    pour mentionner que j'emprunte cette logique là à des travaux antérieurs.}

BanditSUM introduit une baseline pour stabilier les gradients de J
et prend un document à la fois, en posant $N=16$.

\todo{Détails du processus d'entraînement complet BanditSUM}

\commentaire{Dernière rencontre j'avais parlé de considérer un autre processus de génération
    de résumés que le sampling qui arrête à 3, mais après réflexion je pense pas que ça va
    apporter quelque chose tant que ça et que ça va juste alourdir la lecture. Je vais
    le garder pour une conclusion avec les avenues possibles pour les futurs travaux.}

\commentaire{Les expériences pour BanditSUM sont en train de rouler. Je pense surtout
    rapporter la courbe de performance. Peut-être aussi rapporter comme plafond l'état de l'art ?}

\section{Conclusion}

\todo{à faire une fois les résultats de BanditSUM en main}