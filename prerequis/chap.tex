% !TeX root = ./main.tex
\chapter{Prérequis}     % numéroté
\label{chap:montecarlo}                   % étiquette pour renvois (à compléter!)

\commentaire{Tout ceci est seulement en place à titre indicatif.
    Toute information pertinente sera insérée ou retirée en fonction
    du développement des chapitres de corps du mémoire.}

\section{Méthodes de Monte-Carlo}

Approximation statistique de procédés déterministes

\subsection{Estimation de Monte-Carlo}

On a une fonction qui est faite sous la forme d'une espérance.

En samplant directement la distribution, on peut converger rapidement
à un bon estimé de la vraie valeur.

Applications variées: cas de calcul de $\pi$ en tirant au hasard plusieurs
fois deux nombres entre -1 et 1.

\subsection{Recherche arborescente de Monte-Carlo}

On a une fonction qui affiche une structure arborescente
(séquentielle avec direction) et on souhaite avoir son maximum (max de -f si min).

On doit parcourir un nombre suffisant de feuilles de l'arbre de $f$ pour
s'assurer de l'optimalité de notre réponse mais on veut aussi s'éviter d'avoir
à visiter toutes les feuilles (sinon un minmax ferait le travail).

Idée: explorer les sous-arbres en fonction des feuilles vues à présent,
en priorisant les sous-arbres \textit{payants} et ceux qui ont été
moins explorés.

Variante linéaire: si on dispose de représentations pour les noeuds de
l'arbre, on peut utiliser la représentation d'un noeud exploré pour
tenter d'estimer les valeurs associées à d'autre noeuds.

\section{Réseaux de neurones}

\subsection{Réseaux pleinement connectés}

Présentation haut niveau de leur utilité prouvée empiriquement
sur à peu près n'importe quel type de problème d'apprentissage
supervisé.
\begin{itemize}
    \item Préciser architecture : neurone, activation
    \item Préciser fonction de perte : doit être intimement connectée
          au savoir préalable (prior) sur la tâche et la cible.
\end{itemize}

\subsection{Réseaux récurrents}

Cas où les données sont des séquences de tailles variables (i.e. données
textuelles).

\begin{itemize}
    \item RNN
    \item LSTM : pour quelle raison ?
\end{itemize}