\chapter*{Introduction}
\label{chap:introduction}       % étiquette pour renvois

\commentaire{Toute la portion de mise en contexte n'est pas touchée. Pas pertinent
    de la lire, comme elle n'incorpore pas encore les approches par bandit.}

Ère du big data: on a une quantité immense de données, notamment textuelles.

Succès des dernières décennies nous montrent le potentiel qui peut se cacher
derrière une utilisation judicieuse de ces données.

Or, il n'est pas toujours possible d'incorporer toutes les données dans
un processus de traitement.

Notamment, dans certains domaines, les humains sont encore requis dans la loop
dans des applications de traitement de données textuelles en raison de
législations ou d'assurance qualité (i.e. processus de réclamations en assurance).

Comment faire profiter du grand afflux de données dans des systèmes où l'humain (et
sa capacité de traitement limitée) sont essentiels ?

Une option qui peut s'avérer facilement attrayante est l'idée de condenser un ou plusieurs
documents en un texte concis contenant seulement l'information requise pour procéder à la
tâche à exécuter.

On dit alors que l'on fait appel à un système de génération automatique de résumés de textes.

Comment fonctionnent ces systèmes de génération de résumés ?

L'espace des résumés possibles est très vaste; comment faire pour en trouver un qui est
suffisamment bon dans un délai de temps assez raisonnable pour l'utilisation dans un contexte
du monde réel ?

Des outils probabilistes, connus sous le nom de méthodes de Monte-Carlo, ont été
spécifiquement conçus pour de tels contextes.

Ils se basent sur l'aléatoire pour parcourir efficacement des espaces potentiellement
très grands.

En les jumelant à des approches neuronales ayant démontré largement leur efficacité sur des données
textuelles, il est naturel de s'attendre à ce que ces outils permettent d'obtenir des
systèmes de génération automatique qui produisent non seulement d'excellents résumés, mais qui
ne le font pas au détriment d'un temps d'entraînement déraisonnable.

\section*{Objectifs}

\commentaire{À partir d'ici c'est mis à jour et prêt à révision.}

Répondre aux questions suivantes :

\begin{itemize}
    \item Comment peut-on formuler la génération de résumés comme un problème de bandit multi-bras (MAB) ?
          Quels sont les algorithmes de la littérature utilisés habituellement pour résoudre ces problèmes ?
    \item Dans quelle mesure ces algorithmes de bandit sont-ils performants dans le contexte de résumés?
          Sont-ils plus rapides en temps de calcul ? En nombre d'échantillons requis pour une
          bonne performance ? Est-ce que les solutions auxquelles ils convergent sont meilleures ?
    \item Comment peut-on intégrer ces formulations bandits dans un algorithme qui produit
          un système de génération de résumés ?
    \item Comment l'introduction de ces artefacts bandits a-t-elle affecté la performance ?
          Est-ce que les systèmes produits génèrent des résumés de meilleure qualité ?
          A-t-on besoin de moins d'exemples avant d'avoir un système de haute qualité ?
          Est-ce que l'entraînement est plus rapide à effectuer ? Avec quelles ressources de calcul ?
    \item Comment les performances de ces algorithmes se comparent-ils entre eux ? Et par rapport à l'état de l'art ?
\end{itemize}

On répond aux objectifs en proposant 3 formulations de bandits applicables au processus
de la génération de résumés.

Une des approches proposée est présente dans la littérature mais les deux autres sont
des nouveautés, du mieux de notre connaissance.

\section*{Structure du mémoire}

Ce mémoire débutera par une introduction aux concepts de base essentiels à la compréhension
du reste du document.

Le document sera ensuite divisé en trois chapitres pour les trois formulations de MAB explorées:
bandit contextuel, bandit combinatoire et bandit combinatoire linéaire.

Pour chacun des chapitres, on définira le cadre MAB proposé.
Des expériences empiriques seront effectuées pour justifier l'applicabilité
du cadre au contexte de génération de résumés.

On proposera ensuite un modèle neuronal
de génération automatique de textes utilisant le cadre proposé.

Des expériences seront ensuite effectuées pour évaluer la performance du modèle.

Les hyperparamètres de la portion bandit seront choisis en fonction des tests
empiriques préliminaires.