\chapter*{Introduction}
\label{chap:introduction}
\phantomsection\addcontentsline{toc}{chapter}{\nameref{chap:introduction}} 

Lors d'une entrevue en 2018, le président de la multinationale Siemens 
mentionnait que\footnote{Traduction libre.} "Les données sont le nouveau pétrole, ou or,
du 21ème siècle - la matière première sur laquelle nos économies,
sociétés et démocraties sont de plus en plus bâties."
Son affirmation, partagée par bon nombre d'experts dans le domaine de la
technologie, décrit ce que l'on appelle communément l'ère du \textit{Big Data}.
Cette appellation vise à décrire la quantité immense de données maintenant 
générées et collectées par le traffic internet ou encore les appareils de 
l'Internet des ObjeTs (IoT), pour n'en nommer que quelques-uns.

Dans la panoplie de données à disposition se trouvent les données textuelles, 
issues par exemple d'interactions sur des réseaux sociaux ou encore de documents 
numériques.
Ces données textuelles présentent une opportunité d'affaires en or à qui saura
les utiliser.
En effet, des traitements requérant habituellement l'avis d'un expert humain 
peuvent se voir automatiser par des modèles entraînés à reproduire le comportement
des experts sur les données disponibles.
Il suffit de penser aux cas des agents conversationnels pour le service à 
la clientèle ou des détecteurs automatiques de contenus toxiques 
sur les sites webs pour réaliser l'immense potentiel apporté par l'abondance
des données textuelles.

Ce ne sont toutefois pas tous les domaines utilisant des données textuelles 
qui peuvent être automatisables en entièreté.
Certains domaines, notamment en assurance, requièrent explicitement l'intervention 
d'un humain pour des raisons de législation.
Comment alors faire profiter du grand afflux de données dans ces systèmes où l'humain - et
sa capacité de traitement limitée - sont essentiels ?

Une option qui peut s'avérer facilement attrayante est l'idée de condenser un ou plusieurs
documents en un texte concis contenant seulement l'information requise pour procéder à la
tâche à exécuter.
On dit alors que l'on fait appel à un système de génération automatique de résumés de textes.
Une variante particulièrement intéressante de la génération de résumé est la variante 
qui consiste à bâtir un résumé en sélectionnant des phrases du texte original.
Les systèmes de génération de ces résumés, dits résumés extractifs, 
sont le sujet d'étude principal du présent document.

Plus particulièrement, on s'intéresse à la formulation 
de la génération de résumé comme un problème de bandit \citep{Robbins:1952}.
Les problèmes de bandit sont des problèmes d'interaction séquentielle
entre un apprenant et un environnement, où l'apprenant sélectionne 
des actions dans le but de maximiser les récompenses 
qui leur sont assignées par l'environnement.
La résolution de tout problème de bandit est basée sur 
l'identification rapide par l'agent d'actions près 
de l'action optimale sur l'environnement.
En voyant la génération de résumé comme un bandit, on souhaite ainsi bâtir 
des systèmes de génération de résumés qui apprendront rapidement à générer 
des résumés de haute qualité grâce à une exploration efficace du vaste espace de résumés 
possibles. 


\section*{Objectifs}

Pour analyser la viabilité des approches par bandit pour l'entraînement 
de modèles de génération de résumés, ce document souhaite accomplir les objectifs suivants:

\begin{itemize}
      \item Présenter BanditSum \citep{dong2018banditsum}, une approche 
            modélisant la génération de résumés pour un ensemble de documents
            comme un problème de bandit contextuel \citep{contextual_bandits}.
            Avec la formulation contextuelle retenue, BanditSum 
            parvient à entraîner un système de génération de résumés 
            en optimisant l'espérance de sa performance sur un 
            ensemble de documents.
      \item Aborder la génération du résumé d'un seul document comme 
            un problème de bandit combinatoire \citep{pmlr-v28-chen13a}.
            Particulièrement, on s'intéresse à comment la formulation 
            en bandit combinatoire peut être utilisée pour entraîner un système 
            de génération résumés.   
      \item Exploiter les similarités entre les phrases d'un même 
            document pour formuler la génération du résumé d'un 
            document comme un bandit linéaire combinatoire \citep{NEURIPS2018_207f8801}.
            Spécifiquement, c'est l'application de cette formulation linéaire
            à un algorithme d'entraînement de modèle de génération 
            et son impact sur la rapidité de convergence auxquelles on s'intéresse.
      \item Comparer la performance des approches par bandit 
            aux modèles issus de l'état de l'art et à l'heuristique de
            référence Lead-3 \citep{10.5555/3298483.3298681} sur le jeu de données 
            du CNN/DailyMail \citep{hermann2015teaching}.
            Notamment, on désire savoir si les approches par bandit 
            permettent un entraînement plus rapide en nombre de documents
            et produisent des modèles qui génèrent des résumés de meilleure qualité.
\end{itemize}

\section*{Structure du mémoire}

Ce mémoire débute par une introduction aux concepts de base essentiels à la compréhension
du reste du document au chapitre \ref{chap:prerequis}.
S'ensuit au chapitre \ref{chap:bandit_contextuel} une description détaillée 
de la problématique à l'étude ainsi que la présentation d'un algorithme 
basé sur le bandit contextuel: BanditSum.
Au chapitre \ref{chap:bandit_combi}, la formulation en bandit combinatoire est employée
afin d'obtenir des cibles riches pour l'entraînement de modèles de génération de résumé.
Enfin, le chapitre \ref{chap:bandit_combi_lin} explore l'utilisation du bandit linéaire 
combinatoire pour générer des cibles riches encore une fois, mais qui exploitent les similarités 
entre les phrases d'un document pour une plus grande efficacité.