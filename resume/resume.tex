\chapter*{Résumé}               % ne pas numéroter
\label{chap:resume}             % étiquette pour renvois
\phantomsection\addcontentsline{toc}{chapter}{\nameref{chap:resume}} % inclure dans TdM

\begin{otherlanguage*}{french}

Ce mémoire aborde l'utilisation des méthodes par bandit pour résoudre 
la problématique de l'entraînement de modèles de générations de résumés extractifs.
Les modèles extractifs, qui bâtissent des résumés en sélectionnant des phrases d'un 
document original, sont difficiles à entraîner car le résumé cible 
correspondant à un document n'est habituellement pas constitué de 
manière extractive.
C'est à cet effet que l'on propose de voir la production de résumés
extractifs comme différents problèmes de bandit, lesquels sont 
accompagnés d'algorithmes pouvant être utilisés pour l'entraînement.

On commence ce document en présentant BanditSum, une approche tirée 
de la litérature et qui voit la génération des résumés 
d'un ensemble de documents comme un problème de bandit 
contextuel.
Ensuite, on introduit CombiSum, un nouvel algorithme 
qui formule la génération du résumé d'un 
seul document comme un bandit combinatoire.
En exploitant la formule combinatoire,
CombiSum réussit à incorporer la notion du potentiel 
extractif de chaque phrase à son entraînement.
Enfin, on propose LinCombiSum, la variante linéaire de CombiSum 
qui exploite les similarités entre les phrases d'un document 
et emploie plutôt la formulation en bandit linéaire combinatoire.

\end{otherlanguage*}
